\documentclass[12pt,letterpaper]{book}
\usepackage[margin=0.9in]{geometry}
\usepackage[utf8]{inputenc} 
\usepackage[english]{babel}
\usepackage{graphicx}
\usepackage{amsmath, amsfonts, amssymb, amsthm, thmtools}
\usepackage{braket} 
\usepackage{relsize} 
\usepackage{float}
\usepackage{mathtools}
\usepackage{lmodern} 
\usepackage[T1]{fontenc}
\usepackage{fancyhdr}
\usepackage[dvipsnames]{xcolor} % Colors, use dvipsnames for more color options
\usepackage{framed} % Fancy leftbar
\usepackage[normalem]{ulem}
\usepackage{tikz-cd} % Diagrams
\usepackage{tikz} % General purpose graphics
\usepackage{tikz-3dplot}
\usepackage[most]{tcolorbox} % For theorem boxing
\usepackage{bm} % For better bold math font
\usepackage{old-arrows} 
\usepackage[usestackEOL]{stackengine}
\usepackage{sectsty}
\usepackage[hyperfootnotes=false]{hyperref} % For clickable table of contents
\usepackage[linesnumbered, ruled, boxruled]{algorithm2e} % Fancy algorithm presentation
\usepackage{calc} % For simpler calculation - used for spacing the index letter headings correctly
\usepackage{makeidx} % Required to make an index
\usepackage{verbatim}
\usepackage{titlesec} % Allows customization of titles
\usepackage{enumitem} % Customize lists
\usepackage{booktabs} %  Required for nicer horizontal rules in tables
\usepackage{eso-pic} %  Required for specifying an image background in the title page
\usepackage{titletoc, tocloft, pgffor}
\usepackage[titletoc]{appendix} % For appendix
\usepackage{braids}
\usepackage{adjustbox} % Very useful boxing environment
\usepackage{setspace} % Variable line spacing
\usepackage{soul}

%----------------------------------------------------------------------------------------
%	Global Document Settings
%----------------------------------------------------------------------------------------
\setcounter{section}{1} % Avoid counting from zero
\linespread{1.1} % Set line spacing
\graphicspath{{pictures/}}% Set graphics directory
\hypersetup{ % Set hyperref settings
    colorlinks,
    citecolor=black,
    filecolor=black,
    linkcolor=black,
    urlcolor=black
}
\setitemize{noitemsep,topsep=0.5em, parsep=0.5em ,partopsep=0pt}
\setlist{nolistsep} %  Reduce spacing between bullet points and numbered lists

%----------------------------------------------------------------------------------------
%	TikZ Settings
%----------------------------------------------------------------------------------------
\usetikzlibrary{
    arrows, 
    arrows.meta, 
    braids, 
    calc, 
    shapes.geometric,
    decorations.markings,
    decorations.pathreplacing, 
    intersections,
    hobby
}
% Set stealth arrows in TikzCD
\tikzcdset{
    arrow style = tikz, diagrams={>={Stealth[scale=1]}}
} 
% Set stealth arrows in TikZ
\tikzset{
    >={Stealth[scale=1]}
} 
\def\tikzpicc{tikzpicture}

%----------------------------------------------------------------------------------------
%	Math Commands
%----------------------------------------------------------------------------------------
% First we need to overwrite some existing commands. 
\let\*\relax 
\let\path\relax 
\let\ker\relax 
\let\im\relax
\let\hom\relax
\let\Set\relax
\let\top\relax
\let\ll\relax
%% Various Math Operators
\DeclareMathOperator{\im}{\rm{Im}}
\DeclareMathOperator{\id}{\mathrm{id}}
\DeclareMathOperator{\ob}{\mathrm{Ob}}
\DeclareMathOperator{\*}{ \mathbin{*} }
\DeclareMathOperator{\path}{\mathrm{Path}} 
\DeclareMathOperator{\nat}{\mathrm{Nat}}
\DeclareMathOperator{\dom}{\mathrm{Dom}}
\DeclareMathOperator{\cod}{\mathrm{Cod}}
\DeclareMathOperator{\gal}{\mathrm{Gal}}
\DeclareMathOperator{\aut}{\mathrm{Aut}}
\DeclareMathOperator{\cone}{\mathrm{Cone}}
\DeclareMathOperator{\fun}{\mathrm{Fun}}
\DeclareMathOperator{\proj}{\mathrm{proj}}
\DeclareMathOperator{\coker}{\mathrm{Coker}}
\DeclareMathOperator{\aend}{\mathrm{End}} %a is added because LaTeX does not allow commands to begin with "end."
\DeclareMathOperator{\spec}{\mathrm{Spec}}
\DeclareMathOperator{\ker}{\mathrm{Ker}}
\DeclareMathOperator*{\Lim}{\mathrm{Lim }}
\DeclareMathOperator*{\Colim}{\mathrm{Colim }}
\DeclareMathOperator{\hom}{\mathrm{Hom}}
\DeclareMathOperator{\Char}{\mathrm{char}} % We'd prefer \char, but... NEVER overwrite \char! TeX will die.
\DeclareMathOperator{\ll}{\mathcal{L}}

%% General Math Shortcuts
\newcommand{\zz}{\mathbb{Z}}
\newcommand{\rr}{\mathbb{R}}
\newcommand{\nn}{\mathbb{N}}
\newcommand{\qq}{\mathbb{Q}}
\newcommand{\normal}{\unlhd}
\newcommand{\join}{\vee}
\newcommand{\meet}{\wedge}
\renewcommand{\epsilon}{\varepsilon}
\renewcommand{\phi}{\varphi}
\renewcommand{\subset}{\subseteq}
\newcommand{\op}{^{\mathrm{op}}}

%% Category Shortcuts
\newcommand{\bb}{\mathcal{B}}
\newcommand{\cc}{\mathcal{C}}
\newcommand{\dd}{\mathcal{D}}
\newcommand{\ee}{\mathcal{E}}
\newcommand{\ff}{\mathcal{F}}
\newcommand{\mm}{\mathcal{M}}
\newcommand{\pp}{\mathcal{P}} 
\newcommand{\qqq}{\mathcal{Q}}
\newcommand{\vv}{\mathcal{V}}
\newcommand{\ww}{\mathcal{W}}
\renewcommand{\aa}{\mathcal{A}}
\renewcommand{\ee}{\mathcal{E}}
\renewcommand{\ss}{\mathcal{S}}
\renewcommand{\wp}{\mathcal{W}_{\mathrm{P}}}

% Category of Sets
\DeclareMathOperator{\Set}{\mathbf{Set}}
\DeclareMathOperator{\finset}{\mathbf{FinSet}}
\DeclareMathOperator{\finord}{\mathbf{FinOrd}}
% "Algebraic" categories
\DeclareMathOperator{\grp}{\mathbf{Grp}}
\DeclareMathOperator{\ab}{\mathbf{Ab}}
\DeclareMathOperator{\ring}{\mathbf{Ring}}
\DeclareMathOperator{\cring}{\mathbf{CRing}}
\DeclareMathOperator{\mon}{\mathbf{Mon}}
\DeclareMathOperator{\rmod}{\mathbf{-Mod}}
\DeclareMathOperator{\alg}{\mathbf{Alg}}
\DeclareMathOperator{\fld}{\mathbf{Fld}}
% Other Categories
\DeclareMathOperator{\open}{\mathbf{Open}}
\DeclareMathOperator{\top}{\mathbf{Top}}
\DeclareMathOperator{\Toph}{\mathbf{Toph}}
\DeclareMathOperator{\vect}{\mathbf{Vect}}
\DeclareMathOperator{\cat}{\mathbf{Cat}}

% A nice \to command
\renewcommand{\to}{\mathbin{\tikz[baseline] \draw[-{Stealth[length=5pt,width=4pt]}] (0pt,.6ex) -- (3.5ex,.6ex);}} 

% A nice isomorphism arrow
\newcommand{\isomarrow}{\mathrel{\setstackgap{S}{-0.5pt}\ensurestackMath{\Shortstack{\scriptstyle\sim\\ \longrightarrow}}}} 

% A nice \cdot symbol, which is bigger than \cdot but smaller than \bullet. 
\makeatletter % So that we access commands with "@" in them
\newcommand*\bcdot{\mathpalette\bcdot@{.6}}
\newcommand*\bcdot@[2]{\mathbin{\vcenter{\hbox{\scalebox{#2}{$\m@th#1\bullet$}}}}}
\DeclareMathOperator{\bigcdot}{\mathbin{\bcdot}}
\makeatother

% These are \Lim and \Colim with arrows underneath
\newcommand{\Arrowto}[1]{\parbox{#1}{\tikz{\draw[->](0,0)--(#1,0);}}}
\newcommand{\Arrowfrom}[1]{\parbox{#1}{\tikz{\draw[<-](0,0)--(#1,0);}}}
\DeclareMathOperator*{\Limto}{\Lim_{\Arrowto{0.8cm}}}
\DeclareMathOperator*{\Limfrom}{\Lim_{\Arrowfrom{0.8cm}}}
\DeclareMathOperator*{\Colimto}{\Colim_{\Arrowto{1cm}}}
\DeclareMathOperator*{\Colimfrom}{\Colim_{\Arrowfrom{1cm}}}

% To type out adjunctions
\newcommand{\adjunction}[4]{
    \begin{tikzcd}[ampersand replacement = \&] %\newcommand freaks out with & symbols
        #1
        \arrow[r, shift right = -0.5ex, "#2"]
        \&
        #3
        \arrow[l, shift right = -0.5ex, "#4"]
    \end{tikzcd}
}

%----------------------------------------------------------------------------------------
%	Non-Math Commands
%----------------------------------------------------------------------------------------

% Some colors
\definecolor{processBlue}{HTML}{E0F8FF}
\definecolor{mypurple}{HTML}{4D34A4}
\definecolor{navy}{RGB}{20,150,220}
\definecolor{purp}{HTML}{CDCBFF}  

\begin{document}
\pagestyle{empty} % no page numbers

\begin{center}

\begin{center}
    
        \begin{tikzpicture}[>=stealth]
            
            \tikzstyle{surface}=[left color=NavyBlue!30,right color=NavyBlue!60, 
            opacity = 0.8
            ]
            
            \tikzstyle{em_triangle}=[left color=Purple!30,right color=Purple!60,
            opacity = 0.8
            ]
            
            \tikzstyle{triangle}=[left color=Purple!30,right color=Purple!50,
            opacity = 0.8
            ]
            
            \tikzstyle{shadow}=[left color=gray!10,right color=gray!20, opacity = 0.8, rounded corners=0.5cm]
            \pgfmathsetmacro{\scale}{4}
            \pgfmathsetmacro{\zoffset}{-10}
            \pgfmathsetmacro{\yoffset}{-3}
            \pgfmathsetmacro{\xoffset}{-6}
        
            
            \coordinate (origin) at (-\xoffset, -\yoffset, -\zoffset);
        
            
            \coordinate (A) at (\scale -\xoffset, -\yoffset,-\zoffset);
            \coordinate (B) at (-\xoffset, \scale-\yoffset,-\zoffset);
            \coordinate (C) at (-\xoffset, -\yoffset,\scale-\zoffset);
        
            \draw[semithick, ->] (origin) --  (A);
            \draw[semithick, ->] (origin) --  (B);
            \draw[semithick, ->] (origin) --  (C);
    
            
            \pgfmathsetmacro{\axesshift}{8}
            \coordinate (a) at (\scale -\xoffset - \axesshift, -\yoffset,-\zoffset);
            \coordinate (b) at (-\xoffset -  \axesshift, \scale-\yoffset,-\zoffset);
            \coordinate (c) at (-\xoffset - \axesshift, -\yoffset,\scale-\zoffset);
            
            
            \coordinate (left_origin) at (-\xoffset - \axesshift, -\yoffset, -\zoffset);
            \draw[semithick, ->] (left_origin) --  (a);
            \draw[semithick, ->] (left_origin) --  (b);
            \draw[semithick, ->] (left_origin) --  (c);
    
            
            \pgfmathsetmacro{\trianglelength}{2.5}
    
            
            \draw (\trianglelength-\xoffset - \axesshift,-\yoffset,-\zoffset) 
            -- 
            (-\xoffset -\axesshift, \trianglelength-\yoffset, -\zoffset) 
            -- 
            (-\xoffset -\axesshift, -\yoffset, \trianglelength-\zoffset) -- cycle;
    
            
            \shade[triangle]
            (\trianglelength-\xoffset - \axesshift,-\yoffset,-\zoffset) 
            -- 
            (-\xoffset -\axesshift, \trianglelength-\yoffset, -\zoffset) 
            -- 
            (-\xoffset -\axesshift, -\yoffset, \trianglelength-\zoffset) -- cycle;
    
            
            
            \draw[->] ([xshift = -2cm, yshift = 1.5cm]a) 
            to[bend left] 
            ([xshift= 0.5cm, yshift = 4cm]C);
            \node at ([xshift = 0.5cm, yshift = 3.1cm]a) {$f$};
    
            
            \pgfmathsetmacro{\nlines}{22}
            
            \coordinate (A) at (0,0);
            \path (A);\pgfgetlastxy{\ax}{\ay} 
            \coordinate (B) at (3,-2);
            \path (B);\pgfgetlastxy{\bx}{\by}
            \coordinate (C) at (8,0);
            \path (C);\pgfgetlastxy{\cx}{\cy}
            \coordinate (D) at (5, 2);
            \path (D);\pgfgetlastxy{\dx}{\dy}
        
            
            \pgfmathsetmacro{\ABout}{-5}
            \pgfmathsetmacro{\ABin}{150}
            
            \pgfmathsetmacro{\BCout}{-5}
            \pgfmathsetmacro{\BCin}{150}
        
            \pgfmathsetmacro{\CDout}{150}
            \pgfmathsetmacro{\CDin}{-10}
            
            \pgfmathsetmacro{\DAout}{150}
            \pgfmathsetmacro{\DAin}{-10}
        
            
            \shade[shadow]
            ([xshift = 1cm, yshift = -2cm]A)
            --
            ([xshift = 0cm, yshift = -2cm]B) 
            --
            ([xshift = -1cm, yshift = -2cm]C)
            --
            ([xshift = 0cm, yshift = -2.7cm]D) 
            --
            
            cycle;
        
            
            \shade[surface] 
            (A)
            to[out=\ABout,in=\ABin] 
            (B) 
            to[out=\BCout,in=\BCin] 
            (C)
            to[out=\CDout,in=\CDin] 
            (D) 
            to[out=\DAout,in=\DAin]  cycle;
        
            
            \draw[line width = 0.05mm]
            (A)
            to[out=\ABout,in=\ABin] 
            (B) 
            to[out=\BCout,in=\BCin] 
            (C)
            to[out=\CDout,in=\CDin] 
            (D) to[out=155,in=\DAin] cycle;
        
            
            \foreach \n in {1,2, ..., \nlines}{
                \pgfmathsetmacro{\step}{\n/\nlines}
                \path (A) to[out=\ABout,in=\ABin] coordinate[pos=\step] (A\n) (B);
                \path (B) to[out=\BCout,in=\BCin] coordinate[pos=\step] (B\n) (C);
                \path (D) to[in=\CDout,out=\CDin] coordinate[pos=\step] (C\n) (C);
                \path (A) to[in=\DAout,out=\DAin] coordinate[pos=\step] (D\n) (D);
            }
        
            
            \foreach \n in {1,2,...,\nlines}{
                \path (A\n); \pgfgetlastxy{\ax\n}{\ay\n} 
                \path (B\n); \pgfgetlastxy{\bx\n}{\by\n} 
                \path (C\n); \pgfgetlastxy{\cx\n}{\cy\n} 
                \path (D\n); \pgfgetlastxy{\dx\n}{\dy\n} 
                \draw[line width = 0.05mm]
                (\bx\n,\by\n) to[out=\DAout,in=\DAin] (\dx\n,\dy\n);
                \draw[line width = 0.05mm]
                (\ax\n,\ay\n) to[out=\ABout,in=\ABin] (\cx\n,\cy\n);
            }
        
            
            \begin{scope}
                
                \clip ([xshift = 0.1cm]A5)
                to[out=\ABout,in=\ABin] 
                ([xshift = -0.5cm]B5) 
                to[out=\BCout,in=\BCin] 
                ([xshift = 0.1cm]C5)
                to[out=\CDout,in=\CDin] 
                cycle;
    
                
                \pgfmathsetmacro{\nlines}{22}
                
                \coordinate (A) at (0,0);
                \path (A);\pgfgetlastxy{\ax}{\ay} 
                \coordinate (B) at (3,-2);
                \path (B);\pgfgetlastxy{\bx}{\by}
                \coordinate (C) at (8,0);
                \path (C);\pgfgetlastxy{\cx}{\cy}
                \coordinate (D) at (5, 2);
                \path (D);\pgfgetlastxy{\dx}{\dy}
            
                
                \pgfmathsetmacro{\ABout}{-5}
                \pgfmathsetmacro{\ABin}{150}
                
                \pgfmathsetmacro{\BCout}{-5}
                \pgfmathsetmacro{\BCin}{150}
            
                \pgfmathsetmacro{\CDout}{150}
                \pgfmathsetmacro{\CDin}{-10}
                
                \pgfmathsetmacro{\DAout}{150}
                \pgfmathsetmacro{\DAin}{-10}
            
                
                \shade[em_triangle] 
                (A)
                to[out=\ABout,in=\ABin] 
                (B) 
                to[out=\BCout,in=\BCin] 
                (C)
                to[out=\CDout,in=\CDin] 
                (D) 
                to[out=\DAout,in=\DAin]  cycle;
            
                
                \draw[line width = 0.05mm]
                (A)
                to[out=\ABout,in=\ABin] 
                (B) 
                to[out=\BCout,in=\BCin] 
                (C)
                to[out=\CDout,in=\CDin] 
                (D) to[out=\DAout,in=\DAin] cycle;
            
                
                \foreach \n in {1,2, ..., \nlines}{
                    \pgfmathsetmacro{\step}{\n/\nlines}
                    \path (A) to[out=\ABout,in=\ABin] coordinate[pos=\step] (A\n) (B);
                    \path (B) to[out=\BCout,in=\BCin] coordinate[pos=\step] (B\n) (C);
                    \path (D) to[in=\CDout,out=\CDin] coordinate[pos=\step] (C\n) (C);
                    \path (A) to[in=\DAout,out=\DAin] coordinate[pos=\step] (D\n) (D);
                }
            
                
                \foreach \n in {1,2,...,\nlines}{
                    \path (A\n); \pgfgetlastxy{\ax\n}{\ay\n} 
                    \path (B\n); \pgfgetlastxy{\bx\n}{\by\n} 
                    \path (C\n); \pgfgetlastxy{\cx\n}{\cy\n} 
                    \path (D\n); \pgfgetlastxy{\dx\n}{\dy\n} 
                    \draw[line width = 0.05mm]
                    (\bx\n,\by\n) to[out=\DAout,in=\DAin] (\dx\n,\dy\n);
                    \draw[line width = 0.05mm]
                    (\ax\n,\ay\n) to[out=\ABout,in=\ABin] (\cx\n,\cy\n);
                }
        
            \end{scope}
        \end{tikzpicture}

        \emph{A 2-simplex gets embedded into a manifold in $\rr^3$.}
    \end{center}

\end{center}

\end{document}
