\chapter{Abelian Categories}

Abelian categories are generalizations of the structure which can be 
found in the category of abelian groups \textbf{Ab}. This may be obvious from the name;
what is nontrivial, however, is how to preserve the nice structure of the category
without specific reference to the elements themselves. It turns out this is possible, 
but is generally not the way we think about \textbf{Ab}. This is the aim of this chapter. 

\section{Preadditive Categories}      

Consider two abelian groups $(G, +)$ and $(H, \cdot)$ of \textbf{Ab}. 
Recall from group theory that we can turn the
set $\hom(G, H)$ into an abelian group $(\hom(G, H), *)$ as follows.
Given $\phi, \psi: G \to H$, we can create another 
group homomorphism $\phi * \psi: G \to H$ where 
\begin{align*}
    (\phi * \psi)(g) = \phi(g) \cdot \psi(g).
\end{align*}
Observe that this is in fact a group homomorphism: if $g, g' \in G$, then 
\begin{align*}
    (\phi * \psi)(g + g') &= \phi(g + g') \cdot \psi(g + g')\\
    &=\phi(g) \cdot \phi(g') \cdot \psi(g) \cdot \psi(g')\\
    &\textcolor{Red}{=} \phi(g) \cdot \psi(g) \cdot \phi(g') \cdot \psi(g')\\
    &= (\phi * \psi)(g) \cdot (\phi *\psi)(g').
\end{align*}
In the third step we utilized the fact that $(H, \cdot)$ is abelian. 
Thus $(\hom(G, H), *)$ is not necessarily a group unless $H$ is an 
abelian group.
Therefore, this construction doesn't extend to \textbf{Grp}.

At this point, your category-theory-voice in your head is probably asking:
\begin{center}
    \begin{minipage}{0.8\textwidth}
        \textcolor{NavyBlue}{If $H$ is an abelian group, can we create a functor $F_H: \textbf{Ab} \to \textbf{Ab}$ 
        where $G \mapsto \hom(G, H)$?}
    \end{minipage}
\end{center}
The answer is yes; the functor is actually contravariant, for suppose we have a group homomorphism
\[
    \phi: G \to G'.
\]
Then define the function
\[
    F_H(\phi): \hom(G', H) \to \hom(G, H) 
\]    
where
\begin{statement}{NavyBlue!10}
\[
    F_H(\phi)(\psi: G' \to H) = \psi \circ \phi: G \to H.
\]    
\end{statement}
To verify functoriality, we have to check that this function is actually a group 
homomorphism. Towards that goal, consider $\psi, \sigma: G \to H$. Then 
observe that for any $g \in G$,
\begin{align*}
    F_H(\phi)(\psi + \sigma)(g) &= \phi(\psi(g) + \sigma(g))\\
     &= \phi(\psi(g)) + \phi(\sigma(g))\\
     &= F_H(\phi)(\psi)(g) + F_H(\phi)(\psi)(g)
\end{align*}
which verifies that $F_H(\phi)$ is a group homomorphism. Therefore, we see that 
$F_H: \textbf{Ab} \to \textbf{Ab}$ is in fact a functor.

Now your category-theory-voice should be asking: 
\begin{center}
    \begin{minipage}{0.8\textwidth}
        \textcolor{NavyBlue}{If $G$ is an abelian group, can we \emph{also} create 
        a functor $F^G: \textbf{Ab} \to \textbf{Ab}$ where $H \mapsto \hom(G, H)$?}
    \end{minipage}
\end{center}
One can easily show that the answer is yes. In this direction, the functor is covariant. That 
is, for $\psi: H \to H'$, we have that 
\[
    F^G(\psi): \hom(G, H) \to \hom(G, H')
\]
where 
\begin{statement}{NavyBlue!10}
\[
    F^G(\psi)(\phi: G \to H) = \psi \circ \phi: G \to H'.
\]
\end{statement}
Note that for our functors, we have that
\[
    F_H(G) = F^G(H).
\]
This is \emph{bifunctor-ish}. Therefore, our category theory voice is now 
asking: 
\begin{center}
    \begin{minipage}{0.8\textwidth}
        \textcolor{NavyBlue}{Do we have a bifunctor 
        $F: \textbf{Ab}\times \textbf{Ab} \to \textbf{Ab}$ on our hands, where 
        $F(G, H) = \hom(G, H)$? }
    \end{minipage}
\end{center}
To see if this answer is true, we ought to be able to show that, given 
$\phi: G' \to G$ and $\psi: H \to H'$, the diagram 
\begin{center}
    \begin{tikzcd}[column sep = 1.5cm, row sep = 1.5cm]
        F_{H}(G) \arrow[r, "F_H(\phi)"]
        \arrow[d, swap, "F^{G}(\phi)"]
        &
        F_H(G)
        \arrow[d, "F^{G'}(\psi)"]
        \\
        F^{G'}(H) \arrow[r, swap, "F_H(\psi)"]
        &
        F^{G'}(H)
    \end{tikzcd}
    $=$
    \begin{tikzcd}[column sep = 1.5cm, row sep = 1.5cm]
        \hom(G,H) 
        \arrow[r, "(-) \circ \phi"]
        \arrow[d, swap, "\psi \circ (-)"]
        &
        \hom(G', H)
        \arrow[d, "\psi \circ (-)"]
        \\
        \hom(G, H') 
        \arrow[r, swap, "(-) \circ \phi"]
        &
        \hom(G', H')
    \end{tikzcd}
\end{center} 
is commutative. The above diagram is in fact commutative since function composition 
is associative! That is, given $\sigma: G \to H$, observe that going right and then down 
gives
\begin{align*}
    \psi \circ (\sigma \circ \phi) 
\end{align*}
while going down and then right gives 
\begin{align*}
    (\psi \circ \sigma) \circ \phi.
\end{align*}
Hence we have commutativity of the above diagram, and we therefore have a 
true bifunctor $F: \textbf{Ab}\times\textbf{Ab} \to \textbf{Ab}$ where 
\[
    F(G,H) = \hom(G, H).
\]
\textcolor{NavyBlue}{What this really shows is that $\hom(-, -)$ is a functor; specifically, a bifunctor. 
So while we typically think of $\hom(G, H)$ as a set, it had hidden functorial properties. 
Thus what makes \textbf{Ab} special is that plugging in abelian groups outputs an 
abelian group, and this is not the case with other constructions (e.g. \textbf{Grp}).}

Let us now consider a new observation of $\textbf{Ab}$. For any triple of abelian groups
\[
    (G, \star), (H, +), (K, \cdot)
\]
we can create abelian groups 
\begin{align*}
    &\big(\hom(G, H), +'\big) &&(\phi_1 +' \phi_2)(g) = \phi_1(g) + \phi_2(g) \\
    &\big(\hom(H, K), \cdot'\big) &&(\psi_1 \cdot' \psi_2)(h) = \psi_1(h)\cdot \psi_2(h)  \\
    &\big(\hom(G, K), *\big) &&(\sigma_1 * \sigma_2)(g)= \sigma_1(g) \cdot \sigma_2(g) \\
\end{align*}
where $\phi_i \in \hom(G, H), \psi_i \in \hom(H, K)$ and $\sigma_i \in \hom(G, K)$ 
for $i = 1, 2$. Now since these are abelian groups in \textbf{Ab}, there is a composition operator 
\[
    \circ: \hom(G, H)\times \hom(H,K) \to \hom(G, K)
\]
where $\circ(\phi: G \to H, \psi: H \to K ) \mapsto \psi \circ \phi: G \to K$. 
However, we now run into a problem where our operators might not play nicely with each other. Specifically, is 
it true that 
\[
    \psi \circ (\phi_1 +' \phi_2) = (\psi \circ \phi_1) * (\psi \circ \phi_2)
\]
or 
\[
    (\psi_1 \cdot' \psi_2) \circ \phi = (\psi_1 \circ \phi) * (\psi_2 \circ \phi)?
\]
For the first case, the answer is yes. Observe that
\begin{align*}
    \psi \circ (\phi_1 +' \phi_2)(g) &= \psi(\phi_1(g) + \phi_2(g))\\
    &= \psi(\phi_1(g) + \phi_2(g))\\
    &= \psi(\phi_1)(g) \cdot \psi(\phi_2)(g)\\
    &= \big((\psi \circ \phi_1) * (\psi \circ \phi_2)\big)(g).
\end{align*}

\textcolor{NavyBlue}{The reason we have linearity here is because of the way we defined 
the \textbf{group operations} on the homsets. The definition of these operations 
is intuitively correct, but we get accidentally get an extra bonus of obtaining linearity 
so that we don't have to worry about the above equations not holding.}

In order to mimic this behavior, we abstract this into a category to define 
a \textbf{Ab}-category. 

\begin{definition}
    An \textbf{Ab}-category or \textbf{Preadditive Category} is a 
    category $\mathcal{C}$ such that, for each pair of objects $A, B$, 
    there exists an abelian group operation $+$ on the set $\hom(A, B)$ such 
    that 
    \begin{align*}
        &\circ: \hom(A, B)\times \hom(B, C) \to \hom(A, C)\\
        &(f, g) \mapsto g \circ f
    \end{align*}
    is bilinear. What we mean by bilinear is that, given morphisms $f, g: A \to B$ and $h, k: B \to C$, 
    we have that 
    \begin{statement}{Red!10}
        \begin{align_topbot}
            (h + k) \circ f = h \circ f + k \circ f\\
            h \circ (g + f) = h \circ g + h \circ f.
        \end{align_topbot}
    \end{statement}
\end{definition}

\textcolor{NavyBlue}{Note that since we demand that $\hom_{\cc}(A, B)$ always 
be a group, we see that any category such that $\hom_{\cc}(A, B) = \varnothing$ 
can never be an abelian group. A group always requires the existence of an identity; 
a demand that an empty set can never meet}. Therefore, as an example, any discrete 
category cannot be a preadditive category because all of the nontrivial homsets 
are empty. 

As we demonstrated building up to this definition, \textbf{Ab} is a trivial example 
of a preadditive category. A less trivial example is $\textbf{Vect}_K$
where $K$ is a field, but this is nearly automatic since this takes advantage 
of the fact that vector spaces have their own hidden abelian group structure. 

\begin{example}
    Suppose $\cc$ is a one object category $R$ which is also preadditive. Then 
    this means that we have two binary operations $+$ and $\circ$ on the 
    abelian group $\hom_{\cc}(R, R)$ 
    such that 
    \begin{align*}
        (h + k) \circ f = h \circ f + k \circ f\\
        h \circ (g + f) = h \circ g + h \circ f.
    \end{align*}
    However, this is simply a ring! The addition is the ring addition, while the 
    ring multiplication is given by composition.
    Conversely, a ring regarded as the homset of a 
    one object category can be defined to be an abelian category. This is because 
    when regarding a group as a one object category, the group operation becomes the 
    composition operation. Thus adding the extra axiom of an addition bilinear operation 
    grants us that the category is preadditive.
\end{example}


\begin{example}
    Let $\cc$ be a preadditive category. Then 
    $\cc\op$ is also a preadditive category. To demonstrate this, we know that 
    every pair of objects $A, B \in \cc$ gives rise to a group $(\hom_{\cc}(A, B), +)$
    for some operation $+$. This allows us to place a group structure $+'$ on 
    $\hom_{\cc\op}(B, A)$ where for two $f\op, g\op: B \to A$ in $\cc\op$,
    \[
        f\op +' g\op = (f + g)\op.
    \]
    That is, we rely on the preexisting group operation $+$ from  $\hom_{\cc}(A, B)$. 
    Given that the composition operator of $\cc\op$ is $\circ\op$, we can check that 
    this satisfies the bilinearity conditions of $\circ\op$.
    Suppose $h\op, k\op : B \to A$ are two morphisms in $\hom(B, A)$ which 
    are composable with some $f\op$. Then 
    \begin{align*}
        (h\op +' k\op)\circ\op f\op 
        = 
        (h + k)\op \circ \op f\op
        &= 
        f \circ (h + k)\\
        &= f \circ h + f \circ k \\
        &= h\op \circ\op f\op +' k\op \circ\op f\op. 
    \end{align*}
    The other direction can be verified dually, so that the the group operation 
    $+'$ distributes bilinearly over $\circ\op$. Therefore, $\cc\op$ is a preadditive 
    category.
\end{example}

\begin{example}
    If $\cc$ is preadditive, then the functor category $\cc^J$ is preadditive.
    To demonstrate this, consider the hom-set $\hom_{\cc^J}(F, G)$ between two 
    functors $F, G: J \to \cc$. Now consider
    two natural transformations $\eta, \epsilon \in \hom_{\cc^J}(F, G)$. Then 
    for each $f \in \hom_{\cc}(A, B)$, the familiar diagram commutes. 
    \begin{center}
        \begin{tikzcd}[column sep = 1.5cm, row sep = 1.5cm]
            A \arrow[d, "f"]\\
            B
        \end{tikzcd}
        \hspace{1cm}
        \begin{tikzcd}[column sep = 1.5cm, row sep = 1.5cm]
            F(A)
            \arrow[r, shift right = -0.5ex, "\eta_A"]
            \arrow[r, swap, shift right = 0.5ex, "\epsilon_A"]
            \arrow[d, swap, "F(f)"]
            &
            G(A)
            \arrow[d, "G(A)"]
            \\
            F(B)
            \arrow[r, shift right = -0.5ex, "\eta_B"]
            \arrow[r, swap, shift right = 0.5ex, "\epsilon_B"]
            &
            G(B)
        \end{tikzcd}
    \end{center}
    This diagram tells us that $G(f) \circ \eta_A  = \eta_B \circ F(f)$ 
    and that $G(f) \circ \epsilon_A  = \epsilon_B \circ F(f)$. However, since 
    $\cc$ is abelian, we can combine these morphisms and add both equations 
    to get 
    \[
        G(f) \circ \eta_A + G(f) \circ \epsilon_A = \eta_B \circ F(f) + \epsilon_B \circ 
        F(f) 
        \implies G(f) \circ (\eta_A + \epsilon_A) = (\eta_B + \epsilon_B) \circ F(f).
    \]  
    Hence the diagram below 
    \begin{center}
        \begin{tikzcd}[column sep = 1.5cm, row sep = 1.5cm]
            A \arrow[d, "f"]\\
            B
        \end{tikzcd}
        \hspace{1cm}
        \begin{tikzcd}[column sep = 1.5cm, row sep = 1.5cm]
            F(A)
            \arrow[r, "\eta_A + \epsilon_A"]
            \arrow[d, swap, "F(f)"]
            &
            G(A)
            \arrow[d, "G(A)"]
            \\
            F(B)
            \arrow[r, swap, "\eta_B + \epsilon_B"]
            &
            G(B)
        \end{tikzcd}
    \end{center}
    commutes. Therefore, using the group product of $(\hom_{\cc}(F(A), F(B)), +)$,
    we've derived a new natural transformation from $F$ to $G$ using $\eta$ and $\epsilon$ 
    in $\hom_{\cc^J}(F, G)$. This allows us to endow the homset $\hom_{\cc^J}(F, G)$ 
    with the operation $+'$  defined so that for two $\eta, \epsilon \in \hom_{\cc^J}(F, G)$,
    $\eta + ' \epsilon$ is the natural transformation where 
    for each object $A$
    \[
        (\eta +' \epsilon)_A = \eta_A + \epsilon_A
    \]
    where $+$ is the group operation on $(\hom_{\cc}(F(A), G(A)), +)$. The fact that 
    this distributes bilinearly over the composition operator is inherited from 
    $\cc$, and can easily be verified, so that $\cc^J$ is preadditive.
\end{example}

\begin{example}
    Let $\cc$ be a category such that for every pair of objects 
    $A, B$, the hom set $\hom_{\cc}(A, B)$ is nonempty. Then we can create the category 
    $\text{PreAdd}(\cc)$ where the objects are the same as $\cc$, except
    each $\hom_{\text{PreAdd}(\cc)}(A, B)$ is now regarded as the free 
    abelian group generated by the elements of $\hom_{\cc}(A, B)$. This results 
    in a preadditive category if we force the composition operator $\circ'$
    in $\text{PreAdd}(\cc)$ to be bilinear. This forcing makes sense in our case since, 
    if $\sum_{f \in \hom_{\cc}(A,B)}n_f f, \sum_{f \in \hom_{\cc}(A,B)}n'_f f$ 
    are two arbitrary elements in $\hom_{\text{PreAdd}(\cc)}(A, B)$, 
    then if $\sum_{k \in \hom_{\cc}(B,C)}m_k k \in \hom_{\text{PreAdd}(\cc)}(B, C)$ for some object $C$, 
    where  $n_f, n'_f, m_k$ are all nonzero for finitely many integers,
    then 
    \begin{align*}
        &\sum_{k \in \hom_{\cc}(B,C)}m_k k \circ'
        \left( \sum_{f \in \hom_{\cc}(A,B)}n_f f +
        \sum_{f \in \hom_{\cc}(A,B)}n'_f f
        \right)\\
        &= 
        \sum_{f \in \hom_{\cc}(A,B)}\sum_{k \in \hom_{\cc}(B, C)}n_f \cdot m_k(k \circ f) 
        +
        \sum_{f \in \hom_{\cc}(A,B)}\sum_{k \in \hom_{\cc}(B, C)}n'_f \cdot m_k(k\circ f)
    \end{align*}
    and the above last expression is in fact an element of $\hom_{\text{PreAdd}(\cc)}(A, C)$. 
\end{example}



\newpage
\section{Additive Categories}

Let $G$ and $H$ be abelian groups in \textbf{Ab}. A natural question to ask in any 
given category is if a binary product such at $G \times H$ exists in the category. In our case, 
the answer is yes; it is the \textbf{direct sum} $G \oplus H$. The direct sum satisfies the 
universal property 
\begin{center}
    \begin{tikzcd}[column sep = 1.5cm, row sep = 1.5cm]
        &\arrow[dr, "\phi"] K \arrow[dl, swap, "\psi"]
        \arrow[d, dashed, "u"]
        &\\
        G & 
        \arrow[l, "\pi_G"]
        G \oplus H
        \arrow[r, swap, "\pi_H"]
        &
        H
    \end{tikzcd} 
\end{center}
Here, $K$ is a third group, $\phi$ and $\psi$ are arbitrary group homomorphisms, and $\pi_G, \pi_H$ are the natural projection morphisms. 
Interestingly, this object also satisfies the universal property 
\begin{center}
    \begin{tikzcd}[column sep = 1.5cm, row sep = 1.5cm]
        &\arrow[<-, dr, "\phi"] K \arrow[<-, dl, swap, "\psi"]
        \arrow[<-, d, dashed, "u"]
        &\\
        G & 
        \arrow[<-, l, "i_G"]
        G \oplus H
        \arrow[<-, r, swap, "i_H"]
        &
        H
    \end{tikzcd}
\end{center}
Here $i_G$ and $i_H$ are the natural injections, e.g. $i_G(g) = g \otimes e_H$. 
However, this implies that $G \oplus H$ is a coproduct!
What this implies is 
that \textcolor{NavyBlue}{product and coproducts coincide in \textbf{Ab}}.
This is actually a pretty remarkable property because this isn't the case even 
in nice categories. For example, in \textbf{Set}, products and coproducts are 
definitely distinct.

\begin{center}
    Why is this the case? 
\end{center}

\begin{proposition}
    Let $\cc$ be a preadditive category with a zero object $z$. 
    Then for any objects $A, B \in \cc$, the following are equivalent
    \begin{itemize}
        \item[$(i)$] $A \times B$ exists 
        \item[$(ii)$] $A \amalg B$ exists 
    \end{itemize}
    Moreover, there exists an isomorphism 
        \[
            \prod_{i \in \lambda} A_i \isomarrow \coprod_{i \in \lambda}A_i
        \] 
        for any objects $A_i \in \cc$. 
\end{proposition}

\begin{prf}
        We only demonstrate one direction because the proof is self-dual. 

        Suppose $A \times B$ exists. Then then if $C$ is an object equipped 
        with morphisms $f: C \to A$ and $g: C \to B$, the following diagram 
        must hold. 
        \begin{center}
            \begin{tikzcd}[column sep = 1.5cm, row sep = 1.5cm]
                & C 
                \arrow[dr, "g"]
                \arrow[dl, swap, "f"]
                \arrow[d, dashed, "h"]
                & 
                \\
                A &
                A \times B 
                \arrow[l, "\pi_A"]
                \arrow[r, swap, "\pi_B"]
                &
                B
            \end{tikzcd}
        \end{center}
        Equip $A$ with the morphisms $1_A: A \to A$ and the unique  
        zero morphism $\emptyset_A^B: A \to B$. Then there exists a unique 
        $i_A: A \to A \times B$ such that the diagram commutes. 
        \begin{center}
            \begin{tikzcd}[column sep = 1.5cm, row sep = 1.5cm]
                & A
                \arrow[dr, "\emptyset_A^B"]
                \arrow[dl, swap, "1_A"]
                \arrow[d, dashed, "i_A"]
                & 
                \\
                A &
                A \times B 
                \arrow[l, "\pi_A"]
                \arrow[r, swap, "\pi_B"]
                &
                B
            \end{tikzcd}
        \end{center} 
        Symmetrically, equip $B$ with the unique zero morphism $\emptyset_B^A: B \to A$
        and $1_B: B \to B$. Then there exists a unique $i_B: B \to A\times B$ such that the 
        diagram commutes.
        \begin{center}
            \begin{tikzcd}[column sep = 1.5cm, row sep = 1.5cm]
                & B
                \arrow[dr, "1_B"]
                \arrow[dl, swap, "\emptyset_B^A"]
                \arrow[d, dashed, "i_B"]
                & 
                \\
                A &
                A \times B 
                \arrow[l, "\pi_A"]
                \arrow[r, swap, "\pi_B"]
                &
                B
            \end{tikzcd}
        \end{center} 
        Now we'll demonstrate that we have a coproduct structure on our hands. 
        To do this, suppose we have an object $C$ equipped with morphisms 
        $f: A \to C$ and $g: B \to C$. Then we can construct a morphism $h$ such that the following 
        diagram commutes. 
        \begin{center}
            \begin{tikzcd}[column sep = 1.5cm, row sep = 1.5cm]
                A \arrow[dr, swap, "f"] \arrow[r, "i_A"] 
                & A \times B  
                \arrow[d, dashed, "h"]
                & B \arrow[l, swap, "i_B"]
                \arrow[dl, "g"]\\
                & C & 
            \end{tikzcd}
        \end{center}
        Observe that $h = f\circ \pi_A + g \circ \pi_B$ suffices, where 
        $+$ is the group operation on the abelian group $\hom(A \times B, C)$.
        Observe that 
        \begin{align*}
            h \circ i_A &= (f\circ \pi_A + g \circ \pi_B) \circ i_A\\
            &= f \circ (\pi_A \circ i_A) + g \circ (\pi_B \circ i_A)\\
            &= f.
        \end{align*}
        Similarly, 
        \begin{align*}
            h \circ i_B &= (f\circ \pi_A + g \circ \pi_B) \circ 1_B\\
            &= f \circ (\pi_A \circ 1_B) + g \circ (\pi_B \circ 1_B)\\
            &= g.
        \end{align*}
        Hence the commutativity of the above diagram holds; therefore, we see 
        that $A \times B$ is also a coproduct. Finally, recall that if two 
        distinct objects satisfy the same universal property, they are necessarily isomorphic; 
        therefore the existence of an isomorphism between the product and coproduct is immediate. 
\end{prf}
\textcolor{NavyBlue}{The above proof is not hard, but it's also not trivial. 
Moreover, there are three extremely important ingredients we utilized that demonstrate that 
the assumptions we've made so far are actually necessary and useful.}
\begin{itemize}
    \item This proof does not hold for a category without a zero object because 
    there is not, in general, an obviously conceivable morphism to go from any two objects 
    $A$ and $B$. 
    \item Notice that calculating $h$ was only possible because we had an abelian 
    group operation. 
    \item  Finally, notice that we utilized bilinearity of the composition operator in order to 
    calculate $h \circ i_A$ and $h \circ i_B$ and thereby verify the universal property.
\end{itemize}

Therefore, all of our assumptions so far have been necessary and useful. And all of 
this now motivates the following definition. 

\begin{definition}
    Let $\cc$ be an abelian category.
    A \textbf{biproduct} of two objects $A, B$ of $\cc$ is an object 
    $A \otimes B$ which is both a product and coproduct.
    \\
    \textcolor{Purple}{Equivalently}, A biproduct is an object $A \oplus B$ 
    equipped with morphisms 
    \begin{align*}
        &\pi_A: A \oplus B \to A && i_A: A \to A \oplus B\\
        &\pi_B: A \oplus B \to B && i_B: B \to A \oplus B
    \end{align*}
    such that 
    \begin{itemize}
        \item[1.] $\pi_A \circ i_A = 1_A$
        \item[2.] $\pi_B \circ i_B = 1_B$
        \item[3.] $i_A \circ \pi_A + i_B \circ \pi_B = 1_{A\oplus B}$  
    \end{itemize}
\end{definition}

\begin{definition}
    An \textbf{Additive Category} is a preadditive category $\cc$ 
    such that finite biproducts exist. 
\end{definition}

\begin{definition}
    Consider the category $\textbf{Grp}$. 
\end{definition}



\newpage 
\section{Preabelian Categories}
In \textbf{Ab}, kernels and cokernels exists for every group homomorphism.
\\

First, recall their definitions. 
\begin{definition}
    Let $\phi: G \to H$ be a group homomorphism. Then a \textbf{kernel} is 
    an equalizer of $\phi: G \to H$ and $0: G \to H$, where $0$ maps everything to $e_H$, 
    while a \textbf{cokernel} is a coequalizer of $\phi: G \to H$ and $0: G \to H$. 
    \begin{center}
        \begin{tikzcd}[column sep = 1.5cm, row sep = 1.5cm]
            \ker(\phi) \arrow[r, "e"]
            &
            G 
            \arrow[r, shift right = 0.5ex, swap, "0"]
            \arrow[r, shift left = 0.5ex, "\phi"]
            &
            H \arrow[r, "c"]
            &
            \coker(\phi)
        \end{tikzcd}
    \end{center}
\end{definition}

In \textbf{Ab}, we set $\coker(\phi) \cong H/\im(\phi)$ while $\ker(\phi)$ is the natural 
normal subgroup of $G$. 

Note that the necessary conditions for creating kernels and cokernels is 
(1) the existence of a zero object and (2) the existence of equalizers.
If we have these ingredients, can we extend the concept of kernels and cokernels to 
additive categories? We can.

\begin{definition}
    Let $\cc$ be a category with a zero object as well as equalizers 
    and coequalizers.
    Let $f: A \to B$ be a morphism between two objects in $\cc$. We define
    \begin{itemize}
        \item \textbf{kernel} 
        to be the equalizer of $f$ and $\emptyset_A^B: A \to B$, the zero morphism,
        \item  \textbf{cokernel} of $f$
        to be the coequalizer of $f$ and $\emptyset_A^B: A \to B$. 
    \end{itemize}
    In diagrams, we have that 
    \begin{center}
        \begin{tikzcd}[column sep = 1.5cm, row sep = 1.5cm]
            \ker(f) \arrow[r, "e"]
            &
            A
            \arrow[r, shift right = 0.5ex, swap, "\emptyset"]
            \arrow[r, shift left = 0.5ex, "f"]
            &
            B \arrow[r, "c"]
            \arrow[dr, swap, "\psi"]
            &
            \coker(f)
            \arrow[d, dashed, "k"]
            \\
            C \arrow[ur, swap, "\phi"]
            \arrow[u, dashed, "h"]
            &
            &
            &
            D 
        \end{tikzcd}
    \end{center}
\end{definition}

\begin{example}
    In the category \textbf{Grp}, we certainly have a zero object $z = \{e\}$.
    Observe that for a given morphism $\phi: G \to H$, 
    we can also form the equalizer of $\phi$ by considering the pair 
    $(\ker(\phi), e: \ker(\phi) \to G)$ where $\ker(\phi) \subset G$ and $e$ being inclusion.
    For the same morphism, we can form the coequalizer be considering the pair 
    $(\overline{N}, c: H \to H/\overline{N})$ where 
    \[
        \overline{N} = \bigcap_{N \in \lambda} N
    \]
    where $\lambda = \{H' \subset H \mid \im(\phi) \subset H' \text{ and } H' \normal H\}$. 
    It's a simple exercise to show that these satisfy the necessary universal properties. 

    \textcolor{NavyBlue}{However, it's important to observe the subtle difference between the behaviors 
    of \textbf{Grp} and \textbf{Ab}}. Because every subgroup of an abelian group 
    is normal, we know that in the case of \textbf{Ab}, $\overline{N} = \im(\phi)$
    So the coequalizer becomes 
    \[
        (\im(\phi), c: H \to H/\im(\phi)).
    \]
\end{example}

It turns out that kernels and cokernels are extremely flexible in additive categories.


\begin{proposition}
    Suppose $\cc$ is an additive category. Then the following are equivalent. 
    \begin{itemize}
        \item[$(i)$] $\cc$ has equalizers and coequalizers. 
        \item[$(ii)$] $\cc$ has kernels and cokernels.
    \end{itemize}
\end{proposition}

\begin{prf}
    We only prove the statement for equalizers as the proof will be self-dual. 

    First note that $(i) \implies (ii)$ is immediate because a kernel is 
    an equalizer with a morphism $\phi$ and a zero morphism. To show $(ii) \implies (i)$, 
    suppose that we have kernels for every morphism in $\cc$. Then consider two 
    morphisms $\phi, \psi: G \to H$. We can combine these two morphisms by our 
    group operation on $\hom(G, H)$ and consider $\phi - \psi$. Since we can take kernels, we 
    take the kernel of this morphism.
    \begin{center}
        \begin{tikzcd}[column sep = 1.5cm, row sep = 1.5cm]
            \ker(\phi) \arrow[r, "e"]
            &
            G \arrow[r, "\phi - \psi"]
            &
            H 
        \end{tikzcd}
    \end{center}
    We now argue that this is the equalizer of $\phi, \psi$. First observe that 
    \[
        (\phi - \psi)\circ e = 0 \implies \phi \circ e - \psi \circ e = 0 \implies \phi \circ e = \psi \circ e
    \]
    using bilinearity of $\circ$. Hence we see that $e$ equalizes $\phi$ and $\psi$, although 
    we now need to demonstrate its universal property. 

    Now suppose that there exists an object $K$ equipped with a morphism $\sigma: K \to G$ such 
    that $\psi \circ \sigma = \psi \circ \phi$. 
    \begin{center}
        \begin{tikzcd}[column sep = 1.5cm, row sep = 1.5cm]
            \ker(\phi - \psi) 
            \arrow[r, "e"]
            &
            G
            \arrow[r, shift right = 0.5ex, swap, "0"]
            \arrow[r, shift left = 0.5ex, "\phi"]
            &
            H\\
            K \arrow[ur, swap, "\sigma"]
        \end{tikzcd}
    \end{center}
    However, note that 
    \[
        \phi \circ \sigma = \psi \circ \sigma 
        \implies 
        (\phi - \psi)\circ \sigma = 0.
    \]
    Since $e: \ker(\phi) \to G$ is kernel, we note that its 
    universal property implies that because $(\phi - \psi)\circ \sigma = 0$ 
    that there must exists a unique morphism $u: K \to \ker(\phi)$ such that 
    $e \circ u = \sigma$. Thus we have shown the diagram below 
    \begin{center}
        \begin{tikzcd}[column sep = 1.5cm, row sep = 1.5cm]
            \ker(\phi - \psi) 
            \arrow[r, "e"]
            &
            G
            \arrow[r, shift right = 0.5ex, swap, "\phi"]
            \arrow[r, shift left = 0.5ex, "\psi"]
            &
            H\\
            \arrow[u, dashed, "u"]
            K \arrow[ur, swap, "\sigma"]
        \end{tikzcd}
    \end{center}
    must hold so that $(\ker(\phi), e: \ker(\phi) \to G)$, is actually an equalizer!
\end{prf}

Note that we've once more utilized the bilinearity of $\circ$ to construct the above proof, 
which again reminds us that the assumptions we've made so far are necessary and useful. The above proof now motivates 
the following definition.

\begin{definition}
    Let $\cc$ be an additive category. Then we say $\cc$ is \textbf{preabelian}
    if it has kernels and cokernels; or, equivalently, if it has all equalizers and coequalizers. 
\end{definition}

What we have on our hands now is a very nice category where (1) finite biproducts 
exist and (2) all equalizers and coequalizers exist. If we recall from our experience 
with limits, this automatically grants us the following proposition. 

\begin{proposition}
    A preabelian category has all finite limits and finite colimits.
\end{proposition}

\begin{prf}
    If a category has finite products and equalizers, it has finite limits. If 
    it has finite coproducts and coequalizers, it has finite colimits. This is 
    Theorem \ref{products_equalizers_all_limits}.
\end{prf}

The fact that there exist finite limits and colimits is extremely convenient 
in preabelian categories. 


\begin{proposition}
    Let $\cc$ be a preabelian category.
    Let $J$ be a connected category and suppose $F: J \to \cc$ is a functor. 
    Then 
    \[
        \Lim F \cong \Colim F.   
    \]
\end{proposition}

\begin{prf}
    Recall the limit satisfies universal property 
    \begin{center}
        \begin{tikzcd}[column sep = 1.5cm, row sep = 1.5cm]
            \Delta(\Lim F) \arrow[r, "u"]
            &
            F\\
            \Delta(C)
            \arrow[u, dashed, "\Delta(h)"]
            \arrow[ur, swap, "f"]
            &
        \end{tikzcd}
        $\implies$
        \begin{tikzcd}[column sep = 1.5cm, row sep = 1.5cm]
            \Lim F \arrow[r, "u^i"]
            &
            F(i)\\
            C
            \arrow[u, dashed, "h"]
            \arrow[ur, swap, "f^i"]
            &
        \end{tikzcd}
    \end{center}
    for every object $C$ equipped with a family of morphisms $f^i: C \to F(i)$. 
    Construct the family of morphisms 
    \[
        f_i^j = 
        \begin{cases}
            \emptyset_i^j: F(i) \to F(j) & \text{if } i \ne j\\
            1_{F(i)} & \text{if } i = j
        \end{cases}
    \]
    where $\emptyset_i^j: F(i) \to F(j)$ is the unique zero morphism
    from $F(i)$ to $F(j)$.
    Then by the universal property of the limit, for each $i \in J$, there 
    exists a unique morphism $h_i: F(i) \to \Lim F$ such that the diagram below commutes. 
    \begin{center}
        \begin{tikzcd}[column sep = 1.5cm, row sep = 1.5cm]
            \Lim F \arrow[r, "u^j"]
            &
            F(j)\\
            F(i)
            \arrow[u, dashed, "h_i"]
            \arrow[ur, swap, "f_i^j"]
            &
        \end{tikzcd}
    \end{center}
    That is, we have $u^j \circ h_i = f_i^j$. 
    We now argue that we have a colimit on our hands. Specifically, suppose $D$ is an 
    object of $\cc$ equipped with a family of morphisms $g_j: F(j) \to D$. 
    Then observe that we can supply a morphism 
    \[
        \sum_{k \in J}g_ku^k: \Lim F \to D
    \]
    where the addition operation is from the group structure of $\hom(\Lim F, D)$, 
    such that the diagram below commutes. 
    \begin{center}
        \begin{tikzcd}[column sep = 1.5cm, row sep = 1.5cm]
            \Lim F 
            \arrow[d, dashed, swap, "\sum_{k \in J}g_ku^k"]
            &
            \arrow[l, swap, "h_j"]
            F(j) \arrow[dl, "g_j"]\\
            D
            &
        \end{tikzcd}
    \end{center}
    This diagram commutes since 
    \begin{align*}
        \left( \sum_{k \in J}g_ku^k \right) \circ h_j
        =
        \sum_{k \in J}g_k(u^k \circ h_j)
        =
        g_j(u^j \circ h_j)
        = g_j 
    \end{align*}
    where we utilized the bilinearity of the composition operator.
    Thus we see that $\Lim F$ is behaving just like a colimit! 
    
    The only thing we 
    must verify at this point is that this morphism is unique. Towards that goal, 
    suppose that $\ell: \Lim F \to D$ is another morphism such that 
    $\ell \circ h_j = g_j$. Recall that $u^i\circ h_i= 1_{F(i)}$, 
    so that $h_i$ is a monomorphism. Then observe that we can take 
    the image of the map 
    \[
       h_i: F(i) \to \Lim F  
    \]
    under the contravariant hom functor to get an epic group homomorphism 
    \begin{center}
        \begin{tikzcd}[column sep = 1.5cm, row sep = 1.5cm]
            \hom(\Lim F,  D) \arrow[r, "\circ h_i"]
            &
            \hom(F(i), D)
        \end{tikzcd}
    \end{center}
    between abelian groups, as $\circ$ obeys bilinearity 
    properties. By the first isomorphism theorem we then have that 
    \[
        \hom(F(i), D) \cong \hom(\Lim F, C)/\ker(\circ h_i).
    \]
    Now we want to show that this map is also injective, because then 
    we could observe that since
    \[
        \left(\ell - \sum_{k \in J}g_k \circ u^k\right)\circ h_i = 0
    \]
    that 
    \[
        \ell - \sum_{k \in J}g_k \circ u^k = 0.
    \]
    But it seems like we don't have enough to show that at the moment...

\end{prf}

\newpage
\section{Kernels and Cokernels}
At this point we've discussed preadditive, additive, and preabelian categories, where 
preabelian categories are just additive categories with the additional hypothesis 
that kernels and cokernels exist. This additional hypothesis is extremely useful, 
so we will demonstrate what this implies for us. 

Let $\cc$ be a preabelian category.
Consider an arbitrary morphism $f: A \to B$. 
One way to think about kernels and Cokernels is that they give rise to objects 
in the comma categories $(\cc \downarrow A)$ and $(B \downarrow \cc)$. 
\begin{center}
    \begin{tikzcd}[column sep = 1.5cm, row sep = 1.5cm]
        \ker(f) \arrow[r, "e"]
        &
        A
        \arrow[r, shift right = 0.5ex, swap, "0"]
        \arrow[r, shift left = 0.5ex, "f"]
        &
        B \arrow[r, "c"]
        &
        \coker(f)
    \end{tikzcd}
\end{center}
Now in the comma category $(\cc \downarrow A)$, a morphism between two objects 
$(C, f: C \to A)$ and $(D, g: D \to A)$ is a morphism $h: D \to C$ in $\cc$ such that 
$f = g \circ h$. Similarly, a morphism in the comma category $(A \downarrow \cc)$ 
between two objects $(P, m: A \to P)$ and $(Q, n: A \to Q)$ is a morphism $k: P \to Q$ such that 
$n = h \circ m$. These relations give rise to the bow-tie diagram:
\begin{center}
    \begin{tikzcd}[column sep = 1.5cm, row sep = 0.4cm]
        C \arrow[dr, "f"]
        &
        &
        P
        \arrow[dd, "k"]
        \\
        &
        A
        \arrow[ur, "m"]
        \arrow[dr, swap, "n"]
        &\\
        D
        \arrow[uu, "h"]
        \arrow[ur, swap, "g"]
        &
        &
        Q
    \end{tikzcd}
\end{center}
With that said, we can actually turn these categories into partial orders. 
In $(\cc \downarrow A)$, we say $g \le f$ if there exists an $h$ such that $f \circ h = g$, 
and in $(A \downarrow \cc)$, we say $m \le n$ if there exists a $k$ such that 
$n = k \circ m$. 


It turns out that this perspective is actually quite useful. 

\begin{proposition}
    Let $\cc$ be a category with a zero object, equalizers and coequalizers. 
    Then for each object $A$ of $\cc$, we have the functors 
    \begin{align*}
        \ker&: (A \downarrow \cc) \to (\cc \downarrow A)\\
        \coker&:(\cc \downarrow A) \to (A \downarrow \cc).
    \end{align*}
    that assign kernels and cokernels.
    Moreover, these functors establish a antitone Galois correspondence; hence we have that 
    \begin{align*}
        \ker(\coker(\ker(f))) = \ker(f) 
        \quad 
        \coker(\ker(\coker(f))) = \coker(f).
    \end{align*}
    Therefore, any $\phi$ is a kernel if and only if $\phi = \ker(\coker(\phi))$, while 
    any $\psi$ is a cokernels if and only if $\psi = \coker(\psi(\psi))$. 
\end{proposition}

\begin{prf}
    We demonstrate functoriality. First we want our functor 
    to act on objects as 
    \[
       (C, f: A \to C) \mapsto (\ker(f), e_1: \ker(f) \to A). 
    \]
    Now we explain how the functor works on morphisms. 
    Suppose we have two objects of our 
    comma category $(C, f: A \to C)$ and $(D, g: A \to D)$, and that $h: D \to C$
    is a morphism in $(A \downarrow \cc)$ 
    from $(D, g: A \to D)$ to $(C, f: A \to C)$. Then we have the diagram below.
    \begin{center}
        \begin{tikzcd}[column sep = 1.5cm, row sep = 0.4cm]
            \ker(f) \arrow[dr, "e_1"]
            &
            &
            C\\
            &
            \arrow[ur, "f"]
            A
            \arrow[dr, swap, "g"]
            &\\
            \ker(g) 
            \arrow[ur, swap, "e_2"]
            &
            &
            D \arrow[uu, swap, "h"]
        \end{tikzcd}
    \end{center}
    Now note that 
    \[
        f \circ e_2 = (h \circ g) \circ e_2 = h \circ (g \circ e_2) = 0.
    \]
    Thus, by the universal property of $e_1: \ker(f) \to A$, we know there 
    exists a \emph{unique} morphism $h': \ker(g) \to \ker(f)$ such that the diagram 
    below commutes.
    \begin{center}
        \begin{tikzcd}[column sep = 1.5cm, row sep = 0.4cm]
            \ker(f) \arrow[dr, "e_1"]
            &
            &
            C\\
            &
            \arrow[ur, "f"]
            A
            \arrow[dr, swap, "g"]
            &\\
            \ker(g) 
            \arrow[uu, "h'"]
            \arrow[ur, swap, "e_2"]
            &
            &
            D \arrow[uu, swap, "h"]
        \end{tikzcd}
    \end{center}
    
    However, this is exactly what it means to have a morphism between the objects 
    $(\ker(g), e_2: \ker(g) \to A)$ and $(\ker(f), e_1: \ker(f) \to A)$. 
    Thus we see that our functor maps on morphisms in $(A \downarrow \cc)$ in a nice 
    way:  
    \[
        h  \mapsto 
        h':
        (\ker(g), e_2: \ker(g) \to A) 
        \to 
        (\ker(f), e_1: \ker(f) \to A).
    \]
    where $h'$ is the unique map obtained from $h$ as explained above. 
    With the remaining properties easily verified, this defines a functor between the categories. 
    In addition, we can dualize our work above to also get the functor 
    $\coker: (\cc \downarrow A) \to (A \downarrow \cc)$. 

    Now this creates a Galois correspondence by regarding the comma categories as 
    partially ordered sets. Suppose that $g \le \ker(f)$. That is, there exists a $h$ such that 
    $\ker(f) \circ h = g$. Then we can compare $\coker(g)$ and $f$ by considering the diagram below. 
    \begin{center}
        \begin{tikzcd}[column sep = 1.5cm, row sep = 0.4cm]
            \ker(f) \arrow[dr, "e"]
            &
            &
            \coker(g)\\
            &
            \arrow[dr, swap, "f"]
            A
            \arrow[ur, "c"]
            &\\
            B
            \arrow[uu, "h"]
            \arrow[ur, swap, "g"]
            &
            &
            C
        \end{tikzcd}
    \end{center}
    Now observe that 
    \[
        f \circ g = f \circ (e \circ h) = 0 \circ h = 0.
    \]
    Therefore, by the universal property of the cokernel, we know there 
    exists a unique morphism $h': \coker(g) \to f$ such that the diagram 
    below commutes. This then implies that $f \le \coker(g)$. 
    \begin{center}
        \begin{tikzcd}[column sep = 1.5cm, row sep = 0.4cm]
            \ker(f) \arrow[dr, "e"]
            &
            &
            \coker(g)
            \arrow[dd, dashed, "h'"]
            \\
            &
            \arrow[dr, swap, "f"]
            A
            \arrow[ur, "c"]
            &\\
            B
            \arrow[uu, "h"]
            \arrow[ur, swap, "g"]
            &
            &
            C
        \end{tikzcd}
    \end{center}
    By a similar argument, we have that if $f \le \coker(g)$, 
    then $g \le \ker(f)$. Hence we have that 
    \[
        g \le \ker(f) \iff f \le \coker(g)
    \]
    so that, as preorder, the kernel and cokernels functors are adjoint pairs 
    that form an antitone Galois correspondence. Moreover, this implies that
    for each $f: B \to A$ and $g: A \to C$,
    \[
        f \le \coker(\ker(f)) \qquad g \le \ker(\coker(g)).
    \]
    In particular, if $f$ is the cokernel of some morphism $\phi$, and if $g$ is
    the kernel of some morphism $\psi$, then we have that 
    \[
        \coker(\phi) \le \coker(\ker(\coker(\phi)))
        \quad
        \ker(\psi) \le \ker(\coker(\ker(\psi))).
    \]
    However, applying the order reversing functors $\coker$ and $\ker$ on the relations 
    $\phi \le \ker(\coker(\phi))$ and $\psi \le \coker(\ker(\psi))$ yields 
    \[
        \coker(\ker(\coker(\phi))) \le \coker(\phi)
        \quad 
        \ker(\coker(\ker(\psi))) \le \ker(\psi).
    \]
    Hence we have that $\coker(\ker(\coker(\phi))) \cong \coker(\phi)$
    and
    $\ker(\coker(\ker(\psi))) \cong \ker(\psi)$ as desired. 
\end{prf}






\newpage
\section{Abelian Categories}

Let $\cc$ be a preabelian category, and consider an arbitrary morphism 
$\phi: A \to B$. Then, since we are in an abelian category, we can calculate 
the kernel and cokernel of this morphism, which both have their familiar 
universal properties. 
\begin{center}
    \begin{tikzcd}[column sep = 1.5cm, row sep = 1.5cm]
        \ker(\phi) \arrow[r, "e"]
        &
        A
        \arrow[r, shift right = 0.5ex, swap, "\emptyset"]
        \arrow[r, shift left = 0.5ex, "\phi"]
        &
        B \arrow[r, "c"]
        \arrow[dr, swap, "\psi"]
        &
        \coker(\phi)
        \arrow[d, dashed, "k"]
        \\
        C \arrow[ur, swap, "\phi"]
        \arrow[u, dashed, "h"]
        &
        &
        &
        D 
    \end{tikzcd}
\end{center}
One thing we can do is examine both the kernel and the cokernel \emph{of} these two morphisms.
Specifically, we can calculate the kernel $\ker(c)$ of $c$ and the cokernel 
$\coker(e)$ of $e$. However, since we have a map $\phi: A \to B$ such that 
$c \circ \phi = 0$, we see that there exists a unique map $u: A \to \ker(\coker(f))$
such that $\phi = e' \circ u$. 
Dually, since $\phi \circ e = 0$, there exists a unique map $v: \coker(\ker(f)) \to B$. 
such that $\phi = v \circ c'$. 
\begin{center}
    \begin{tikzcd}[column sep = 1.5cm, row sep = 1.5cm]
        & &[-32.5pt] 
        \ker(\coker(f)) 
        \arrow[dr, "e'"]
        &[-32.5pt] & 
        \\[-15pt]
        \ker(f) \arrow[r, "e"]
        &
        A
        \arrow[rr, "\phi"]
        \arrow[ur, dashed, "u"]
        \arrow[dr, swap, "c'"]
        &[-32.5pt]
        &[-32.5pt]
        B \arrow[r, "c"]
        \arrow[ddr, swap, "\psi"]
        &
        \coker(f)
        \arrow[dd, dashed, "k"]
        \\[-15pt]
        &
        &[-32.5pt]
        \coker(\ker(f))
        \arrow[ur, swap, dashed, "v"]
        &[-32.5pt]
        &
        \\[-40pt]
        C \arrow[uur, swap, "\phi"]
        \arrow[uu, dashed, "h"]
        &
        &[-32.5pt]
        &[-32.5pt]
        &
        D 
    \end{tikzcd}
\end{center}












