\chapter{Adjunctions.}

\section{Introduction to Adjunctions.}
As promised, we now build upon the work we did with universal morphisms to 
define the concept of an adjunction. Adjunctions are special cases of universal morphisms 
that occur between two functors $F$ and $G$ which assemble between two categories $\cc$ and $\dd$ 
as below. 
\begin{center}
    \adjunction{\cc}{F}{\dd}{G}
\end{center}
Studying adjunctions allows us to give an answer to many questions that appear 
in categories. For example, adjunctions can explain why, for instance, given 
two sets $X$, $Y$, we have the isomorphism 
\[
    F(X \times Y) \cong F(X) * F(Y)    
\]
where $F: \textbf{Set} \to \textbf{Grp}$ is the free group functor and $*$ denotes the 
\hyperref[example:free_product]{\textcolor{blue}{free product}}. They can also 
explain why this property, and other similar properties, hold for similar free functors. 

We begin with an example of an adjunction. 

\begin{example}
    Recall that for a fixed unital ring $R$ in $\textbf{Ring}$, we may 
    form the functor 
    \[
        R[-]: \textbf{Grp} \to R-\textbf{Alg} 
    \]
    which sends a group $G$ to its \hyperref[example:group_ring_functor]{\textcolor{blue}{group ring}} $R[G]$.
    Recall that 
    \[
        R[G] = \left\{ \sum_{g \in G}a_g g \;\middle|\; g \in G, \, a_g \in R, \text{ and } a_g = 0 \text{ for all but finitely many } a_g \right\}.
    \]
    Recall also that we can form the functor 
    \[
        (-)^{\times}: R\textbf{-Alg} \to \textbf{Grp}
    \]
    which sends an $R$-algebra $A$ to its group of units $A^{\times}$.
    These two functors are related in the following way. Consider a group $G$ and 
    its group ring $R[G]$. 
    In general, the units of $R[G]$ are nontrivial. One thing we do know 
    is that 
    elements of the form $1_R g$, with $g \in G$, are units of $R[G]$. (The multiplicative 
    inverse of such an element is $1_R g^{-1}$.) 
    This allows us to construct a group homomorphism
    \[
        i: G \to (R[G])^{\times} \qquad g \mapsto 1_R g.
    \]
    What is interesting about this is the following fact: $(G, i: G \to (R[G])^{\times})$
    is universal from \hyperref[definition:universal_morphism_from_D_to_F]{\textcolor{blue}{$G$ to $(-)^{\times}$}}.
    That is, if $K$ is a 
    ring, and we have a mapping $\phi: G \to K^{\times}$, then there exists a
    unique ring homomorphism $h: R[G] \to K$ such that the diagram below commutes. 
    \begin{center}
        \begin{tikzcd}[column sep = 1.4cm, row sep = 1.4cm]
            G 
            \arrow[r, "i"]
            \arrow[dr, swap, "\phi"]
            &
            (R[G])^{\times}
            \arrow[d, dashed, "(h)^{\times}"]
            \\
            &
            (K)^{\times}
        \end{tikzcd}
        \hspace{1cm}
        \begin{tikzcd}[column sep = 1.4cm, row sep = 1.4cm]
            R[G]
            \arrow[d, dashed, "h"]
            \\
            K
        \end{tikzcd}
    \end{center}
    The reason why this works is as follows: $\phi$ tells us to where to send 
    elements of $G$. Since a map on $R[G]$ can be defined by (1) defining where 
    elements of $G$ go and (2) extending linearly, $\phi$ induces the existence of $h$. 
    
    By Proposition \ref{proposition:universality_bijection}, we then have the following result: 
    If $K$ is an $R$-algebra, then for each group $G$ there is a natural
    bijection
    \[
        \hom_{\textbf{Ring}}(R[G], K) \cong \hom_{\textbf{Grp}}(G, (K)^{\times})
    \]
    Specifically, the bijection is natural in $G$.

    But wait---There's more! For every ring $K$, there is a natural 
    ring homomorphism 
    \[
        \epsilon: R[(K)^{\times}] \to K \qquad \sum_{k \in K^{\times}}a_k k \mapsto z(a_k) k
    \]
    where $z(a_k) = 1_k$, the identity of $K$, if $a_k \ne 0$, and $z(a_k) = 0$ if $a_k = 0$. 
    The reason why we care about this is because $(K, R[(K)^{\times}] \to K)$
    is universal from \hyperref[definition:universal_morphism_from_F_to_D]{\textcolor{blue}{$R[-]$ to $(K)^{\times}$}}.
    That is, if $G$ is a group and we have a mapping $\psi: R[G] \to K$, 
    then there exists a unique $j: G \to (K)^{\times}$ such that the following diagram 
    commutes. 
    \begin{center}
        \begin{tikzcd}[column sep = 1.4cm, row sep = 1.4cm]
            R[(K)^{\times}]
            \arrow[r, "\epsilon"]
            &
            K
            \\
            R[G]
            \arrow[ur, swap, "\psi"]
            \arrow[u, dashed, "{R[j]}"]
            &
        \end{tikzcd}
        \hspace{1cm}
        \begin{tikzcd}[column sep = 1.4cm, row sep = 1.4cm]
            (K)^{\times}
            \\
            G
            \arrow[u, dashed, "j"]
        \end{tikzcd}
    \end{center}
    We obtain $j$ as follows: Note that $\phi(1_R g) \in K^{\times}$, since 
    ring homomorphisms send units to units. Hence, the composite 
    \begin{center}
        \begin{tikzcd}[column sep = 1.4cm, row sep = 1.4cm]
            G 
            \arrow[r, "i"]
            &
            (R[G])^{\times}
            \arrow[r, "\phi^{\times}"]
            &
            K^{\times}
        \end{tikzcd}
    \end{center}
    where $i$ is defined earlier, yields $j$. Moreover, the diagram commutes in this way. 
    By Exercise \ref{exercise:universality_bijection}, if $K$ is a ring, then for every group $G$ 
    we have the following natural bijection 
    \[
        \hom_{\textbf{Grp}}(G, (K)^{\times})\cong \hom_{\textbf{Ring}}(R[G], K).
    \] 
    Specifically, the bijection is natural in $K$. However, we just saw this isomorphism before!
    This demonstrates our first example of an adjunction. 
\end{example}

\begin{definition}
    Let $\cc, \dd$ be categories. Consider a pair of functors 
    \begin{center}
        \begin{tikzcd}
            \cc
            \arrow[r, shift right = -0.5ex, "F"]
            &
            \dd
            \arrow[l, shift right = -0.5ex, "G"]
        \end{tikzcd}
    \end{center}
    We say that $F, G$ form an \textbf{adjunction} and that 
    $F$ is \textbf{left adjoint to $G$} (and so $G$ is \textbf{right adjoint to $F$})
    if, for all $C \in \cc$, $D \in \dd$, there is a natural bijection 
    \[
        \hom_{\dd}\bigg(F(C), D\bigg) \cong \hom_{\cc}\bigg(C, G(D)\bigg) 
    \]
\end{definition}

This definition is somewhat strange, so we comment a few remarks. 
\begin{remark}
    \begin{itemize}
        \item To define an adjunction between two functors, it suffices to specify which 
        functor is the left adjoint, or which functor is the right adjoint (since one specification determines the other). 
        Thus, the sentence ``$F$ and $G$ form an adjunction'' alone 
        does not make sense; namely, it is missing information of which 
        functor is the left or the right adjoint. 

        \item
        In an adjunction, we are always going to have some 
        kind of bijection as above. But there are two different ways we could decide to write it:
        \[
            \hom_{\dd}(F(C), D) \cong \hom_{\cc}(C, G(D)) 
            \quad
            \text{ or }
            \quad
            \hom_{\cc}(C, G(D)) \cong \hom_{\dd}(F(C), D)
        \]
        This can potentially confuse us on which functor is the left adjoint, and which one is the right. 
        However, one thing that does not change in the above expressions is
        the position of $F(C)$ and $G(D)$ in their hom-sets. In their hom-sets, 
        the symbol $F(C)$ is always in the left position, while $G(D)$ is in the right.
        Hence we can determine if $F$ or $G$ is left or right based on glancing at the 
        bijection. Conversely, knowing the left and rightedness of our functors 
        tells us how to write down the bijection. 
    \end{itemize}
\end{remark}

We now observe that this definition is equivalent to the existence of universal morphisms; 
this is something we already saw in our introductory example. 

\begin{proposition}
    Let $\cc, \dd$ be categories and consider a pair of functors         
    \begin{tikzcd}
        \cc
        \arrow[r, shift right = -0.5ex, "F"]
        &
        \dd
        \arrow[l, shift right = -0.5ex, "G"]
    \end{tikzcd}.
    The following are equivalent. 
    \begin{itemize}
        \item[(\emph{i}.)] The functors $F$, $G$ form an adjunction where
        $F$ is left adjoint to $G$ (and so $G$ is right adjoint to $G$). 
        \item[(\emph{ii}.)] There exist natural transformations 
        \[
            \eta: I_{\cc} \to G \circ F \qquad \epsilon: F \circ G \to I_{\dd}
        \]
        such that
        \begin{itemize}
            \item For each $C \in \cc$, the morphism 
            $\eta_C: C \to G(F(C))$ is universal from 
            \hyperref[definition:universal_morphism_from_D_to_F]{\textcolor{blue}{$C$ to $G$}}

            \item For each $D \in \dd$, the morphism 
            $\epsilon_D: F(G(D)) \to D$ is universal from 
            \hyperref[definition:universal_morphism_from_F_to_D]{\textcolor{blue}{$F$ to $D$}}
        \end{itemize}
    \end{itemize}
\end{proposition}

\begin{prf}
    Since $F$ is left adjoint to $G$, we have the natural bijection 
    \[
        \hom_{\dd}(F(C), D) \cong \hom_{\cc}(C, G(D)).
    \]
    This is natural in $C$ and $D$.

    By Proposition \ref{proposition:universality_bijection}, 
    the above bijection is natural in $D$ if and only 
    if there exists a morphism $\eta_C: C \to G(F(C))$ which is 
    universal from $C$ to $G$. 
    However, the bijection holds for all $C$. Therefore, we obtain a family 
    of universal morphisms 
    \[
        \eta_C: C \to G(F(C)).
    \]
    Since this bijection is also natural in $C$, 
    we ultimately obtain a natural transformation $\eta: I_{\cc} \to G \circ F$. 

    Using the same bijection from our adjunction, we can use
    Exercise \ref{exercise:universality_bijection} to conclude the existence of 
    a family of morphisms $\epsilon_D: F(G(D)) \to D$ which is universal from $F$ to $D$. 
    We then use the fact that the bijection is natural to form the natural 
    transformation $\epsilon: F \circ G \to I_{\dd}$, as desired.

    As we used if and only if propositions, our work proves both directions, which 
    completes the proof.
\end{prf}

\begin{definition}
    Let \adjunction{\cc}{F}{\dd}{G} be an adjunction. We establish the 
    following terminology.
    \begin{itemize}
        \item The natural transformation $\eta: I_{\cc} \to G \circ F$ is 
        the \textbf{unit of the adjunction}.
        \item The natural transformation $\epsilon: F \circ G \to I_{\dd}$ 
        is the \textbf{counit of the adjunction}.
    \end{itemize}
\end{definition}

\begin{example}
    We already saw this proposition in action in the introductory example. 
    In that example, we found a pair functors 
    \begin{center}
        \adjunction{\textbf{Grp}}{{R[-]}}{\textbf{Ring}}{(-)^{\times}}
    \end{center}
    that formed an adjunction with universal morphisms 
    \[
        i_G: G \to (R[G])^{\times}
        \qquad 
        \epsilon_K: R[(K)^{\times}] \to K
    \]
    for all groups $G$ and rings $K$. Hence $i_G$ is the unit of the adjunction, while 
    $\epsilon_K$ is the counit. 
    These units and counits are what allowed us to establish the bijection
    \[
        \hom_{\textbf{Ring}}(R[G], K) \cong \hom_{\textbf{Grp}}(G, (K)^{\times})
    \]
    natural in $G$ and $K$. Hence, the group ring functor $R[-]$ is left adjoint 
    to the group of units functor $(-)^{\times}$. 
\end{example}

Using our previous work, we very quickly and (hopefully) painlessly established a connection between the natural 
bijection that appears in the definition of an adjunction and the unit and counit morphisms. 
However, we did not really describe what the bijection actually does on elements.
The next proposition characterizes the bijection.



\begin{proposition}\label{proposition:adjunction_isomorphism_behavior}
    Let $\cc$, $\dd$ be categories, and suppose 
    \begin{tikzcd}
        \cc
        \arrow[r, shift right = -0.5ex, "F"]
        &
        \dd
        \arrow[l, shift right = -0.5ex, "G"]
    \end{tikzcd}
    form an adjunction with $F$ left adjoint to $G$. 
    Let $\eta$, $\epsilon$ be the unit and counit.
    
    For each $C, D$, the natural bijection
    \[
        \phi_{C,D}: \hom_\dd(F(C), D) \isomarrow \hom_\cc(C, G(D))
    \]
    is given by the function where for each $f: F(C) \to D$ and $g: C \to G(D)$, 
    \[
        \phi(f) = G(f) \circ \eta_C \qquad \phi^{-1}(g) = \epsilon_D \circ F(g).
    \]
\end{proposition}

The proof is left to the reader. 

\begin{example}
    We have already encountered the pair of functors 
    \begin{center}
        \begin{tikzcd}
            \textbf{Set}
            \arrow[r, shift right = -0.5ex, "F"]
            &
            \textbf{Mon}
            \arrow[l, shift right = -0.5ex, "U"]
        \end{tikzcd}
    \end{center}
    where $F$ is the free monoid functor and $U$ is the forgetful monoid functor.
    We previously saw that given a set $X$, there
    exists an inclusion morphism
    \[
        i_X: X \to U(F(X)) 
    \]
    and this morphism is universal from \universalDToF{$X$ to $U$}. 
    In addition, we know that the monoid homomorphism 
    \[
        \epsilon_M: F(U(M)) \to M 
    \]
    and this morphism is from \universalFToD{$F$ to $M$}. Therefore, we see that 
    $F$ and $U$ are adjoint functors; specifically, $F$ is left adjoint to $G$ and $G$ 
    is right adjoint to $F$, and we have the natural bijection 
    \[
        \hom_{\textbf{Mon}}(F(X), M) \cong \hom_{\textbf{Set}}(X, U(M)).
    \]
    Moreover, we know exactly how this bijection works. 
    \begin{itemize}
        \item For $f: F(X) \to M$, we send $\phi(f)$ to $U(f) \circ i_X$. 
        \item For $g: X \to U(M)$, we send $\phi^{-1}(g)$ to $\epsilon_M \circ F(g)$. 
    \end{itemize}
    This data assembles into the commutative diagrams as below. 
    \begin{center}
        \begin{tikzcd}[column sep = 1.4cm, row sep = 1.4cm]
            X 
            \arrow[r, "i_X"]
            \arrow[dr, swap, "\phi(f)"]
            &
            U(F(X))
            \arrow[d, "U(f)"]
            \\
            &
            U(M)
        \end{tikzcd}
        \hspace{1cm}
        \begin{tikzcd}[column sep = 1.4cm, row sep = 1.4cm]
            F(U(M))
            \arrow[r, "\epsilon_M"]
            &
            M
            \\
            F(X)
            \arrow[u, "F(g)"]
            \arrow[ur, swap, "\phi^{-1}(g)"]
            &
        \end{tikzcd}
    \end{center}

\end{example}

    Now we offer some sufficient conditions for establishing an
    adjunction. 
    \begin{proposition}
        Let $G: \dd \to C$ be a functor. Suppose that for each $C \in \cc$,
        there exists an object $F_0(C) \in \dd$ and a universal morphism
        $\eta_C: C \to G(F_0(C))$ from $C$ to $G$. 
        Then there exists a functor $F: \cc \to \dd$ which is left-adjoint 
        to $G$.
    \end{proposition}

    \begin{prf}
        To have universality from $C$ to $G$, the diagram 
        \begin{center}
            \begin{tikzcd}[column sep = 1.4cm, row sep = 1.4cm]
                {C} \arrow[r, "\eta_C"] 
                \arrow[dr, swap, "g"]
                & G(F(C)) \arrow[d, dashed, "G(f)"] \\
                & {G(D)}
            \end{tikzcd}
            \hspace{1cm}
            \begin{tikzcd}[column sep = 1.4cm, row sep = 1.4cm]
                {F(C)} \arrow[d, dashed, swap, "f"]\\
                {D}
            \end{tikzcd}
        \end{center}
        must commute. Hence we have a bijection 
        \[
            \hom_{\dd}({F(C), D}) \cong \hom_\cc({C, G(D)}).  
        \]
        Now suppose $h: C \to C'$. Then the dashed arrow 
        \begin{center}
            \begin{tikzcd}[column sep = 1.4cm, row sep = 1.4cm]
                C \arrow[r, "\eta_C"] \arrow[d, swap, "f"] 
                &
                G(F_0(C)) \arrow[d, dashed, "G(h)"]\\
                C' \arrow[r, swap, "\eta_{C'}"]
                &
                G(F_0(C')) 
            \end{tikzcd}
        \end{center}
        must exist by universality; we simply utilize the previous
        diagram. In other words, if $h: C \to C'$, then there exists a
        morphism $f: F_0(C) \to F_0(C')$. With that said, we can then
        define a functor where
        $F: \cc \to \dd$ with $F(C) = F_0(C)$ and $F(h) = F_0(C) \to
        F_0(C')$. By construction, this functor is left adjoint to $G$. 
    \end{prf}
    
    A similar proposition holds for the establishing a right adjoint.
    \begin{proposition}
        Let $F: \cc \to \dd$ be a functor. Suppose for each object $D
        \in \dd$ there exists an object $G_0(D) \in \cc$ 
        and a universal morphism $\epsilon_d: F(G_0(D)) \to
        D$ from $F$ to $D$.
        
        Then there exists a functor $G: \dd \to \cc$ which is right-adjoint 
        to $F$. 
    \end{proposition}

    We now introduce a proposition which offers sufficient conditions
    for an adjunction, although it is not parallel to either of our
    previous propositions. 

    \begin{proposition}
        Let $F: \cc \to \dd$ and $G: \dd \to \cc$ be functors, and
        suppose we have the pair of natural transformations:
        \begin{align*}
            \eta_{C} : I_C \to G \circ F \quad
            \epsilon_{D}: I_D \to F \circ G
        \end{align*}
        such that the following composites are the identity: 
        \begin{center}
            \begin{tikzcd}
            G \arrow[r, "\eta_{G}"] & G \circ F \circ G
            \arrow[r,"G(\epsilon)"] & G
            \end{tikzcd}
            \hspace{0.5cm}
            \begin{tikzcd}
            F \arrow[r, "F(\eta)"] & F \circ G \circ F
            \arrow[r, "\epsilon_{F}"]
            & 
            F
            \end{tikzcd}
        \end{center}
        Then there exists a bijective $\phi$ such that $(F, G, \phi)$
        form an adjunction between $\cc$ and $\dd$. 
    \end{proposition}


    \begin{example}
    Let $U: \textbf{R-Mod} \to \textbf{Ab}$ be the forgetful functor,
    which forgets the $R$-module structure on the underlying abelian
    group $M$. Consider the functor $F: \textbf{Ab} \to
    \textbf{R-Mod}$, where $F(A) = R \otimes A$. We'll show that this
    is left-adjoint to $U$ as follows. 

    To show this, we'll propose a morphism which we will show to be
    universal. If $A$ is an abelian group, then we let 
    $\eta_A : A \to U(F(A))$ where $\eta_A(a) = 1 \otimes a$. 

    Thus let $M$ be an $R$-module, and suppose there exists a morphism
    $f: A \to U(M)$. Then we can define a morphism $\phi: F(A) \to M$
    where 
    \[
        \phi(r \otimes a) = r\cdot f(a).
    \]
    Our construction ensures that this is a well-defined $R$-module
    homomorphism. Hence we clearly have the equality $U(\phi) \circ \eta_A =
    f$. Visually, this becomes 
    \begin{center}
        \begin{tikzcd}[column sep = 1.4cm, row sep = 1.4cm]
            A \arrow[r, "\eta_A"] \arrow[dr, swap, "f"]
            &
            U(F(A)) \arrow[d, dashed, "U(\phi)"]\\
            & U(M)
        \end{tikzcd}
        \hspace{1cm}
        \begin{tikzcd}[column sep = 1.4cm, row sep = 1.4cm]
            F(A) \arrow[d, dashed, "\phi"]\\
            M
        \end{tikzcd}
    \end{center} 
    Since the construction of $\phi$ depends directly on the existence
    of $f$, we see that it is unique. Hence we see that $\eta_A: A \to
    U(F(A))$ is universal from $A$ to $U$. Then by Theorem 4.1, we see
    that we have an adjunction, so that $F$ is truly left adjoint to
    the forgetful functor $U$. 
    
    \end{example}

    The following proposition is one of the main reasons why adjoint functors are 
    extremely useful.

    \begin{proposition}
        Let $F, F' : \cc \to \dd$ be two left adjoints of the functor
        $G : \dd \to \cc$. Then $F, F'$ are naturally isomorphic. 
    \end{proposition}

    \begin{prf}
        Let $(F, G, \phi)$ and $(F', G, \phi')$ be two adjunctions
        between  $\cc$ and $\dd$. Then these adjoints give rise to the
        universal morphisms
        \[
            \eta_C : C \to G(F(C)) \quad \eta'_{C}: C \to G(F'(C))
        \]
        for every $C \in \cc$. Since these are both universal
        morphisms from $C$ to $G$, we know that they are isomorphic.
        Hence there exists a unique isomorphism $\theta_C :  F(C) \to
        F'(C)$ by universality such that $G(\theta_C) \circ \eta_C =
        \eta_C'$ (think of a universal diagram). 

        Now let $h: C \to C'$ be a morphism in $\cc$. Then 
        $F'(h) \circ \theta_C = \theta_{C'} \circ F(h)$ so that the diagram 
        \begin{center}
            \begin{tikzcd}[column sep = 1.4cm, row sep = 1.4cm]
                F(C) \arrow[r, "\theta_C"] \arrow[d, swap, "F(h)"]
                & F'(C) \arrow[d, "F'(h)"]\\
                F(C') \arrow[r, swap, "\theta_{C'}"]
                &
                F'(C')
            \end{tikzcd}
        \end{center}
        commutes. Hence we see that $\theta: F \to F'$ is a natural
        isomorphic transformation between $F$ and $F'$, so that these
        two functors are naturally isomorphic.
    \end{prf}
    The other direction holds as well. That is, two right adjoints to
    one left adjoint are naturally isomorphic as well, and the proof
    is the same. We now have our last proposition for this section. 

    \begin{proposition}
        Let $G: \dd \to \cc$ be a functor. Then $G$ has a left-adjoint
        $F: \cc \to \dd$ if and only if for each $C \in \cc$, the
        functor $\hom_{\cc}(C, G(-))$ is representable as a functor of
        $D \in \dd$. Furthermore, if $\phi: \hom_{\dd}(F_0(C), D)
        \cong \hom_{\cc}(C, G(D))$ is a representation of this
        functor, then $F_0$ is the object function of $F$. 
    \end{proposition}

    Finally, we end this section by realizing that we can actually
    form composition of adjoints. 
    \begin{proposition}
        Let $\cc, \dd$ and $\ee$ be categories. Suppose we have
        two adjunctions as below.
        \begin{center}
            \begin{tikzcd}
                \cc
                \arrow[r, shift right = -0.5ex, "F"]
                &
                \dd
                \arrow[l, shift right = -0.5ex, "G"]
                \arrow[r, shift right = -0.5ex, "F'"]
                &
                \ee
                \arrow[l, shift right = -0.5ex, "G'"]
            \end{tikzcd}
        \end{center}
        Then the functors $F'\circ F$, $G \circ G'$
        form an adjunction between $\cc$ and $\ee$.
        Further, if $(\eta, \epsilon)$ and $(\eta', \epsilon')$ 
        are unit and counits of the adjunction 
        from $(F, G)$ and $(F', G')$, then the unit and counit of the new adjunction 
        is 
        \begin{align*}
            \overline{\eta}_C = G(\eta'_{F(C)}) \circ \eta_C: C \to
            (G\circ G) \circ (F'\circ F(C))\\
            \overline{\epsilon}_E = \epsilon'_E \circ F'(\epsilon_{G'(E)}) 
            : (F' \circ F) \circ (G \circ G'(E)) \to E
        \end{align*}
        \vspace{-0.7cm}
    \end{proposition}

    \begin{prf}
        First, observe that the two given adjunctions give rise to 
        \[
            \hom_{\dd}(F(C), D) \cong \hom_{\cc}(C, G(D)) 
            \qquad 
            \hom_{\ee}(F'(D), E) \cong \hom_{\dd}(D, G'(E)).
        \]
        which are relations that are natural in objects $C, D$ and
        $E$. Observe that in the second relation, we can set $D =
        F(C)$. This then translates to 
        \[
            \hom_{\ee}(F'(F(C)), E) \cong \hom_{\dd}(F(C), G'(E)).
        \]
        Using the first relation, we know that 
        $\hom_{\dd}(F(C), G(E)) \cong \hom_{\cc}(C, G(G'(E)))$.
        Putting this together, we then have the bijection of homsets 
        \[
            \hom_{\ee}(F'\circ F(C)), E) \cong \hom_{\cc}(C, G \circ G'(E))
        \]
        which is natural in $C$ and $E$. Now, describing the
        unit and counit is a bit ugly, and not exactly necessary, since in
        the end we know what these adjunctions look like. The
        punchline here is that we can write our new unit and counit in
        terms of the original ones. 

        Observe that for any object $C$ of $\cc$, we have the universal
        morphism 
        \[
             \eta_C: C \to G(F(C)).
        \]
        Since $F(C) \in \dd$, we can use $\eta'$ that
        \[
            \eta'_{F(C)}: F(C) \to G'(F'(F(C))).            
        \]
        Finally, note that $G(\eta'_{F(C)}): G(F(C)) \to G(G'(F'(F(C))))$.
        However, we can precompose this with $\eta_C$ to have that 
        \[
            G(\eta'_{F(C)}) \circ \eta_C: C \to G(G'(F'(F(C)))).
        \]
        On the other hand, for any object $E$ of $\ee$ that 
        \[
            \epsilon'_E : F'(G'(E)) \to E.
        \]
        We also have $\epsilon_D : F(G(D)) \to D$ for any object $D
        \in \dd$. Hence, 
        we can set $D = G'(E)$ for some object $E$ of $\ee$
        to get
        \[
            \epsilon_{G'(E)} : F(G(G'(E))) \to G'(E).
        \]
        We can then get that $F'(\epsilon_{G'(E)}) : F'(F(G(G'(E))))
        \to F'(G'(E))$. Composing this with the original
        $\epsilon'_D$, we get that 
        \[
            \epsilon'_E \circ F'(\epsilon_{G'(E)}): F'(F(G(G'(E)))) \to E
        \] 
        as desired. Now showing that these remain universal is not
        hard.
    \end{prf}

    {\large \textbf{Exercises}
    \vspace{0.5cm}}

    \begin{itemize}
        \item[\textbf{1.}]
        Give a proof of Proposition \ref{proposition:adjunction_isomorphism_behavior}.  

        \item[\textbf{2.}] 
        Let $U: \textbf{Ab} \to \textbf{Grp}$ be the
        forgetful functor, and suppose $F: \textbf{Grp} \to \textbf{Ab}$ is 
        the abelianization functor. That is, if $G$ is a group and $\phi:
        G \to G'$ is a group homomorphism then 
        \[
            F(G) = G/[G,G] \qquad F(\phi) : G/[G, G] \to G'/[G',G'].
        \]
        where $[G,G]$ is the commutator subgroup.

        Show that we have an adjunction \adjunction{\textbf{Grp}}{F}{\textbf{Ab}.}{U}
        Give a description of the unit and counits. 
    \end{itemize}
    


    
    \newpage
    \section{Reflective Subcategories.} 

    \begin{definition}
        Let $\aa$ be a full subcategory of $\cc$. We say $\aa$ is
        \textbf{reflective} in $\cc$ whenever the inclusion functor
        $I: \aa \to \cc$ has a left adjoint $F: \cc \to \aa$. We then
        say the functor $F$ is the \textbf{reflector}, and the
        adjunction $(F, I, \phi)$ is a \textbf{reflection} of $B$. 
    \end{definition}
    In the case of a reflection, we obtain the bijection of hom-sets 
    \[
        \hom_{\aa}(F(C),A) \cong \hom_{\cc}(C, I(A)) \implies \hom_{\aa}(F(C),A) \cong \hom_{\cc}(C, A) 
    \]
    which is natural in both $C$ and $A$. 


  
    \begin{example}
        Let $F: \textbf{Grp} \to \textbf{Ab}$ be the abelianization functor,
        which sends a group $G$ to its free abelian group $G/[G,G]$.
        From Exercise \ref{exercise:abelianization_functor_is_left_adjoint},
        we know that this is left adjoint to the forgetful functor $U: \textbf{Ab} \to \textbf{Grp}$. 

        However, the functor $U: \textbf{Ab} \to \textbf{Grp}$ is isomorphic to the inclusion 
        functor $I: \textbf{Ab} \to \textbf{Grp}$. Hence, $F$ is also left adjoint 
        to the inclusion functor, so that \textbf{Ab} is a reflective subcategory 
        of \textbf{Grp}. 
    \end{example}

    \begin{example}
        Let $\textbf{Top}$ be the category of topological spaces with
        morphisms continuous functions. Let $\textbf{CHaus}$, the
        category of compact Hausdorff spaces, which is a subcategory
        of $\textbf{Top}$. 

        If we let $X$ be a topological space, then we denote
        $\beta(X)$ to be the Stone-Cech compactification. Let $I :
        \textbf{CHaus} \to \textbf{Top}$ be the inclusion functor. 
        Then the definition of the Stone-Cech compactification of a space $X$ is the
        universal property: 
        \begin{center}
            \begin{tikzcd}[column sep = 1.4cm, row sep = 1.4cm]
                X \arrow[dr, swap, "f'"] \arrow[r, "u"] & I(\beta(X)) \arrow[d, dashed, "\beta(f)"]\\
                & I(C)
            \end{tikzcd}
            \hspace{1cm}
            \begin{tikzcd}[column sep = 1.4cm, row sep = 1.4cm]
                \beta(X) \arrow[d, dashed, "f"]\\
                C
            \end{tikzcd}
        \end{center}
        That is, the Stone-Cech compactification is a topological
        space $\beta(X)$ with a morphism $u: X \to \beta(X)$ which is
        universal across all morphisms $f: X \to C$ where $C$ is
        compact, Hausdorff. 

        Thus we see that a Stone-Cech compactification gives
        rise to an object $\beta(X) \in \textbf{CHaus}$ and a
        universal morphism $X \to I(\beta(X))$ from $X$ to $I$. Now by
        Proposition 4.1, this makes $\beta : \textbf{Top} \to
        \textbf{CHaus}$ a functor, which is left adjoint to the
        inclusion functor $I: \textbf{CHaus} \to \textbf{Top}$. 

        This then makes $\beta: \textbf{Top} \to \textbf{CHaus}$ a
        reflector, so that the adjunction is a reflection between
        $\textbf{Top}$ and $\textbf{CHaus}$. Consequently we have the
        bijection 
        \[
            \hom_{\textbf{Top}}(X, I(C)) \cong \hom_{\textbf{CHaus}}(\beta(X), C)
            \implies 
            \hom_{\textbf{Top}}(X, C) \cong \hom_{\textbf{CHaus}}(\beta(X), C).
        \]
        since $I(C)$ is technically no different than from $C$. This
        bijection is natural in both $X$ and $C$. 
    \end{example}

    \begin{example}
        Let $\textbf{Ab}_{\textbf{TF}}$ represent the category of
        abelian groups with torsion free elements (for a lack of
        better notation). Then we have a natural inclusion functor 
        $I: \textbf{Ab}_{\textbf{TF}} \to \textbf{Ab}$.
        Now consider the functor $F : \textbf{Ab} \to
        \textbf{Ab}_{\textbf{TF}}$, which we define as follows:
        \begin{description}
            \item[Objects.] Let $G$ be an abelian group. Then 
            $F(G) = G_{TF}$ where 
            \[
                G_{TF} = \{g \in G \mid g^n \ne e \text{ for } n = 1, 2, 3, \dots\}.
            \] 
            That is, it sends $G$ to its underlying abelian group of
            torsion-free elements. It's not hard to show this is an
            abelian group.
            
            \item[Morphisms.] Suppose $\phi: G \to H$ is a morphism
            between abelian groups. Then we set $F(\phi) = \phi_{TF}$
            where 
            \[
                \phi_{TF}: G_{TF} \to H_{TF} \qquad \phi_{TF}(g) = \phi(g).
            \]
            Note that this definition will cause no issues, since
            $\text{ord}(g) = \text{ord}(\phi(g))$. Thus we simply
            obtain $\phi_{TF}$ by restricting $\phi$ to $G_{TF}$.
        \end{description} 
        To show that $F$ is left adjoint to $I$, we need to
        demonstrate that there exists a universal morphism $\eta_{G} :
        G \to I(F(G))$ for every $G \in \textbf{Ab}$. Hence we propose
        $\eta_{G}$ takes on the form 
        \[
            \eta_{G}(g) = 
            \begin{cases}
                g & \text{ if } \text{ord}(g) = \infty\\
                e & \text{ otherwise. }
            \end{cases}
        \]
        To show this is universal from $G$ to $I$, suppose we have a
        morphism $\phi: G \to I(H)$, where $H \in
        \textbf{Ab}_{\textbf{TF}}$. Then there exists a morphism
        $\psi: F(G) \to H$ such that $I(\phi) \circ \eta_{G} = \phi$.
        Visually, that is, 
        \begin{center}
            \begin{tikzcd}[column sep = 1.4cm, row sep = 1.4cm]
                G \arrow[dr, swap, "\phi"]\arrow[r, "\eta_{G}"] & I(F(G)) \arrow[d, dashed,
                "I(\psi)"]\\
                & I(H)
            \end{tikzcd}
            \hspace{1cm}
            \begin{tikzcd}[column sep = 1.4cm, row sep = 1.4cm]
                F(G) \arrow[d, dashed, "\psi"]\\
                H
            \end{tikzcd}
        \end{center}
        Sure such a morphism exists, but why the equality? 
        \begin{description}
            \item[$\bm{g \in \ker(\eta_G)}$.] If $g \in \ker(\eta_G)$,
            then $g$ has finite order. Hence we see that $\phi(g) =
            e$; this is because $\text{ord}(\phi(g)) =
            \text{ord}(g) < \infty$, but the only element in $I(H)$
            with finite order is $e$. We then have that $g \in
            \ker(\phi)$. Therefore, 
            \[
                I(\psi)\circ \eta_G(g) = I(\psi)(e) = e = \phi(g).
            \]
            Hence $I(\psi) \circ \eta_G = \phi$ if $g \in
            \ker(\eta_{G})$. 

            \item[$\bm{g \not\in \ker(\eta_G)}$.]
            if $g \not\in \ker(\eta_G)$, then we know that
            $\text{ord}(g) = \infty$. Therefore, we see that 
            \[
                I(\psi) \circ \eta_G(g) = I(\phi)(g) = \phi(g).
            \] 
            Hence $I(\psi) \circ \eta_G = \phi$ for $g \not\in
            \ker(\eta_G)$. 
        \end{description}
        By our previous work, we then have that $I(\psi) \circ \eta_G
        = \phi$, as desired. Now $\psi$ is of course unique based on
        its construction, since its definition depends directly on
        $\phi$. We then have that $\eta_G: G \to I(F(G))$ is universal
        from $G$ to $I$ for each $G \in \textbf{Ab}$! 

        We then have by Theorem 4.1 that $F, I$ form an adjunction, so
        that $F$ is the left adjoint of $I$. Hence by definition, we
        see that $\textbf{AB}_{\textbf{TF}}$ forms a full reflective
        subcategory of $\textbf{Ab}$.
    \end{example}

    {\large \textbf{Exercises}
    \vspace{0.5cm}}

    \begin{itemize}
        \item[\textbf{1.}] Is \textbf{FinSet} a reflective subcategory of \textbf{Set}?  
        \item[\textbf{2.}]
        Let $G$ and $H$ be a groups. Prove that
        \[
            G*H/[G*H, G*H] \cong G/[G,G]\oplus H/[H,H]
        \] 
        where $G*H$ denotes the \hyperref[example:free_product]{\textcolor{blue}{free product}} 
        of $G$ and $H$.
        (What this is saying is that $F: \textbf{Grp} \to \textbf{Ab}$, the abelianization 
        functor, preserves coproducts. Eventually, this fact will immediately 
        follow by our knowledge of the adjunction \adjunction{\textbf{Grp}}{F}{\textbf{Ab}.}{U}) 
    \end{itemize}
    
    \newpage
    \section{Equivalence of Categories}
    In an ideal world, if we have a category of which we are
    interested in, our goal would be to find an isomorphism between it
    and a category of which we understand very well. We then know that
    certain mathematical structures are invariant between
    transitioning between the two, so that we could better understand
    our desired category. 

    However, this is generally too much to ask for. Many categories
    which are constructed are constructed in such a way that they're
    not isomorphic to anything we're familiar with; if they were, then
    they probably wouldn't be interesting.
    Hence we have a more useful notion of equivalence between
    categories. 

    \begin{definition}
        Let $F : \cc \to \dd$ be a functor. We say that $\cc$ is
        \textbf{equivalent} to $\dd$ if there exists a functor $G:\dd \to \cc$
        and natural isomorphisms $\eta: I_\cc \to G \circ F $ and
        $\epsilon: F \circ G \to I_\dd$.

        In this case, we say both $F$ and $G$ are an \textbf{equivalence of categories}.
    \end{definition}

    \begin{example}
        Let $X$ and $Y$ be sets, and regard them as discrete categories. Then a
        functor $F: X \to Y$ is just a function between sets. In this case, to say 
        that $X$ and $Y$ are equivalent is if there exists a functor (function!) 
        $G: Y \to X$ such that we have natural isomorphisms 
        $\eta_x: x \to G(F(x))$ and $\epsilon_x: F(G(x)) \to x$. However, 
        each category has nontrivial morphisms; hence we see that each of 
        these must be identity morphisms so that 
        \[
            G(F(x)) = x \qquad F(G(x)) = x.
        \]
        What this then means is that an equivalence of categories 
        for sets is just a pair of invertible functions. That is, 
        it gives rise to an isomorphism. 
    \end{example}

    Since $\eta, \epsilon$ are already natural transformations, this
    simply makes them natural isomorphisms. It turns out that the
    notion of equivalence is more useful than of an isomorphism. An
    isomorphism is just too much to ask, but equivalence does give us
    nice invariants too. 

    \begin{definition}
        A \textbf{adjoint equivalence} between categories $C$ and $D$
        is an adjunction $(F, G, \eta, \epsilon)$ where the unit and
        counit $\eta$ and $\epsilon$ are natural isomorphisms.
    \end{definition}

    It turns our an adjoint equivalence is the same thing as an
    equivalence between categories. But before we move on, we prove a
    lemma and a proposition.
    \begin{lemma}
        Let $\cc$ be a category, and $f: A \to B$ a morphism. Then 
        $f$ induces a natural transformation 
        \[
            f^*: \hom_{\cc}(C, -) \to \hom_{\cc}(C', -)
        \]
        Then $f^{*}$ is a monomorphism if and only if $f$ is an epimorphism, 
        and $f^{*}$ is an epimorphism if and only if $f$ is a split monomorphism 
        (that is, if and only if $f$ has a left-inverse.)
    \end{lemma}

    \begin{prf}
        \begin{description}
            \item[$\bm{\implies}$] Observe that $\hom_{\cc}(C, -) \to
            \cc \to \textbf{Set}$ is a functor. Then $f^*:
            \hom_{\cc}(C, -) \to \hom_{\cc}(C', -)$ is a natural transformation where
            $f: C' \to C$. Now suppose $\eta, \eta': F
            \to \hom_{\cc}(C, -)$, where $F: \cc \to \textbf{Set}$ is a functor, are
            natural transformations. Then if $f^*$ is monic, 
            \[  
                f^* \circ \eta = f^* \circ \eta' \implies \eta = \eta'.
            \]
            Now let $h: A \to A'$ be a morphism in $\cc$. Then we have the
            commutative diagram 
            \begin{center}
                \begin{tikzcd}[column sep = 1.4cm, row sep = 1.4cm]
                    A \arrow[d, "h"]\\
                    A'
                \end{tikzcd}
                \hspace{1cm}
                \begin{tikzcd}[column sep = 1.4cm, row sep = 1.4cm]
                    F(A) \arrow[r, "{\eta_A, \eta'_A}"] \arrow[d, swap, "F(h)"] 
                    &
                    \hom_{\cc}(C, A) \arrow[d, "h_*"] \arrow[r, "f^*"]
                    &
                    \hom_{\cc}(C', A) \arrow[d, "h_*"]
                    \\
                    F(A') \arrow[r,swap, "{\eta_{A'}, \eta'_{A'}}"] 
                    &
                    \hom_{\cc}(C, A') \arrow[r,swap, "f^*"]
                    &
                    \hom_{\cc}(C', A')
                \end{tikzcd}
            \end{center}
            where we denote $\eta_A, \eta'_{A}$ on the arrow to
            signify the fact that both $\eta_A, \eta_A'$ are morphisms
            from $F(A)$ to $\hom_{\cc}(C, A)$. Now 
            let $x \in F(A)$. Then
            \[
                f^* \circ \eta_A(x) = f^* \circ \eta'_A(x) \iff \eta_A(x) \circ f = \eta'_A(x) \circ f.
            \]
            But if $f$ is monic, then $f^* \circ \eta_A(x) = f^*
            \circ'_A(x)$ implies that $\eta_A = \eta'_A$. Hence
            we see that $\eta_A(x) \circ f = \eta'_A(x) \circ f
            \implies \eta_A(x) = \eta'_A(x).$

            \item[$\bm{\impliedby}$] Now suppose $f$ is epic. Then
            using the same notation as earlier, note that 
            \[
                f^* \circ \eta_A(x) = f^* \circ \eta'_A(x) \iff \eta_A(x) \circ f = \eta'_A(x) \circ f \implies \eta_{A} = \eta_{A}.
            \]
            Hence we see that $f^*$ is a monomorphism. 
        \end{description}
    \end{prf}
    Taking the dual of what we proved, we prove the second part of the
    lemma. Now we'll use this lemma in the theorem below, one which will be very useful.

    \begin{proposition}
        Let $(F, G, \eta, \epsilon)$ be an adjunction between
        categories $\cc$ and $\dd$. Then 
        \begin{description}
            \item[$\bm{(i)}$] $G$ is faithful if and only if for each
            $D \in \dd$, $\epsilon_{D}$ is epic 
            \item[$\bm{(ii)}$] $G$ is full if and only if every
            $\epsilon_{D}$ is split monic. 
        \end{description}
        Therefore, $G$ is full and faithful if and only if
        $\epsilon_D$ is an isomorphism between $F(G(D))$ and $D$. 
    \end{proposition}

    \begin{prf}
        If $G: \dd \to \cc$ is a functor, then we see that $G$ itself
                becomes a natural transformation between the two
                functors: 
                \[
                    G_{D,-}: \hom_{\dd}(D, -) \to \hom_{\dd}(G(D), G(-)).    
                \]
                Recall that we have an adjunction given by $F, G$.
                Then there exists a bijection $\phi$ where 
                \[
                    \phi_{C, D'}: \hom_{\cc}(F(C), D') \to \hom_{\dd}(C, G(D)).
                \]
                Thus $\phi^{-1}: \hom_{\dd}(C, G(D)) \to
                \hom_{\dd}(F(C), D')$. Moreover, if $D$ is an
                arbitrary object, this becomes a natural
                transformation between the two functors: 
                \[
                    \phi^{-1}_{C, -}: \hom_{\dd}(C, G(-)) \to \hom_{\cc}(F(C), -).
                \]
                Let $C = G(D)$. Then we have the following sequence of
                natural transformations:
                \begin{center}
                    \begin{tikzcd}[column sep = 1.4cm, row sep = 1.4cm]
                        \hom_{\dd}(D,-) \arrow[r, "G_{D, -}"]
                        &
                        \hom_{\cc}(G(D), G(-)) \arrow[r, "\phi^{-1}_{G(D), G(-)}"]
                        &
                        \hom_{\dd}(F(G(D)), -)
                    \end{tikzcd}
                \end{center}
                Composing the natural transformations, we finally
                obtain a natural transformation 
                $\phi^{-1}_{G(D), G(-)} \circ G_{D, -} : \hom_{\dd}(D,
                -) \to \hom_{\dd}(F(G(D)), -)$. How is this natural
                transformation given? We can assign $-$ as $D$ itself,
                and see what happens when we consider the identity
                morphism $1_D: D \to D$. In this case
                \[
                    \phi^{-1}_{G(D), G(D)} \circ G_{D, D}(1_D) 
                    = 
                    \phi^{-1}_{G(D), G(D)}(1_{G(D)})
                    = 
                    \epsilon_{D}
                \]
                by definition of the counit $\epsilon_D$. Now we
                understand how this poorly-notated natural
                transformation works! In general, for and $f: D \to
                D'$, we see that 
                \begin{align}
                    \phi^{-1}_{G(D), G(D')} \circ G_{D, D'}(f) = f \circ \epsilon_{D}.
                \end{align}
                Thus, we see that this natural transformation is in
                disguise; it's actually just $\epsilon_D^*:
                \hom_{\dd}(D, -) \to \hom_{\dd}(F(G(D), -)$!
        \begin{description}
            \item[$\bm{(i)}$]
            \begin{description}
                \item[$\bm{\iff}$]
                If $G$ is faithful, then the natural
                transformation in equation (7) is one to one. This makes
                $\epsilon_D^*$ a monomorphism. By the previous lemma,
                this holds if and only if $\epsilon_D$ is epic for every $D$
                in $\dd$.
             
            \end{description} 
            \item[$\bm{(ii)}$]
            \begin{description}
                \item[$\bm{\iff}$]
                On the other hand, if $G$ is full, then this natural
                transformation in equation (7) surjective. This makes
                $\epsilon_{D}^*$ an epimorphism, and by the previous
                lemma, that holds if and only if $\epsilon_D$ is a split monomorphism. 
            \end{description}   
        \end{description}
    \end{prf}



    \begin{thm}\label{equivalence_theorem}
        Let $F: \cc \to \dd$ be a functor. Then the following are
        equivalent. 
        \begin{description}
            \item[$\bm{(i)}$] $G$ is an equivalence of categories 
            \item[$\bm{(ii)}$] $G$ is part of an adjunction $(F, G,
            \eta, \epsilon)$ where $\eta, \epsilon$ are natural
            isomorphisms 
            \item[$\bm{(iii)}$] $F$ is full and faithful, and each
            object $C$ is isomorphic to $G(D)$ for 
            some object $D$. 
        \end{description} 
    \end{thm}
    Note that this theorem is symmetric; one could interchange $G$
    with $F$, and then obtain the same exact results. \textcolor{MidnightBlue}{Thus, one 
    way of stating this theorem is that $\cc$ and $\dd$ are equivalent
    as categories if and only if there exits full and faithful
    functors $F: \cc \to \dd$ and $G: \dd \to \cc$; or if and only if
    $F, G$ form an adjoint equivalence.}

    \begin{prf}
        \begin{description}
            \item[$\bm{(i) \implies (iii)}$] 
            Suppose we have an equivalence of categories given by $F:
            \cc \to \dd$ and $G: \dd \to \cc$, with natural
            isomorphisms 
            \[
                \phi: F \circ G \cong I_\dd \qquad \psi: G \circ F \cong I_\cc.
            \]
            Let $f: C \to C'$ be a morphism in $\cc$. Then observe
            that the following diagram
            \begin{center}
                \begin{tikzcd}[column sep = 1.4cm, row sep = 1.4cm]
                    C \arrow[d, "f"]\\
                    C'
                \end{tikzcd}
                \hspace{1cm}
                \begin{tikzcd}[column sep = 1.4cm, row sep = 1.4cm]
                    G(F(C)) \arrow[r, "\psi_C"] \arrow[d, swap, "G(F(C))"]
                    &
                    C \arrow[d, "f"]\\
                    G(F(C')) \arrow[r, swap, "\psi_C'"] 
                    &
                    C'
                \end{tikzcd}
            \end{center}
            is commutative. In an equations, we have that $f = \psi_C'
            \circ G(F(f)) \circ \psi_{C'}^{-1}$. Thus suppose that
            $f_1, f_2: C \to C'$ are two morphisms such that $F(f_1)
            =F(f_2)$. Then we get a pair of commutative diagrams,
            similar to the ones above, which translate into the
            equations 
            \[
                f_1 = \psi_{C}'\circ G(F(f_1)) \circ \psi_{C'}^{-1}  
                \qquad 
                f_2 = \psi_{C'}\circ G(F(f_2)) \circ \psi_{C'}^{-1}.
            \]
            Then if $F(f_1) = F(f_2)$, the above equations guarantee
            that $f_1 = f_2$. Hence we see that $F$ is a faithful
            functor. Since the statement is symmetric in both $F$ and
            $G$, we have also that $G$ is faithful.
            
            To show that $F$ is full, suppose there exists a morphism
            $h: F(C) \to F(C')$ for a pair of objects $C, C'$. Let $f
            = \psi_{C'} \circ G(h) \circ \psi_C$. Then we have the
            commutative squares
            \begin{center}
                \begin{tikzcd}[column sep = 1.4cm, row sep = 1.4cm]
                    G(F(C)) \arrow[r, "\psi_C"] \arrow[d, swap, "G(h)"]
                    &
                    C \arrow[d, "f"]\\
                    G(F(C')) \arrow[r, swap, "\psi_C'"] 
                    &
                    C'.
                \end{tikzcd}
                \hspace{1cm}
                \begin{tikzcd}[column sep = 1.4cm, row sep = 1.4cm]
                    G(F(C)) \arrow[r, "\psi_C"] \arrow[d, swap, "G(F(f))"]
                    &
                    C \arrow[d, "f"]\\
                    G(F(C')) \arrow[r, swap, "\psi_C'"] 
                    &
                    C'.
                \end{tikzcd}
            \end{center}
            and hence we have that $G(h) = G(F(f))$. But since $G$ is faithful, 
            this implies that $h = F(f)$. Hence we have that there
            exists a $f': C \to C'$ such that $h = F(f)$, so that $F$
            is full. Again, by symmetry, we have that $G$ is full,
            as desired. 

            Now since $\phi: G \circ F \cong I_\cc$, we see that every
            object $C$ is assigned an isomorphism $\phi_C: G(F(C)) \to
            C$. Hence every object $C$ is isomorphic to some $G(D)$
            where $D = F(C)$. 
            
            Similarly, since $\psi: F \circ G \cong I_\dd$, we know
            that each object $D$ is assigned an isomorphism $\psi_D:
            F(G(D)) \to D$. Hence every object $D$ is isomorphic to
            some object $F(C)$ for $C = G(D)$. 

            \item[$\bm{(iii) \implies (ii)}$]
            Suppose $(iii)$ holds. For any  arbitrary object $C \in
            \cc$, there exists an isomorphism $\eta_C: C \to G(D)$ for some
            object $D \in \dd$. Denote such an object as $F_0(C)$. 
            Now consider any other morphism $g: C
            \to G(D')$. Then we have that 
            \begin{center}
                \begin{tikzcd}[column sep = 1.4cm, row sep = 1.4cm]
                    C \arrow[r, "\eta_C"] \arrow[dr, swap, "g"] & G(F_0(C)) \arrow[d, dashed, "g \circ f^{-1}"]\\
                    & G(D') 
                \end{tikzcd}
            \end{center}
            is commutative. Now since $g \circ \eta_C^{-1}: G(F_0(C)) \to
            G(D')$, and because $G$ is full, we know that there exists
            a $h: F_0(C) \to D'$ such that $g \circ \eta_C^{-1} = G(h)$. To show
            that this is unique, suppose there existed another $k:
            G(F_0(C)) \to G(D')$ such that $g = k \circ \eta_C$. Then by the
            same argument, there exists a $h': F_0(C) \to D'$ such that
            $G(h') = k$. Furthermore, we'll have that 
            \[
                k = G(h') = g \circ \eta_C^{-1} \qquad G(h) = g \circ \eta_C^{-1}
            \]
            so that $G(h') = G(h)$. However, since $G$ is faithful, we
            have that $h' = h$. Hence, $h$ is unique! 

            Since $h$ is unique, this implies that $\eta_C : C \to G(F_0(C))$ is
            universal from $C$ to $G$. Since such a universal
            isomorphism exists for each object of $C$, we have by
            Proposition 4.1 that there exists a functor $F: \cc \to
            \dd$ with object function $F_0(C)$ which is left adjoint
            to $G$. Hence we have an adjunction $(F, G, \eta',
            \epsilon)$. However, since universal morphisms are unique,
            we see that $\eta' = \eta$, so that $\eta$, our unit, is a
            natural isomorphism. 
            
            Finally, observe that for any object $D$, we have that 
            \[
                G(\epsilon_D) \circ \eta_{G(D)} = 1_{G(D)}
            \]
            for our adjunction. Since $\eta_{G(D)}$ is an isomorphism,
            we have that $G(\epsilon_D) = \eta_{G(D)}^{-1}$. Sine $G$
            is full and faithful, we see that $\epsilon_D$ must be an isomorphism as well. 
            
            Thus, in total, we have an adjoint equivalence $(F, G, \eta,
            \epsilon)$, as desired. 
            
            \item[$\bm{(ii) \implies (i)}$] This direction is clear,
            since an adjoint equivalence automatically establishes an
            equivalence of categories. 
        \end{description}
        With $(i) \implies (iii) \implies (ii) \implies (i)$, we see
        that all of the conditions are equivalent. 
    \end{prf}

    \begin{example}
        Let $R$ and $S$ be rings and consider the categories 
        $R$\textbf{-Mod} and $S$\textbf{-Mod}. Then there are two different 
        ``product'' categories we can form: The categories $(R \times S)$\textbf{-Mod} 
        and $R\textbf{-Mod}\times S\textbf{-Mod}$
    \end{example}

    Next, we introduce some properties of equivalences. 
    \begin{proposition}
        Let $F: \cc \to \dd$ be an equivalence of categories with the
        corresponding inverse functor $G: \dd \to \cc$. Let $f: C \to
        C'$ be a morphism in $\cc$. Then  
        \begin{description}
            \item[$(i)$] $f$ is a monomorphism (epimorphism)
            if and only if 
            $F(f)$ is a monomorphism (epimorphism)
            \item[$(ii)$] $C$ is initial (terminal) if and only if $F(C)$ is initial (terminal).
        \end{description}
    \end{proposition}

    Consequently, we have that $f$ is an isomorphism (a monomorphism and epimorphism) if and only
    if $F(F)$ is an isomorphism. 
    \textcolor{Red!90}{Note this is not generally true!}
    Additionally, we also have that
    $C$ is a zero object (terminal and initial) if and only if
    $F(C)$ is a zero object. Finally, observe that this
    proposition is symmetric, so that the same conclusions hold
    for morphisms and objects in $\dd$ governed by $G: \dd \to
    \cc$. 

    \begin{prf}
        \begin{description}
            \item[$\bm{(i)}$] 
            \begin{description}
                \item[$\bm{\implies}$] Suppose $f: C \to C'$ is a
                monomorphism. Consider two morphisms $g,h: D \to F(C)$
                such that $F(f) \circ g = F(f) \circ h$. By the
                previous theorem, we know however that there exists an
                object $A$ of $\cc$ such that $D \cong F(A)$. Hence
                there exists an isomorphism $\theta: F(A) \to D$. We
                then have the diagram:
                \begin{center}
                    \begin{tikzcd}
                        F(A) \arrow[r, "\theta"]
                        &
                        D \arrow[r, swap, shift right =0.6ex, "g"]
                        \arrow[r,shift right =-0.6ex, "h"] 
                        &
                        F(C) \arrow[r, "F(f)"]
                        &
                        F(C')
                    \end{tikzcd}
                \end{center}
                Note that $h \circ \theta, g \circ \theta: F(A) \to
                F(C)$. Since $F$ is full, we know that there exists 
                morphism $k, k': A \to C$ such that $g \circ \theta =
                F(k)$ and $h \circ \theta = F(k')$. 
                Now observe that 
                \begin{align*}
                    &F(f \circ k) = F(f) \circ F(k) = F(f) \circ h \circ \theta\\
                    &F(f \circ k')= F(f) \circ F(k') = F(f) \circ g \circ \theta.
                \end{align*}
                However, since $F(f) \circ h = F(f) \circ g$, we see
                that $F(f \circ k) = F(f \circ k')$. However, since
                $F$ is faithful, we have that $f \circ k = f \circ
                k'$. But since $f$ is a monomorphism, we have that $k
                = k'$. Hence $F(k) = F(k') \implies g \circ \theta = k
                \circ \theta$, and since $\theta$ is an isomorphism,
                we have that $h = g$. Therefore, $F(f)$ is also monic.

                \item[$\bm{\impliedby}$] 
                Suppose $f: C \to C'$ and $F(f)$ is monic. Consider
                two morphism $g, h: A \to C'$ in $\cc$, and suppose
                that $f \circ g = f \circ k$. Then $F(f) \circ F(g) =
                F(f) \circ F(k) \implies F(g) = F(k)$, since $F(f)$ is
                monic. However, $F$ is faithful, so that $g = k$. Hence $f$ is monic as well.
            \end{description}  

            \item[$\bm{(ii)}$] 
            \begin{description}
                \item[$\bm{\implies}$] Suppose $C$ is initial in
                $\cc$. Let $D$ be an object in $\dd$. Then observe
                that, since $\cc$ and $\dd$ are equivalent, there
                exists an isomorphism $\theta: F(A) \to D$ for some
                object $A$ of $\cc$. Since $C$ is initial, we know
                that there exists a unique morphism $f_C: C \to A$.
                Hence $F(f_C): F(C)\to F(A)$. We then have that $F(f_c)
                \circ \theta : F(C) \to D$. Hence there exists a
                morphism from $F(C)$ to $D$. 

                Now suppose $f_1, f_2; F(C) \to D$. Then $\theta^{-1}
                \circ f_1, \theta^{-1}\circ f_2: F(C) \to F(A)$. Since
                $F$ is full, we know that there exist morphism $k_1,
                k_1: C \to A$ such that $F(k_1) = \theta^{-1}\circ
                f_1$ and $F(k_2) = \theta^{-1}\circ f_2$. However,
                since $C$ is initial, we see that $k_1 = k_2 = f_C$.
                Hence $f_1 = f_2$, so that there is exactly one
                morphism $f_1=f_2:F(C) \to D$. 

                Since $D$ was an arbitrary object of $\dd$, we have
                that $F(C)$ is initial. 
                
                \item[$\bm{\impliedby}$] 
                Suppose $F(C)$ is an initial object. Consider any
                object $C'$ of $\cc$. Then since $F(C)$ is initial, there exists a unique
                morphism $f: F(C) \to F(C')$. Since $F$ is full, we
                know that this corresponds with a morphism $k: C \to
                C'$ such that $F(k) = f$. Hence we have a unique
                morphism $k: C \to C'$. And since $C'$ was an 
                arbitrary object of $\cc$, we have that $C$ is
                initial, as desired. 
            \end{description}  
        \end{description}
    \end{prf}
    The proofs in which we proved $f$ to be an epimorphism, and for $C$ 
    to be a terminal object, are very similar. This proposition will soon be
    generalized, but this gives us insight into how useful the concept
    of equivalent categories truly is. 

    

    \newpage
    \section{Adjoints on Preorders.}

    Interesting things happen when one applies adjoint concepts to
    functors between preorders; ones which preserve order in a special
    way. It's actually often the case where we have two mathematical
    structures involving chains of arrows which reverse when
    transferring between one and the other. We give such a concept a
    definition first, before introducing a theorem about such structures. 
     
    \begin{definition}
        Let $\pp$ and $\qqq$ be two preorders. If there exists functors 
        $F: \pp \to \qqq$ and $G:\qqq \to \pp$ such that 
        \[
            F(P) \le Q \iff P \le G(Q),  
        \]
        That is, there exists $f:F(P) \to Q$ if and only if there
        exists $g: P \to G(Q)$, then $F$ and $G$ are called a
        \textbf{monotone Galois connection}. On the other hand, 
        if we have that 
        \[
            F(P) \le Q \iff P \ge G(Q)
        \]
        then $F$ and $G$ are called a \textbf{antitone Galois
        connection}. 
    \end{definition}

    \begin{thm}
        Let $\mathcal{P}, \mathcal{Q}$ be two preorders, and suppose
        $F: \mathcal{P} \to \mathcal{Q}\op$ and
        $G:\mathcal{Q}\op \to \mathcal{P}$ are two order preserving 
        functors. Then $F$ is
        left adjoint to $G$ if and only if for all $P \in \mathcal{P}$
        and $Q \in \mathcal{Q}$ 
        \[
            F(P) \ge Q \iff P \le G(Q).
        \]
        Given such an adjunction, we then have that our unit
        establishes $P \le G(F(P))$ and the counit establishes $F(G(Q)) \le
        Q$. 
    \end{thm}

    \begin{prf}
        Observe that if $F$ is left adjoint to $G$, then we have the 
        bijection 
        \[
            \hom_{\mathcal{Q}\op}(F(P), Q) \cong \hom_{\mathcal{P}}(P, G(Q)
            )
        \]
        which gives rise to the desired correspondence; on the other
        hand, such a bijection gives rise to an adjunction. 
        With such an adjunction, we know that for each $P, Q$, there
        exist morphisms $\eta_P: P \to G(F(P))$ and $\epsilon_Q:
        F(G(Q)) \to Q$. Hence $P \le G(F(P))$  and $F(G(Q)) \ge Q$. 
    \end{prf}

    The above theorem came out of the observation that there is a
    connection between fields, their subfields, and their groups of
    automorphisms, an observation which arises in Galois Theory.
    The goal of Galois Theory is to understand polynomials and their
    roots; when they can be factorized, when and where we can find
    their roots. The study of Galois groups is now used widely in
    number theory. For example, part of Andrew Wiles' work in proving
    Fermat's Last Theorem involved Galois representations.

    
    It was this theorem, rooted in Galois Theory, that motivated the
    Theorem 4.\ref{galois_connections} at the beginning of this
    section. 
    The Fundamental Theorem
    of Galois Theory is simply a \textit{stronger}, special case, since in
    this case, the functors are literally inverses of each other. The
    theorem we introduced, however, simply requires the functors to be
    adjoints of one another. 

    \begin{example}
        Let $U, V$ be sets, and observe that their power sets $\mathcal{P}
        (U)$ and $\mathcal{P}(V)$ form categories; specifically,
        preorders, ordered by set inclusion. 

        Suppose $f: U \to V$ is a function in \textbf{Set}. Then $f$
        induces a functor $f_*: \mathcal{P}(U) \to \mathcal{P}(V)$,
        where 
        \[
            f_*(X) = \{f(x) \mid x \in X\}.
        \]
        Note that if $X\subset X'$, then $f_*(X) \subset f_*(X')$.
        Hence this is an order-preserving functor. Now observe that
        $f$ also induces a functor $f^*:
        \mathcal{P}(V) \to \mathcal{P}(U)$ where 
        \[
            f^*(Y) = \{x \mid f(x) \in Y\}.
        \] 
        Note that this also preserves order. In addition, we have that if 
        $f_*(X) \le Y$, then this holds if  and only if $f(X) \subset
        Y$. We then have that this holds if and only if $X \subset
        f_*(Y)$, Hence we have a Galois connection, so that we may apply 
        Theorem 4.\ref{galois_connections} to conclude that $f_*$ is
        left adjoint to $f^*$.  
    \end{example}





    


    \newpage
    \section{Exponential Objects and Cartesian Closed Categories.}
    Before we introduce the notion of cartesian closed category, we 
    begin with a preliminary proposition. 

    \begin{proposition}
        Suppose $\cc$ is a category, and consider the functors 
        \[
            U: C \to \textbf{1} \qquad \Delta: \cc \to \cc \times \cc.
        \]
        where $\textbf{1}$ is the one object category.
        \begin{description}
            \item[$\bm{(i)}$] If $U$ has a left adjoint, then 
            $\cc$ has an initial object.   

            \item[$\bm{(ii)}$] If $\Delta$ has a left adjoint, then
            $\cc$ has finite coproducts.

            \item[$\bm{(iii)}$] If $U$ has a right adjoint, then $\cc$
            has a terminal object. 

            \item[$\bm{(iv)}$] If $\Delta$ has a right adjoint, then 
            $\cc$ has finite products.  
        \end{description}
    \end{proposition}
    The proof is a straightforward, although tedious, so we sketch it
    out as follows. 

    \begin{prf}
        
    \textcolor{RedViolet}{\textbf{Adjoints of $\bm{U}.$}}
    First, let $F: \textbf{1} \to \cc$ be a left adjoint of $U$. Suppose
    $F(1) = I$ in $\cc$. Then for any $C \in \cc$, we have the bijection
    $\hom_{\cc}(F(1), C) \cong \hom_{\textbf{1}}(1, U(C))$ which
    implies that
    \[ 
        \hom_{\cc}(I, C) 
        \cong \hom_{\textbf{1}}(1, 1).
    \]
    In other words, for each object $C$, there is exactly one and only
    one morphism $i_C: I \to C$, which makes $I$ an initial object. 
    
    On
    the other hand, suppose $G: {1} \to \cc$ is a right adjoint of
    $U$. Then if $G(1) = T$, we have the bijection 
    $\hom_{\textbf{1}}(U(C), 1) \cong \hom_{\cc}(C, G(1))$
    which implies that 
    \[
        \hom_{\textbf{1}}(1, 1) \cong \hom_{\cc}(C, T)
    \]
    so that for each object $C$ there exists a unique morphism $t_C: C
    \to T$, which makes $T$ a terminal object. Hence left and right
    adjoints guarantee the existence of initial and terminal objects.\\
    \\
    \textcolor{RedViolet}{\textbf{Adjoints of $\bm{\Delta}$.}}
    Let $F: \cc \times \cc \to \cc$ be a left adjoint of $\Delta$, so
    that we have the relation
    \begin{center}
        \begin{tikzcd}[column sep = 1.4cm, row sep = 1.4cm]
            \cc \times \cc \arrow[r, shift left=0.6ex,"F"] & \cc 
            \arrow[l,shift right =-0.6ex, "\Delta"]
        \end{tikzcd}
    \end{center}
    Then for each object $(A, B) \in \cc \times \cc$, we have the
    morphism $\eta_{(A,B)}: (A, B) \to \Delta(F(A,B))$, which we can 
    rewrite as $\eta_{(A,B)}: (A, B) \to (F(A,B), F(A,B))$.
    We can put this into a universal diagram 
    \begin{center}
        \begin{tikzcd}[column sep = 1.4cm, row sep = 1.4cm]
            (A, B) \arrow[r, "\eta_{(A,B)}"] 
            \arrow[dr, swap, "f"]
            & (F(A, B), F(A,B)) 
            \arrow[d, dashed, "{(g' , g')}"]
            \\
            & (C, D)
        \end{tikzcd}
        \hspace{1cm}
        \begin{tikzcd}[column sep = 1.4cm, row sep = 1.4cm]
            (A, B) \arrow[r, "{(i,j)}"]
            \arrow[dr,swap, "f"]
            & (A \amalg B, A \amalg B) 
            \arrow[d, dashed, "{(f', f')}"]
            \\
            & (C, D) 
        \end{tikzcd}
    \end{center}
    where the diagram on the right is the coproduct diagram of $A \times
    B$. Since both of the pairs $\Big((F(A,B), F(A,B)), \eta_{(A,B)}\Big)$ 
    and $\Big((A\times B, A \times B), (\pi_A, \pi_B)\Big)$ are universal
    from $(A, B)$ to $\Delta$, they must be isomorphic. As two
    universal objects are isomorphic, we therefore have, 
    \[
        F(A, B) \cong A \amalg B
    \]
    so that a left adjoint gives rise to products. 
    

    Let $G: \cc \times \cc \to \cc$ be a right adjoint of $\Delta$, so
    that we have 
    \begin{center}
        \begin{tikzcd}[column sep = 1.4cm, row sep = 1.4cm]
            \cc \arrow[r, shift left=0.6ex,"\Delta"] & \cc \times \cc
            \arrow[l,shift right =-0.6ex, "G"]
        \end{tikzcd}
    \end{center}
    The adjunction gives rise to a universal morphism
    $\epsilon_{(A,B)}: \Delta(G(A,B)) \to (A,B)$, which we can rewrite
    as $\epsilon_{(A,B)}: (G(A,B), G(A,B)) \to (A,B)$. We then have
    the diagram 
    \begin{center}
        \begin{tikzcd}[column sep = 1.4cm, row sep = 1.4cm]
            (A, B) 
            & (G(A, B), G(A,B)) 
            \arrow[l, swap, "\epsilon_{(A,B)}"] 
            \\
            & (C, D) \arrow[ul, "f"]
            \arrow[u, dashed, swap, "{(g' , g')}"]
        \end{tikzcd}
        \hspace{1cm}
        \begin{tikzcd}[column sep = 1.4cm, row sep = 1.4cm]
            (A, B) 
            & (A \times B, A \times B) 
            \arrow[l, swap, "{(\pi_A, \pi_B)}"]
            \\
            & (C, D) 
            \arrow[ul, "f"]
            \arrow[u, swap, dashed, "{(f', f')}"]
        \end{tikzcd}
    \end{center}
    where the diagram on the right is the product diagram of $A \times B$. 
    Thus we see that\\
    $\Big((G(A,B), G(A, B)), \epsilon_{(A,B)}\Big)$ 
    and $\Big((A\times B, A \times
    B), (\pi_A, \pi_B)\Big)$ are both universal from $\Delta$ to $(A, B)$.
    As universal objects from the same construction are isomorphic, we
    have that 
    \[
        G(A, B) \cong A \times B
    \]
    so that this adjunction gives rise to coproducts. 
    \end{prf}
    
    \textcolor{MidnightBlue}{Thus if we have left and right adjoints
    of the functors $U$ and $\Delta$, we get initial and terminal
    objects as 
    well as finite products and coproducts. Note, however, that finite
    products require (and give rise to) initial objects, and similarly
    that finite coproducts require (and give rise to) terminal
    objects.} 

    Next, we make the following definition. 
    \begin{definition}
        Let $\cc$ be a category with finite products. Suppose $Y, Z$ are 
        objects in $\cc$. We say $Z^Y$ is an \textbf{exponent object} in
        $\cc$ if there exists a morphism $\textbf{eval}: (Z^Y \times Y)
        \to Z$ which is universal from $-\times Y: \cc \to \cc$ to the
        object $Z$. 

        Visually, this translates into requiring that the following
        diagram commutes. 
        \begin{center}
            \begin{tikzcd}[column sep = 1.4cm, row sep = 1.4cm]
                Z & Z^Y \times Y \arrow[l, swap, "{\textbf{eval}}"]\\
                & X \times Y \arrow[ul, "g"] 
                \arrow[u, swap, dashed, "{(h, \text{id}_Y)}"]
            \end{tikzcd}
            \hspace{1cm}
            \begin{tikzcd}[column sep = 1.4cm, row sep = 1.4cm]
                Z^Y \\
                X \arrow[u, dashed, swap, "h"]
            \end{tikzcd}
        \end{center}
    \end{definition}
    \textcolor{Purple}{Hence, every morphism,
    with the domain being any product with $\bm{Y}$, and codomain being
    $\bm{Z}$, uniquely factors through $\bm{Z^Y \times Y}$.}

    Here, we'll stop and look at a pretty cool real world example.

    \begin{example}
        Consider the category \textbf{Set}. Then we know that, for any
        two given objects $Y$ and $Z$, we can form a set of functions 
        between the objects:
        \[
            \hom_{\textbf{Set}}(Y, Z).
        \]
        Thus, the collection of morphisms from sets $Y$ to $Z$ is
        \textit{itself a set}, and hence a member of \textbf{Set}.
        Now let $A$ be any object in $\textbf{Set}$, and let 
        \[
            X = \{f \in \textbf{Set} \mid f: A \times Y \to Z\}.            
        \]
        Define $\textbf{eval}: \hom_{\textbf{Set}}(Y, Z)\times Y \to
        Z$ as, who would've guessed, the evaluation:
        \[
            \textbf{eval}(f(y), y') = f(y').
        \] 
        Now for each $a \in A$, we can define a function $g_a: X
        \times Y \to Z$ where for each $f: A \times Y \to Z$
        \[
            g_a(f, y') = f(a, y') \in Z
        \]
        so this is sort of a "double" evaluation function.
        Then for every such $g_a$, there exists a unique $h_a: X \to
        \hom_{\textbf{Set}}(Y, Z)$ where for each $f: A \times Y \to Z$
        \[ 
            h_a(f) = f(a, y): Y \to Z.
        \] 
        Thus we get the following commutative diagram:
        \begin{center}
            \begin{tikzcd}[column sep = 1.4cm, row sep = 1.4cm]
                Z & \hom_{\textbf{Set}}(Y, Z) \times Y \arrow[l, swap, "
                {\textbf{eval}}"]\\
                & X \times Y \arrow[ul, "g_a"] 
                \arrow[u, swap, dashed, "{(h_a, \text{id}_Y)}"]
            \end{tikzcd}
            \hspace{1cm}
            \begin{tikzcd}[column sep = 1.4cm, row sep = 1.4cm]
                Z^Y \\
                X \arrow[u, dashed, swap, "h_a"]
            \end{tikzcd}
        \end{center}
        What is this? What's really going on and why do we care?\\
        \textcolor{MidnightBlue}{This construction relates to a concept in
        computer science called
        \textbf{currying}. Applied category theory in computer science
        generally works in \textbf{Set}, so that's why this idea
        transfers over.
        \\
        \\
        The idea is: given a multivariable function, do we evaluate
        all arguments at once, or evaluate just one argument, thereby
        sending a function to another function? Both methods can offer 
        advantages. But universality tells us that, in the end, they're the same thing.}

        We can think of $X \times Y$ as being elements $(f(a, y),
        y')$ where $f: A \times Y \to Z$. Then $h$ evaluates $f(a',y)$ for
        some $a'$, thus sending the function $f: A \times Y \to Z$ to the 
        function $f:Y \to Z$. That is, 
        \[
            (h \times \id_y) \circ \big( (f(a, y), y') \big) = (f(a', y), 
            y').
        \]
        Finally, $\textbf{eval}$ evaluates $f(a', y)$
        at $y'$, returning an object in $Z$.

        Alternatively, we can start with the object $(f(a, y), y')$, and
        simply act on $g$, which evaluates it at both $a'$ and $y'$,
        returning the same object $f(a', y')$. 
        Thus in the realm of computer science, we may think of the
        morphisms $(h, \id_y)$, $g$ and \textbf{eval} as commands, as this
        is how currying is often done. 

        The universality of this constructions states that both methods
        are the same; that is, 
        \[
            g = \textbf{eval}\circ (h \times \id_Y).
        \]

        Since we started with arbitrary objects in \textbf{Set}, the
        consequence for computer science is that we can always curry
        these functions. Typically what is curried are types, such as 
        \textbf{Bool} or \textbf{Int}. 
    \end{example}

    In an arbitrary category of finite products, the exponential
    object is just a generalization of currying. But in \textbf{Set},
    we see that an exponential object exists for any two pairs of
    sets. Thus, can we turn this exponential assignment into a
    functor? Yes,we can. 

    \begin{definition}
        Let $\cc$ have finite products and exponential objects for
        every pair of objects. Then for each $Y$ in $\cc$ we can create an 
        \textbf{exponential functor} $E^Y: \cc \to \cc$ as follows. 
        \begin{description}
            \item[Objects.] For each $Z \in \cc$, we define $E^Y(Z) =
            Z^Y$. 

            \item[Morphisms.] Let $f: A \to B$ be in $\cc$. Then we 
            note that we have the following diagrams. 
            \begin{center}
                \begin{tikzcd}[column sep = 1.4cm, row sep = 1.4cm]
                    A & A^Y \times Y \arrow[l, swap, Red,
                    "{\textbf{eval}_A}"]\\
                    & X \times Y \arrow[ul, "g"] 
                    \arrow[u, swap, dashed, "{(h, \text{id}_Y)}"]
                \end{tikzcd}
                \hspace{0.5cm}
                \begin{tikzcd}[column sep = 1.4cm, row sep = 1.4cm]
                    A^Y \\
                    X \arrow[u, dashed, swap, "h"]
                \end{tikzcd}
                \hspace{1cm}
                \begin{tikzcd}[column sep = 1.4cm, row sep = 1.4cm]
                    B & B^Y \times Y \arrow[l, swap, RoyalBlue,
                     "{\textbf{eval}_B}"]\\
                    & X \times Y \arrow[ul, "g'"] 
                    \arrow[u, swap, dashed, "{(h', \text{id}_Y)}"]
                \end{tikzcd}
                \hspace{0.5cm}
                \begin{tikzcd}[column sep = 1.4cm, row sep = 1.4cm]
                    B^Y \\
                    X \arrow[u, dashed, swap, "h'"]
                \end{tikzcd}
            \end{center}
            Now observe that we can form the morphism $f \circ
            \textcolor{Red}{\textbf{eval}_A} : A^Y \times Y \to B$. 
            Hence by universality
            of $B^Y$, there exists a unique morphism $h': A^Y \to B^Y$.
            Diagrammatically, we take the above diagram on the right, and
            replace $X$ with $A^Y$ and $g$ with $f \circ \textbf{eval}_A$.
            \begin{center}
                \begin{tikzcd}[column sep = 1.4cm, row sep = 1.4cm]
                    B & B^Y \times Y \arrow[l, swap, RoyalBlue, 
                    "{\textbf{eval}_B}"]\\
                    & A^Y \times Y 
                    \arrow
                    [ul, "f \circ \textcolor{Red}{\textbf{eval}_A}"] 
                    \arrow[u, swap, dashed, "{(h', \text{id}_Y)}"]
                \end{tikzcd}
                \hspace{0.5cm}
                \begin{tikzcd}[column sep = 1.4cm, row sep = 1.4cm]
                    B^Y \\
                    A^Y \arrow[u, dashed, swap, "h'"]
                \end{tikzcd}
            \end{center}
            Since $h$ exists if $f: A \to B$ exists, we therefore define 
            \[
                E^Y(f: A \to B) = h': A^Y \to B^Y
            \]
            where $h'$ is the unique morphism such that 
            \[
                f \circ \textcolor{Red}{\textbf{eval}_A}
                = \textcolor{RoyalBlue}{\textbf{eval}_A} \circ (h', \id_Y).
            \]
        \end{description}    
    \end{definition}
    Note that there's one more cool connection here. If we have a
    category 
    with finite products, and one in which exponential objects exist,
    then we have a morphism $\textbf{eval}_A: A^Y \times Y
    \to A$ which is universal from the functor $-\times Y: \cc \to
    \cc$ 
    to $A$.
    \textbf{Therefore, this is a counit!} There's an adjunction hiding
    here. 

    \begin{proposition}
        Let $\cc$ be a category with finite products and exponential
        objects. Let $Y$ be an object, and define the functors 
        \begin{align*}
            P_Y &= (-) \times Y: \cc \to \cc\\
            E^Y &= (-)^Y: \cc \to \cc.
        \end{align*}
        Then $E^Y$ is right adjoint to $P_Y$ for every $Y \in \cc$.
        Therefore, 
        \[
            \hom_{\cc}(X \times Y, Z) \cong \hom_{\cc}(X, Z^Y)
        \]
        which is natural for all objects $X, Y, Z \in \cc$. 
    \end{proposition}

    \begin{prf}
        For each object $A \in \cc$, the exponential object gives rise
        to a universal morphism $\textbf{eval}_A: A^Y \times Y \to A$.
        So on one hand, we get the diagram on the left
        \begin{center}
            \begin{tikzcd}[column sep = 1.4cm, row sep = 1.4cm]
                A & A^Y \times Y \arrow[l, swap,
                "{\textbf{eval}_A}"]\\
                & X \times Y \arrow[ul, "g"] 
                \arrow[u, swap, dashed, "{(h, \text{id}_Y)}"]
            \end{tikzcd}
            \hspace{0.5cm}
            \begin{tikzcd}[column sep = 1.4cm, row sep = 1.4cm]
                A^Y \\
                X \arrow[u, dashed, swap, "h"]
            \end{tikzcd}
            \hspace{1cm}
            \begin{tikzcd}[column sep = 1.4cm, row sep = 1.4cm]
                A & P_Y(E^Y(A)) \arrow[l, swap,
                "{\textbf{eval}_A}"]\\
                & P_Y(X) \arrow[ul, "g"] 
                \arrow[u, swap, dashed, "{(h, \text{id}_Y)}"]
            \end{tikzcd}
            \hspace{0.5cm}
            \begin{tikzcd}[column sep = 1.4cm, row sep = 1.4cm]
                A^Y \\
                X \arrow[u, dashed, swap, "h"]
            \end{tikzcd}
        \end{center}
        but on the other hand, the diagram on the right is exactly
        equivalent. Hence we see that $\textbf{eval}$ is actually a
        counit $\epsilon_A: P_Y(E^Y(A)) \to A$. Since such a counit
        exists for each $A$, this gives rise to an adjunction, so that
        $E^Y$ is right adjoint to $P_Y$ for every object $Y$ in $\cc$.
    \end{prf}

    Finally, we have everything we need to move onto to the main point
    of this section. 

    \begin{definition}
        Let $\cc$ be a category. We say $\cc$ is a \textbf{cartesian closed 
        category} if the functors 
        \[
            U: \textbf{C} \to \textbf{1} 
            \qquad
            \Delta: \cc \to \cc  \times \cc 
            \qquad 
            P_Y = (-)\times Y: \cc \to \cc
        \]
        have right adjoints. In other words, $\cc$ is 
        \textbf{cartesian closed} if 
        \begin{itemize}
            \item[1.] There exists a terminal object $T$ 
            \item[2.] $\cc$ has finite products 
            \item[3.] An exponential object $A^Y$ for every $A \in
            \cc$ for all $Y$.   
        \end{itemize}
    \end{definition}

    Thus the work we just did was used in showing that our
    three-bullet point list is another definition of a cartesian
    closed category. Often, only one definition or the other is
    offered, and it's not trivial how they're equivalent, so it can be
    confusing. Thus our work shows that either definition is
    equivalent. 

    Some examples include \textbf{Set}, which we already dealt with.
    \textbf{Set} has a terminal object (empty set), has finite
    products, and has an exponential object. More interesting is
    \textbf{Cat}, which is cartesian closed. In this case, \textbf{1}
    is the terminal object, \textbf{Cat} is closed under finite
    products, and the exponential object exists. In this case,
    $\cc^{\bb}$ is simply the functor category! 

    \textcolor{MidnightBlue}{At first, it seemed silly to define
    $\cc^\bb$ as the category of functors \textit{from} $\bb$ to
    $\cc$, since it seemed that it ought to be denoted $\bb^\cc$.
    However, we see that this was really just because of the concept
    of exponentials, which isn't known when being introduced functor
    categories.}
    