\chapter{Duality and Categorical Constructions} 
    \section{$\mathcal{C}\op$ and Contravariance}
    
    \begin{definition}
        Consider a category $\mathcal{C}$. Then we define the
        \textbf{opposite category of $\cc$}, denoted $\mathcal{C}\op$, 
        to be the category where 
        \begin{description}
            \item[Objects.] The same objects of $\cc$.
            \item[Morphisms.] If $f: A \to B$ is a
            morphism of $\cc$, then we let 
            $f\op : B \to A$ be a morphism of $\cc\op$.
        \end{description}
        In this case, composition isn't exactly obvious, so we will explain how that 
        works. 

        Let $f: A \to B$ and $g: B \to C$ be morphisms of $\cc$. Then we obtain 
        morphisms $f\op: B \to A$ and $g\op: C \to B$. In this case $f\op, g\op$ are composable, 
        and we define composition of $\cc\op$, 
        denoted as $\circ\op$, 
        to be the morphism 
        \[
            f\op \circ g\op: C \to A.
        \]
        Moreover, we have the relation $(g \circ f)\op = f\op \circ g\op$.
    \end{definition}

    Taking the opposite category might seem very strange, 
    but we are doing nothing more than just taking the same category 
    and swapping the domain and codomain of every
    morphism. 
    
    Consequently, many properties of morphisms are similarly reversed.
    For example, if $f: A \to B$ is
    monomorphism in $\cc$, then $f\op: B \to A$ is an epimorphism in 
    $\cc\op$. 
    More generally, every logically valid statement that can be made in
    $\cc$ using its objects and morphisms can be dualized to achieve an
    equivalent, logically valid statement in $\cc\op$ using its
    objects and morphisms. 

    \begin{example}
        Consider a category $\cc$ containing $3$ objects
        whose morphisms are arranged as follows:
        \begin{center}
            \begin{tikzcd}[scale = 2]
                & A\arrow[dr, "f"]
                \arrow[out=120,in=60,looseness=3,loop, "1_A"] &\\
                C \arrow[ur, "h"]
                \arrow[out=120,in=60,looseness=3,loop, "1_C"] & & B
                \arrow[ll, "g"] \arrow[out=115,in=65,looseness=4,loop,
                "1_B"]
            \end{tikzcd}
        \end{center}
        What does the dual category $\cc\op$ look like? Well, $\cc\op$
        contains the same objects $A, B$ and $C$. As for the morphisms, $\cc$
        has the three morphisms $f, g, h$, in addition to their composites.
        Therefore, $\cc\op$ also has three morphisms 
        $f\op:B \to A$, $g\op: C \to
        B$ and $h\op: A \to C$ and their composites. Hence, $\cc\op$ looks like this:
        \begin{center}
            \begin{tikzcd}[scale = 2]
                & A\arrow[dl, swap, "h"]
                \arrow[out=120,in=60,looseness=3,loop, "1_A"] &\\
                C \arrow[rr, swap, "g"]
                \arrow[out=120,in=60,looseness=3,loop, "1_C"] & &
                \arrow[ul, swap, "f"]
                B \arrow[out=115,in=65,looseness=4,loop,
                "1_B"]
            \end{tikzcd}
        \end{center}
    \end{example}

    \begin{example}
        Let $P$ be a preorder, specifically a partial order. 
        Recall that this means that $P$ has 
        a binary relation  $\le$ and if $p \le p'$
        and $p' \le p$, then $p = p'$. 
        
        We claim that that $P\op$ is still a partial order. But first, 
        what does $P\op$ even look like? If we have some elements $p_1, p_2, p_3$ in
        $P$ such that 
        \[
            p_1 \le p_2 \le p_3
        \]
        Then, as a category, $P$ has the unique morphisms $f: p_1 \to p_2$ and 
        $g: p_2 \to p_3$. Hence, in $P\op$, we have the unique morphisms 
        $g\op : p_3 \to p_2$ and $f\op:p_2 \to p_1$, 
        so that we obtain a reversed binary relation $\le\op$ in $P$, which 
        reorder $p_1, p_2, p_3$ as below.
        \[
            p_3 \le\op p_2 \le\op p_1
        \]
        This is kinda weird to write, and in fact, it makes 
        more sense if we 
        write $\le\op = \ge$ as the binary relation in $P\op$. We then have that 
        \[
            p_1 \le p_2 \le p_3 \text{ in } P 
            \implies p_3 \ge p_2 \ge p_1 \text{ in } P\op
        \]
        which is nice!
        Things are even nicer in a linear order, for if $P = \{p_1,
        p_2, p_3, \dots \}$ is a linear order, then we can write that 
        \[
            \cdots p_i \le p_j \le p_k \cdots   
        \]
        and hence in $P\op$ this becomes 
        \[
            \cdots p_i \ge p_j \ge p_k \cdots.
        \]
    \end{example}

    \begin{example}
        Let $(G, \cdot)$ be a group. 
        In group theory one can formulate the \textbf{opposite group} $(G\op, \cdot\op)$ 
        as follows. Define
        $(G\op, \cdot\op)$ to be group with the same set of elements as $G$, 
        whose product $\cdot\op$ works as 
        \[
            g_1 \cdot\op g_2 = g_2 \cdot g_1.   
        \]
        Since both $(G, \cdot)$ and $(G, \cdot\op)$ are groups, we can regard them both 
        as one object categories. What is interesting to realize is that under the 
        categorical interpretation, they are opposite categories of each other.
        
        % However, the categories $\cc$ and $\cc'$ are opposites of each other; that 
        % is, $\cc\op \cong \cc'$. 
    \end{example}

    We thus see that dualizing a category simply involves changing the 
    directions of the morphisms  on the objects. But can we dualize a 
    functor?
    
    % Let $F: \cc \to \dd$ be a functor between two categories. Then $F$
    % induces a functor between $F\op : \cc\op \to \dd\op$, where 
    % \[ 
    %     C \mapsto F(C) \text{ and } f\op \mapsto (Ff)\op
    % \]
    % for $C$ is an object of $\cc\op$ (and hence an object of $\cc$)
    % and $f\op$ is a morphism of $\cc\op$. Thus the induced functor
    % $F\op$ doesn't change how the objects are mapped, but it does
    % change how the morphisms are mapped in the way one would expect it
    % to.

    \begin{definition}
        Let $F: \cc \to \dd$ be a functor and suppose $f: A \to B$ is
        morphism in $\cc$. We say $F$ is a \textbf{contravariant functor}
        if $F(f): F(B) \to F(A)$.
    \end{definition}
    This is in sharp contrast to a \emph{covariant} functor, in which $f: A \to B$
    is sent to $F(f): F(A) \to F(B)$. 

    We next introduce a few examples to demonstrate a contravariant functor. 
    
    \begin{example}
        Let $k$ be an algebraically closed field. Recall that $A^n(k)$ is the set of tuples 
        $(a_1, a_2, \dots, a_n)$ with $a_i \in k$. In algebraic geometry, it 
        is of interest to associate each subset $S \subset A^n(k)$
        with the ideal
        \[
            I(S) = \bigg\{f \in k[x_1, \dots, k_n] \;\bigg|\; f(s) = 0 \text{ for all } s \in S \bigg\}.  
        \]
        of $k[x_1, \dots, x_n]$.
        Observe that this is always non-empty since $0 \in I(S)$ for any $S$. 
        In additional, it is clearly an ideal of $k[x_1, \dots, x_n]$, 
        since for any $p \in k[x_1, \dots, x_n]$,$q \in I(S)$, we have that 
        \[
            (p \cdot q)(s) = p(s)\cdot q(s) = p(s) \cdot 0 = 0 \text{ for all } s \in S.
        \]
        so that $p\cdot q \in I(S)$. Now it's usually an exercise to show that 
        if $S_1 \subset S_2$ are two subsets of $A^n(k)$, then one has that 
        $I(S_2)\subset I(S_1)$. Hence this defines a contravariant functor
        \[
            I: \textbf{Subsets}(A^n(k)) \to \textbf{Ideals}(k[x_1,\dots, x_n]).
        \]
        where $\textbf{Subsets}(A^n(k))$ is the category of subsets with inclusion morphisms, 
        and $\textbf{Ideals}(k[x_1,\dots, x_n])$ is the category of ideals with inclusion 
        ring homomorphisms.
    \end{example}

    \begin{example}
        Consider again $k$ as an algebraically closed field. In algebraic geometry, 
        one often wishes to associated each ideal of $k[x_1, \dots, x_n]$
        with its ``zero set'' 
        \[
            Z(I) = \bigg\{s = (a_1, \dots, a_n) \in A^n(k) \;\bigg|\; f(s) = 0 \text{ for all } s \in I\bigg\}.
        \]
        It is usually an exercise to show that if $I_1 \subset I_2$ are two ideals, 
        then $Z(I_2) \subset Z(I_1)$. Hence we see that this defines a contravariant functor 
        \[
            Z: \textbf{Ideals}(k[x_1, \dots, x_n]) \to \textbf{Subsets}(A^n(k)).
        \]
    \end{example}

    It is usually at the beginning of an algebraic geometry course that one 
    will understand the relationship between these two constructions, which themselves 
    are secretly functors.

    What follows is a very interesting example. In fact, this example is an example of a 
    beautiful concept of a \emph{sheaf}, and it is usually used as a motivating 
    example. But that is for later. 

    \begin{example}
        Let $X$ be a topological space, and 
        consider the thin category $\textbf{Open}(X)$, which contains
        all open sets $U \subset X$, equipped with the inclusion
        function $i_{U, X}: U \to X$. 

        For each $U \in \textbf{Open}(X)$, define the set
        \[
            C(U) = \{f: U \to \rr \mid f \text{ is continuous.}\}
        \]
        Note that if $U \subset V$ are in $\textbf{Open}(X)$, then 
        we define the function 
        $\rho_{U,V}: C(V) \to C(U)$ where 
        \[
            \rho_{U,V}(f: V \to \rr) = f\big|_{U}: U \to \rr.
        \]
        That is, $\rho_{U,V}$ sends continuous, real-valued functions 
        on $V$ to such functions on $U$ by restriction. 
        It is not difficult to show that this respects identity and composition 
        requirements, so that we have a contravariant functor
        \[
            C(-) : \textbf{Open}(X) \to \textbf{Set}
        \]
        for each topological space $X$. 
    \end{example}

    What follows is another very important example. 

    \begin{example}
        Let $\cc$ be a locally small category. In this case, we know that 
        each $A \in \cc$ induces the covariant functor
        \[
            \hom_{\cc}(A, -) : \cc \to \textbf{Set}
        \]
        which sends objects $C$ to the set $\hom_{\cc}(A, C)$. 
        It is natural to ask if we may similarly define a functor  
        \[
            \hom_{\cc}(-, A): \cc \to \textbf{Set}.
        \]
        The answer is yes. We did not make this observation in the past 
        for pedagogical reasons, since it's actually a contravariant functor 
        (and we didn't know what that was until now). We can now safely say 
        that $\hom_{\cc}(-, A)$ is a contravariant functor. 
    \end{example}

    We now comment on the relationship between contravariant and covariant functors. 

    \begin{proposition}
        Let $\cc$, $\dd$ be categories. 
        \begin{itemize}
            \item Let $F: \cc \to \dd$ be a contravariant functor. Then $F$ corresponds to a 
            contravariant functor $\overline{F}: \cc\op \to \dd$ where for a $f\op : B \to A \in \cc\op$, 
            \[
                \overline{F}(f\op : B \to A) = F(f: A \to B) = F(f): F(B) \to F(A).
            \] 
            \item Conversely, let $F: \cc \to \dd$ be a covariant functor. Then $F$ 
            corresponds to a contravariant functor $\overline{F}: \cc\op \to \dd$ 
            where 
            \[
                \overline{F}(f\op : B \to A) = F(f: A \to B) = F(f): F(A) \to F(B)
            \]
        \end{itemize}
    \end{proposition}

    The above proposition allows us to treat any functor as covariant or 
    contravariant. Thus, if we don't like the behavior of our functor on morphisms, we can 
    find an equivalent functor that behaves on morphisms in our preferred way. 

    Generally, covariant functors are easier to think about, so we often like 
    to turn contravariant functors into covariant functors. 

    \begin{example}
        Recall that the functor 
        \[
            C(-): \textbf{Open}(X) \to \textbf{Set}
        \]
        is contravariant. What if we want to treat this as a covariant functor? 
        Well, we can define the functor 
        \[
            \overline{C}(-): \textbf{Open}(X)\op \to \textbf{Set}            
        \]
        as follows. If $U \subset V$ are open subsets of the topological space $X$, 
        then let $i: U \to V$ be the inclusion. This is a morphism in $\textbf{Open}(X)$.
        Hence, $i\op: V \to U$ is a morphism in $\textbf{Open}(X)\op$. 
        Therefore, we define
        \[
            \overline{C}(i\op: V \to U) = C(i: U \to V) = \rho_{U,V}: C(V) \to C(U).
        \]
        Thus we see that this functor $\overline{C}$ acts the same way as $C$, except 
        it behaves covariantly on the morphisms now instead of contravariantly. 
    \end{example}

    \newpage
    \section{Products of Categories, Functors}
    As one may expect, the product of categories can be easily
    defined.
    
    \begin{definition}
        Let $\cc$ and $\dd$ be categories. Then the \textbf{product
        category} $\cc \times \dd$ is the category where 
        \begin{description}
            \item[Objects.] All pairs  $(C, D)$ with $C \in \ob(\cc)$ and $D \in \ob(\dd)$
            \item[Morphisms.] All pairs $(f, g)$ where $f \in \hom(\cc)$ and $g \in \hom(\dd)$.   
        \end{description}
        To define composition in this category, suppose we have composable morphisms in $\cc$ 
        and $\dd$ as below. 
        \begin{center}
            \begin{tikzpicture}
                \filldraw[rounded corners, yellow!30]
                (-3.75,0.25) rectangle (3.75,2.5);
                \node at (-2.75, 2){$\cc$};
                \node at (0,1.25){
                    \begin{tikzcd}[column sep = 1.4cm, row sep =  1.4cm]
                        \cdots
                        &[-1.5cm] 
                        C_1
                        \arrow[r, "f"]
                        \arrow[rr, bend left, "f' \circ f"]
                        &
                        C_2
                        \arrow[r, "f'"]
                        &
                        C_3
                        &[-1.5cm]
                        \cdots
                    \end{tikzcd}
                };
                \begin{scope}[xshift = 8cm]
                    \filldraw[rounded corners, yellow!30]
                    (-3.75,0.25) rectangle (3.75,2.5);
                    \node at (-3, 2){$\dd$};
                    \node at (0,1.25){
                        \begin{tikzcd}[column sep = 1.4cm, row sep =  1.4cm]
                            \cdots
                            &[-1.5cm]
                            D_1
                            \arrow[r, "g"]
                            \arrow[rr, bend left, "g' \circ g"]
                            &
                            D_2
                            \arrow[r, "g'"]
                            &
                            D_3   
                            &[-1.5cm]
                            \cdots           
                        \end{tikzcd}  
                    };
                \end{scope}
            \end{tikzpicture}                
        \end{center}
        Then the morphisms $(f, g)$ and $(f', g')$ in $\cc \times \dd$ 
        are composable too, and their composition is defined as $(f', g') \circ (f , g) = ( f' \circ f, g' \circ g)$.
        \begin{center}
            \begin{tikzpicture}
                \filldraw[rounded corners, yellow!30]
                (-5,-2.25) rectangle (5,0.25);
                \node at (-3.5, -0.25){$\cc \times \dd$};
                \node at (0,-1){
                    \begin{tikzcd}[column sep = 1.4cm, row sep =  1.4cm]
                        \cdots
                        &[-1.5cm]
                        (C_1,D_1) 
                        \arrow[r, "{(f,g)}"]
                        \arrow[rr, bend left, "{(f', g') \circ (f, g) = (f' \circ f, g' \circ g)}"]
                        &
                        (C_2, D_2)
                        \arrow[r, "{(f',g')}"]
                        &
                        (C_3, D_3)
                        &[-1.5cm]
                        \cdots
                    \end{tikzcd}  
                };
            \end{tikzpicture}
        \end{center}           
        We also define the \textbf{projection functors}
        $\pi_\cc:
        \cc\times\dd \to \cc$ and $\pi_\dd: \cc\times\dd \to \dd$
        where on objects $(C, D)$ and morphism $(f, g)$, we have that 
        \begin{align*}
            &\pi_\cc(C, D) = C \quad &&\pi_\dd(C, D) = D\\
            &\pi_\cc(f, g) = f \quad &&\pi_\dd(f, g ) = g
        \end{align*}
    \end{definition}

    These projection functors have the following property.
    Consider a pair of functors $F: \bb \to \cc$ and $G:\bb \to \dd$.
    Then $F$ and $G$ determine a unique functor $H: \bb \to \cc
    \times \dd$ where 
    \[
        \pi_\cc \circ H = F \qquad \pi_\dd \circ H = G.
    \]
    That is, we see that for any morphism $f$ in $\bb$ we have that $H(f)
    = ( F(f), G(f) )$. Hence the following diagram commutes
    \begin{center}
        \begin{tikzcd}[row sep = 1.5cm, column sep = 1.5cm]
            & \bb \arrow[dl, swap, "F"] \arrow[dr, "G"] \arrow[d, dashed, "H"]&\\
            \cc & \arrow[l, "\pi_\cc"] \cc \times \dd \arrow[r, swap,"\pi_\dd"] & \dd 
        \end{tikzcd}
    \end{center}
    and we dash the middle arrow to represent that $H$ is induced, or
    defined, by this process.

    We can also take the product of two different functors. 

    \begin{definition}
        Let $F: \cc \to \cc'$ and $G: \dd \to \dd'$ be two functors.
        Then we define the \textbf{product functor} to be the functor 
        $F \times G: \cc \times \dd \to \cc' \times \dd'$ for which 
        \begin{itemize}
            \item[1.] If $(C, D)$ is an object of $\cc\times\dd$ then
            $(F\times G)(C, D) = (F(C), G(D))$ 
            \item[2.] If $(f, g)$ is a morphism of $\cc\times\dd$
            then $(F \times G)(f,g) = (F(f), G(g))$ 

            Additionally, we can compose the product of functors (of
            course, so
            long as they have the same number of factors). Thus suppose $G,F$ and $G', F'$
            are composable functors. Then observe that 
            \[
                (G \times G') \circ (F \times F') = (G \circ F) \times (G' \circ F').
            \] 
        \end{itemize}
    \end{definition}

    Note that in this formulation we have that 
    \[
        \pi_{\cc'}\circ (F\times G)  = F \circ \pi_\cc \quad \pi_{\cc'} \circ (F \times G) = G \circ \pi_{\dd}
    \]  
    Hence, we have the following commutative diagram.
    
    \begin{center}
        \begin{tikzcd}[row sep = 1.4cm, column sep = 1.4cm]
            \cc \arrow[d, swap, "F"] & \arrow[l,swap, "\pi_\cc"] \cc \times \dd \arrow[r,
            "\pi_\dd"] \arrow[d, dashed, "F\times G"] & \dd \arrow[d, "G"]\\
            \cc' & \arrow[l, swap, "\pi_{\cc'}"] \cc'\times\dd' \arrow[r,
            "\pi_{\dd'}"] & \dd'
        \end{tikzcd}
    \end{center}
    Again, the dashed arrow is written to express that $F \times G$ is
    the functor defined by this process and makes this diagram
    commutative.
    
    \begin{definition}
        If $F$ is a functor such that $F: \bb \times \cc \to \dd$,
        that is, its domain is a product category, then $F$ is said
        to be a \textbf{bifunctor}.
    \end{definition}

    An example of a bifunctor is the cartesian product $\times$, which
    we can apply to sets, groups, and topological spaces. In these
    instances we know that value of a cartesian product is always
    determined uniquely by the values of the individual factors, which
    holds more generally for bifunctors. 

    \begin{proposition}
        Let $\bb, \cc$ and $\dd$ be categories. For $B \in \bb$ and $C
        \in \cc$, define the functors
        \[
            H_C: \bb \to \dd \quad K_B: \cc \to \dd   
        \]
        such that $H_C(B) = K_B(C)$ for all $B, C$. Then there exists
        a functor $F:\bb \times \cc \to \dd$ where $F(B, -) = K_B$
        and $F(-, C) = H_C$ for all $B, C$ if and only if 
        for every pair of morphisms $f:B \to B'$ and
        $g:C\to C'$ we have that 
        \[
            K_{B'}(g) \circ H_C(f) = H_{C'}(f) \circ K_B(g).
        \]
        Diagrammatically, this condition is
        \begin{center}
            \begin{tikzcd}[column sep = 1.4cm, row sep = 1.4cm]
                H_C(B)=
                K_{B}(C) 
                \arrow[r, "K_{B}(g)"]
                \arrow[d, swap, "H_C(f)"]
                &
                H_{C'}(B)
                =
                K_{B}(C')
                \arrow[d, "H_{C'}(f)"]
                \\
                H_C(B')
                =
                K_{B'}(g)
                \arrow[r, swap, "K_{B'}(g)"]
                &
                H_{C'}(B')
                =
                K_{B'}(C')
            \end{tikzcd}
        \end{center}
    \end{proposition}

    The proof is left as an exercise for the reader.
   

    \begin{example}
    We now introduce what is probably one of the most important examples of 
    a bifunctor. Note that for any (locally small) category $\cc$, 
    we have for each object $A$ a functor.
    \[
        \hom(A, -): \cc \to \textbf{Set}
    \]
    We also have a functor from $\cc\op$ (we at the $\op$ simply for convenience)
    for each $B \in \cc\op$.
    \[
        \hom(-, B): \cc\op \to \textbf{Set}    
    \]
    As an application of the proposition,
    one can  see that that these two functors act as the $K_B$
    and $H_C$ functors in the above proposition, and give rise to
    bifunctor 
    \[ 
        \hom: \cc\op \times \cc \to \textbf{Set}.
    \] 
    This is
    because for any $h: A \to A'$ and $k: B \to B'$, the diagram,
    \begin{center}
        \begin{tikzcd}[row sep = 1.2cm, column sep = 1.4cm]
            \hom(A', B) \arrow[r, "h^*"] \arrow[d, swap, "k_*"] & \hom(A, B)
            \arrow[d, "k_*"]\\
            \hom(A', B') \arrow[r,"h^*"] & \hom(A, B')
        \end{tikzcd}    
    \end{center}
    commutes.
    Hence the proposition guarantees that $\hom:\cc\op\times\cc \to
    \textbf{Set}$ exists and is unique.
    \end{example}

    \begin{example}
        Recall that for an integer $n$ and for a ring $R$ with identity $1 \ne 0$,  
        we can formulate the group $\text{GL}(n, R)$, consisting of $n\times n$ matrices 
        with entry values in $R$. As this takes in arguments, we might guess that we have 
        a bifunctor 
        \[
            GL(-, -): \bm{\mathbb{N}} \times \textbf{Ring} \to \textbf{Grp}
        \]
        where $\bm{\mathbb{N}}$ is a the discrete category with elements as natural numbers. This intuition 
        is correct: for a fixed ring $R$, we have a functor
        \[
            GL(-, R): \bm{\mathbb{N}} \to \textbf{Grp}
        \]
        while for a fixed natural number $n$ we have a functor 
        \[
            GL(n, -): \textbf{Ring} \to \textbf{Grp}.     
        \]
        Below we can visualize the activity of this functor:
        \begin{center}
            \begin{tikzpicture}
                \begin{scope}
                    \begin{tikzcd}[column sep = 0.1cm, row sep = 0.1cm]
                        \vdots
                        &
                        \vdots
                        &
                        \cdots 
                        &
                        \vdots
                        &
                        \cdots
                        \\
                        GL(1, S)
                        &
                        GL(2, S)
                        &
                        \cdots 
                        &
                        GL(k, S)
                        &
                        \cdots
                        \\
                        \vdots
                        &
                        \vdots
                        &
                        \cdots 
                        &
                        \vdots
                        &
                        \cdots
                        \\
                        GL(1, \zz)
                        &
                        GL(2, \zz)
                        &
                        \cdots 
                        &
                        GL(k, \zz)
                        &
                        \cdots
                    \end{tikzcd}
                \end{scope}
                \draw (0, -1.7) -- (8, -1.7);
                \draw (0, -1.7) -- (0, 1.7);
                \node at (1,-2) {$n = 1$};
                \node at (3,-2) {$n = 2$};
                \node at (4.7,-2) {$\cdots $};
                \node at (6.1,-2) {$n=k $};
                \node at (7.75,-2) {$\cdots $};
        
                \node at (-1, -1.4) {$R = \zz$};
                \node at (-1, -0.5) {$\vdots$};
                \node at (-1, 0.4) {$R = S$};
                \node at (-1, 1.3) {$\vdots$};
            \end{tikzpicture}
        \end{center}
        Above, we start with $\zz$ since the this is the initial object of the category 
        \textbf{Ring}. 
    \end{example}
    

    Now that we understand products of categories a functors, and we
    have a necessary and sufficient condition for the existence of a
    bifunctor, we describe necessary and sufficient conditions for the
    existence of a natural transformation.

    \begin{definition}
        Suppose $F, G: \bb \times \cc \to \dd$ are bifunctors. Suppose
        that there exists a morphism $\eta$ which assigns objects of $\bb
        \times \cc$ to morphisms of $\dd$. Specifically, $\eta$ assigns
        objects
        $B \in \bb$ and $C \in \cc$ to the morphism 
        \[
            \eta_{(B, C)} : F(B, C) \to G(B, C).
        \]
        Then $\eta$ is said to be \textbf{natural} in $B$ if, for all
        $C \in \cc$, 
        \[
            \eta_{(-, C)} : F(-, C) \to G(-, C)
        \]
        is a natural transformation of functors from $\bb \to \dd$. 
    \end{definition} 

    With the previous definition, we can now introduce the necessary
    condition for a natural transformation to exist between bifunctors.
    
    \begin{proposition}\label{prop_bifunctors}
        Let $F, G: \bb \times \cc \to \dd$ be bifunctors. Then there
        exists a natural transformation $\eta: F \to G$ if and only if
        $\eta(B, C)$ is natural in $B$ for each $C \in C$, and natural
        in $C$ for each $B \in \bb$.
    \end{proposition}

    \begin{prf}
        \begin{description}
            \item[($\bm{\implies}$)] Suppose that $\eta: F \to G$ is a
            natural transformation. Then every object $(B, C)$ is
            associated with a morphism $\eta_{(B, C)}: F(B, C) \to
            G(B, C)$ in $\dd$, and this gives rise to the following diagram:
            \begin{center}
                \begin{tikzcd}[row sep = 1.2cm]
                    (B, C) \arrow[d, "{(f, g)} "]\\
                    (B', C') 
                \end{tikzcd}
                \hspace{2cm}
                \begin{tikzcd}[row sep = 1.2cm, column sep = 1.4cm]
                    F(B, C) \arrow[r, "{\eta_{(B,C)}}"] \arrow[d,swap,
                    "{F(f,g)}"] 
                    &
                    G(B,C) \arrow[d, "{G(f,g)}"]\\
                    F(B',C') \arrow[r, "{\eta_{(B',C')}}"] 
                    &
                    G(B',C')
                \end{tikzcd}
            \end{center}
            Now let $C \in \cc$ and observe that 
            \[
                \eta_{(-, C)}: F(-, C) \to G(-,c)
            \]
            is a natural transformation for all $B$. On the other
            hand, for any $B \in \bb$, 
            \[
                \eta_{(B, -)}: F(B, -) \to G(B, -)
            \]
            is a natural transformation for all $C$. Therefore, $\eta$
            is both natural in $B$ and $C$ for all objects $(B, C)$

            \item[($\bm{\impliedby}$)] Suppose on the other hand that
            $\eta$ is a function which assigns objects $(B, C)$ to a
            morphism $F(B, C) \to G(B, C)$ in $\dd$. Furthermore,
            suppose that $\eta(B, C)$ is natural in $B$ for all $C \in
            \cc$ and natural in $C$ for all $B \in \bb$. 

            Consider a morphism $(f, g) : (B, C) \to (B', C')$ in
            $\bb \times \cc$. Then since $\eta$ is natural for all $B
            \in \bb$, we
            know that for all $C \in \cc$, 
            \[
                \textcolor{red}{\eta}_{(-, C)} : F(-, C) \to G(-,C)  
            \]
            is a natural transformation. In addition, $\eta$ is
            natural for all $C \in \cc$ since for all $B \in \bb$ 
            \[
                \textcolor{blue}{\eta}_{(B, -)} : F(B, -) \to G(B, -)
            \]  
            is a natural transformation. Hence consider the natural
            transformation $\textcolor{red}{\eta}_{(-, C)}$ acting on
            $(B, C)$ and
            $\textcolor{blue}{\eta}_{(B', -)}$ acting on $(B', C)$.
            Then we get the following commutative diagrams.

            \begin{center}
                \begin{tikzcd}[row sep = 1.2cm, column sep = 1.4cm]
                    F(B, C) \arrow[r, "{\textcolor{red}{\eta}_{(B,C)}}"] \arrow[d,swap,
                    "{F(f, 1_C)}"] 
                    &
                    G(B,C) \arrow[d, "{G(f,1_C)}"]\\
                    F(B',C) \arrow[r, "{\textcolor{red}{\eta}_{(B',C)}}"] 
                    &
                    G(B',C)
                \end{tikzcd}
                \hspace{1cm}
                \begin{tikzcd}[row sep = 1.2cm, column sep = 1.4cm]
                    F(B', C) \arrow[r, "{\textcolor{blue}{\eta}_{(B',C)}}"] \arrow[d,swap,
                    "{F(1_{B'}, g)}"] 
                    &
                    G(B',C) \arrow[d, "{G(1_{B'},g)}"]\\
                    F(B',C') \arrow[r, "{\textcolor{blue}{\eta}_{(B',C')}}"] 
                    &
                    G(B',C')
                \end{tikzcd}
            \end{center}
            Observe that
            the bottom row of the first diagram matches the top row of
            the second. 
            Also note that $f: B \to B'$ and $g: C \to C'$, and that the
            diagrams imply the equations 
            \begin{align}
                G(f, 1_C) \circ \textcolor{red}{\eta}_{(B,C)} &= \textcolor{red}{\eta}_{(B', C)} \circ F(f, 1_C) \label{eq1_prop_2_2}\\
                G(1_{B'}, g) \circ \textcolor{blue}{\eta}_{(B', C)} &= \textcolor{blue}{\eta}_{(B', C')} \circ F(1_{B'}, g). \label{eq2_prop_2_2}
            \end{align}
            Now suppose we compose equation (\ref{eq1_prop_2_2}) with
            $G(1_{B'}, g)$ on the left. Then we get that 
            \begin{align*}
                G(1_{B'}, g)\circ G(f, 1_C) \circ \textcolor{red}{\eta}_{(B, C)} &= \overbrace{G(1_{B'}, g) \circ \textcolor{red}{\eta}_{(B', C)}}^{\text{replace via equation (2)}} \circ F(f, 1_C)\\
                &= \textcolor{blue}{\eta}_{(B', C')} \circ F(1_{B'}, g) \circ F(f, 1_C)\\
                &= \textcolor{blue}{\eta}_{(B',C')} \circ F(1_{B'}\circ f, g \circ 1_C)\\
                &=\textcolor{blue}{\eta}_{(B',C')} \circ F(f, g).
            \end{align*}
            where in the second step we applied equation 
            (\ref{eq2_prop_2_2}), and in the third step we composed
            the morphisms. Also note that we can simplify the left-hand
            side since 
            \[ 
                G(1_{B'}, g)\circ G(f, 1_C) =
            G(1_{B'}\circ f, g \circ 1_C) = G(f, g).
            \] Therefore, we have that 
            \[
                G(f, g) \circ \textcolor{red}{\eta}_{(B, C)} = \textcolor{blue}{\eta}_{(B', C')} \circ F(f, g) 
            \]
            which implies that $eta$ itself is a natural
            transformation. Specifically, it implies the following
            diagram. 
            \begin{center}
                \begin{tikzcd}[row sep = 1.2cm]
                    (B, C) \arrow[d, "{(f, g)} "]\\
                    (B', C') 
                \end{tikzcd}
                \hspace{2cm}
                \begin{tikzcd}[row sep = 1.2cm, column sep = 1.4cm]
                    F(B, C) \arrow[r, "{\eta_{(B,C)}}"] \arrow[d,swap,
                    "{F(f,g)}"] 
                    &
                    G(B,C) \arrow[d, "{G(f,g)}"]\\
                    F(B',C') \arrow[r, "{\eta_{(B',C')}}"] 
                    &
                    G(B',C')
                \end{tikzcd}
            \end{center}
        \end{description}
    \end{prf}
    Note: A way to succinctly prove the
    reverse implication of the previous proof is as follows. Since we
    know the diagrams on the left are commutative, just "\textcolor{Green}{stack}" them
    on top of each other to achieve the diagram in the upper right
    corner, and then "\textcolor{Orange}{squish}" this diagram down to obtain the third
    diagram in the bottom right. 
    
    \begin{minipage}{0.3\textwidth}
        \begin{center}
            \begin{tikzcd}[row sep = 1.2cm, column sep = 1.4cm]
                F(B, C) \arrow[r, "{\textcolor{red}{\eta}_{(B,C)}}"] \arrow[d,swap,
                "{F(f, 1_C)}"] 
                &
                G(B,C) \arrow[d, "{G(f,1_C)}"]\\
                F(B',C) \arrow[r, "{\textcolor{red}{\eta}_{(B',C)}}"] 
                &
                G(B',C)
            \end{tikzcd}
            \vspace{0.5cm}

            \begin{tikzcd}[row sep = 1.2cm, column sep = 1.4cm]
                F(B', C) \arrow[r, "{\textcolor{blue}{\eta}_{(B',C)}}"] \arrow[d,swap,
                "{F(1_{B'}, g)}"] 
                &
                G(B',C) \arrow[d, "{G(1_{B'},g)}"]\\
                F(B',C') \arrow[r, "{\textcolor{blue}{\eta}_{(B',C')}}"] 
                &
                G(B',C')
            \end{tikzcd}
        \end{center}
    \end{minipage}  
    \hspace{1cm}
    \begin{minipage}{0.1\textwidth}
        \begin{tikzpicture}
            \draw[white] (0,0) -- (1,0);
            \draw[thick, Green, ->] (0,-2) -- (2,0);
            \draw[thick, Orange, ->] (2.6,-2) to [bend right = 80] (2.6,-4);
        \end{tikzpicture}

    \end{minipage}
    \hfill
    \begin{minipage}{0.5\textwidth}
        \begin{center}
        \begin{tikzcd}[row sep = 1.2cm, column sep = 1.4cm]
            F(B, C) \arrow[r, "{\textcolor{red}{\eta}_{(B,C)}}"] \arrow[d,swap,
            "{F(f, 1_C)}"] 
            &
            G(B,C) \arrow[d, "{G(f,1_C)}"]\\
            F(B', C) \arrow[r, "{\textcolor{blue}{\eta}_{(B',C)}}"] \arrow[d,swap,
            "{F(1_{B'}, g)}"] 
            &
            G(B',C) \arrow[d, "{G(1_{B'},g)}"]\\
            F(B',C') \arrow[r, "{\textcolor{blue}{\eta}_{(B',C')}}"] 
            &
            G(B',C')
        \end{tikzcd}
        \vspace{1cm}

        \begin{tikzcd}[row sep = 1.2cm, column sep = 1.4cm]
            F(B, C) \arrow[r, "{\textcolor{red}{\eta}_{(B,C)}}"] \arrow[d,swap,
            "{F(f,g)}"] 
            &
            G(B,C) \arrow[d, "{G(f,g)}"]\\
            F(B',C') \arrow[r, "{\textcolor{blue}{\eta}_{(B',C')}}"] 
            &
            G(B',C')
        \end{tikzcd}
        \end{center}
    \end{minipage}
    \vspace{1cm}

    This is essentially what we did in the proof, although this is more crude
    visualization of what happened, and we were more formal throughout
    the process. 

    {\large \textbf{Exercises}
    \vspace{0.5cm}}
    \begin{itemize}
        \item[\textbf{1.}]
        Let $\cc$ and $\dd$ be categories. Prove that 
        $(\cc \times \dd)\op \cong \cc\op\times\dd\op$. 

    
    \end{itemize}

    

\newpage 
\section{Functor Categories}
    In the proof for the last proposition, we used a trick of forming
    a desired natural transformation by composing two composable
    natural transformations. Hence, we see that natural
    transformations can be ``composed.'' We refine this notion as follows.

    Let $\cc$ and $\dd$ be categories and consider three functors
    $F, G, H: \cc \to \dd$.
    Suppose further that we have two natural transformations $\sigma, \tau$ as below:
    \begin{center}
        \begin{tikzcd}
            F \arrow[r, "\sigma"]
            &
            G \arrow[r, "\tau"]
            &
            H
        \end{tikzcd}
    \end{center}
    (This might seem like a weird way to write this, but we are trying to hint at something.)
    Using these two natural transformations,
    we can define a natural transformation 
    \[
        \tau \cdot \sigma: F \to H
    \]
    where, for each $C \in \cc$, we define 
    \[
        (\tau \cdot \sigma)_C = \tau_C \circ \sigma_C: F(C) \to H(C).
    \]
    Visually, we can picture what we are doing as follows.
    For a given morphism $f: A \to B$ in $\cc$,
    we define the morphism $(\tau \cdot \sigma)_C$ as
    \begin{center}
        \begin{tikzcd}[row sep = 1.4cm, column sep = 1.4cm]
            F(A)
            \arrow[lddr, to path= {%
             -|  ([xshift=-3ex]\tikztotarget.west)node[near end,left]{$(\tau\cdot\sigma)_A$}
              |- (\tikztotarget)}]
            \arrow[d, swap, "\sigma_A"]
            \arrow[r, "F(f)"] 
          & F(B)
            \arrow[d, "\sigma_{B}"]
            \arrow[rddl, to path= {%
             -| ([xshift=3ex]\tikztotarget.east)node[near end,right]{$(\tau\cdot\sigma)_{B}$}
             -- (\tikztotarget)}]
          \\
            G(A)
            \arrow[d, swap, "\tau_A"] 
            \arrow[r, "G(f)"] 
          & G(B)
            \arrow[d, "\tau_{B}"]
          \\
            H(A) 
            \arrow[r, "H(f)"]
          & H(B)
        \end{tikzcd} 
    \end{center}
    Thus, we see that natural transformations can be ``composed,''
    and we can thus ask: If we view functors as objects, and view natural transformations 
    as morphisms, do we get a category? The answer is yes. 

    \begin{definition}
        Let $\cc$ and $\dd$ be small categories and consider set of
        all functors $F: \cc \to \dd$. Then the \textbf{functor
        category}, denoted as $\dd^\cc$ or $\fun(\cc, \dd)$, is the
        category where 
        \begin{description}
            \item[Objects.] Functors $F: \cc \to \dd$
            \item[Morphisms.] Natural transformations $\eta: F \to G$   
        \end{description}
    \end{definition}
    Functor categories are extremely useful, as we shall see that 
    they're the categorical version of representations. 
    
    When we think of 
    representations, we usually think of a group homomorphism $\rho: G \to \text{GL}_n(V)$ 
    for some vector space $V$ over a field $k$. However, suppose we wanted to be 
    a real smart-ass and say ``Well, can't we regard $\rho$ as actually a functor 
    between two one-object categories whose morphisms are all isomorphism?''
    The answer is yes! 

    What this then means is that the category of representations of a group $G$
    is actually a functor category. Specifically, 
    \[
        \fun(G, \text{GL}_n(V)) \cong R\textbf{-Mod}.
    \]
    Hence in some cases it helps to think of $\fun(\cc, \dd)$ as a 
    category of representations of $\cc$. This makes sense, since that is really 
    what a functor is. A functor preserves composition; and if we stop thinking 
    like the set theorists, we can realize that composition controls a great deal 
    of structure 
    in a category $\cc$. Hence a functor $F: \cc \to \dd$ ``represents"" that structure in a 
    category $\dd$. 

    \begin{example}
        Let $\bm{1}$ be the one element category with a
        single identity arrow. Then for any category $\cc$, the
        functor category $\cc^{\bm{1}}$ is isomorphic to $\cc$. This
        is because each functor $F: \bm{1} \to \cc$ simply associates
        the element $1 \in \bm{1}$ to an element $C \in \cc$, and the
        identity $1_1: 1 \to 1$ to the identity morphism $1_C$ in
        $\cc$. 
    \end{example}

    \begin{example}
        Let $\bm{2}$ be the category consisting of two
        elements, containing the two identities and one nontrivial
        morphism between the objects. 
        \begin{center}
            \begin{tikzcd}
                1 \arrow[out=120,in=60,looseness=2, loop, "\id_1"] \arrow[r,
                "f"] & 2 \arrow[out=120,in=60,looseness=2,, loop, "\id_2"]
            \end{tikzcd}

            \textit{The category $\bm{2}$.}
        \end{center}
        Now consider the functor category $\cc^{\bm{2}}$ where $\cc$ is
        any category. Each functor $F:\bm{2} \to \cc$ maps the pair of
        objects to objects $F(1)$ and $F(2)$ in $\cc$. However, since
        functors preserve morphisms, we see that 
        \[ 
            f: 1 \to 2 \implies F(f): F(1) \to F(2).
        \]
        This is what each $F \in \cc^{\bm{2}}$ does. Hence, every
        morphism $g \in \hom(\cc)$ corresponds to an element in
        $\cc^{\bm{2}}$. Hence, we call $\cc^{\bm{2}}$ the category of
        arrows of $\cc$. 
        \begin{prf}
            Let $g: C \to C'$ be any morphism between objects $C, C'$
            in $\cc$. Construct the element $G \in
            \cc^{\bm{2}}$ as follows:
            $G(1) = C$, $G(2) = C'$ and $G(f) : G(1) \to G(2) = g$.
            Hence, $\hom(\cc)$ and $\cc^{\bm{2}}$ are isomorphic.
            Moreover, $\hom(\cc)$ determines the members of
            $\cc^{\bm{2}}$. 
            
            A crude way to visualize this proof is
            imaging $1 \to 2$ is a "stick" with 1 and 2 on either end,
            and so the action of any functor is simply taking the
            stick and applying it  to anywhere on the direct graph
            generated by the category $\cc$. Hence, this is why we say
            $\hom(\cc)$ determines the functor category $\cc^{\bm{2}}$.
        \end{prf}
    \end{example}

    \begin{example}
        Let $X$ be a set. Hence, it is a discrete category,
        which if recall, it's objects are elements of $X$ and the
        morphisms are just identity morphisms. 

        Now consider $\{0, 1\}^X$, the category of functors $F: X \to
        \{0, 1\}$. Then every functor assigns each element of $x \in
        X$ to either $0$ or $1$, and assigns the morphism $1_x: x \to
        x$ to either $1_0: 0\to 0$ or $1_1: 1 \to 1$. 

        One way to view this is to consider $\mathcal{P}(X)$, and for
        each $S \in \mathcal{P}$, assign $x$ to $1$ if $x \in  S$ or
        $x$ to $0$ if $x \not\in S$. All of these mappings may be
        described by elements of $\mathcal{P}$, but we can also
        realize that each 
        of these mappings correspond to the functors in $\{0, 1\}^X$.
        Hence, we see that $\{0, 1\}^X$ is isomorphic to
        $\mathcal{P}(X)$. 
    \end{example}

    \begin{example}
        Recall from Example \ref{group_ring_functors} that, given 
        a group $G$ and a ring $R$ (with identity), we can create a 
        \emph{group ring} $R[G]$ with identity, in a functorial way, establishing 
        a functor 
        \[
            R[-]: \textbf{Grp} \to \textbf{Ring}.
        \]  
        However, we then noticed that the above functor establishes a process 
        where we send rings $R$ to functors $R[-]: \textbf{Grp} \to \textbf{Ring}$. 
        It turns out that this process is itself a functor, and we now 
        have the appropriate language to describe it:
        \[
            F: \textbf{Ring} \to \textbf{Ring}^{\textbf{Grp}}
        \]
        Specifically, let $\psi: R \to S$ be a ring homomorphism. 
        Now observe that $\psi$ induces another ring homomorphism 
        \[
            \psi_G^*: R[G] \to S[G] \qquad \sum_{g \in G}a_g g \mapsto \sum_{g \in G}\phi(a_g) g.
        \]
        As a result, we see that such a ring homomorphism induces a natural transformation.
        To show this, let $\phi: G \to H$ be a group homomorphism. Then observe that
        we get the diagram in the middle. 
        \begin{center}
            \begin{tikzcd}[row sep = 1.4cm, column sep = 1.4cm]
                G 
                \arrow[d, "\phi"]
                \\
                H                
            \end{tikzcd}
            \hspace{1cm}
            \begin{tikzcd}[row sep = 1.4cm, column sep = 1.4cm]
                R[G]
                \arrow[d, swap, "R(\phi)"]
                \arrow[r, "\psi_G^*"]
                &
                S[G]
                \arrow[d, "S(\phi)"]\\
                R[H]
                \arrow[r, swap, "\psi_H^*"]
                &
                S[H]
            \end{tikzcd}
            \hspace{1cm}
            \begin{tikzcd}[row sep = 1.4cm, column sep = 1.4cm]
                \sum_{g \in G} a_g g 
                \arrow[r, maps to]
                \arrow[d, maps to]
                &
                \sum_{g \in G}\psi(a_g)g
                \arrow[d, maps to]\\
                \sum_{g \in G}a_g\phi(g)
                \arrow[r, maps to]
                &
                \sum_{g \in G}\psi(a_g)\phi(g)
            \end{tikzcd}
        \end{center}
        However, we can follow the elements as in the diagram on the right, which shows 
        us that the diagram commutes. Hence we see that $\psi^*$ is a natural transformation 
        between functors $R[-] \to S[-]$. Overall, this establishes that we do in fact have 
        a functor 
        \[
            F: \textbf{Ring} \to \textbf{Ring}^{\textbf{Grp}}   
        \]
        which we wouldn't be able to describe without otherwise introducing 
        the notion of a functor category.
    \end{example}

    \begin{example}
        Let $M$ be a monoid category (one object) and
        consider the functor category $\textbf{Set}^M$. The objects of
        \textbf{Set}$^M$ are functors $F: M \to \textbf{Set}$, each
        of which have the following data: 
        \[
            F(f) : F(M) \to F(M)
        \]
        where $f: M \to M$ is an morphism in $M$. Now if we interpret
        $\circ$ as the binary relation equipped on $M$, we see that
        for any $g : M \to M$,
        \[
            F(g \circ f) = F(g) \circ F(f)
        \]
        by functorial properties. Hence, each functor $F$ maps $M$ to
        a set $X$ which induces the operation of $M$ on $X$. Therefore
        the objects of \textbf{Set}$^M$ are other monoids $X$ in
        \textbf{Set} equipped with the same operation as $M$ and as
        well as the morphisms between such monoids. 
    \end{example}

    % \begin{example}
    %     Consider \textbf{Grp}$^{M}$, where $M$ is a monoid.
    %     This is similar to the previous example, so we may say the the
    %     objects of \textbf{Grp}$^{M}$ are the mappings from monoids to
    %     groups, equipped with the operation of $M$, such that the
    %     mappings preserve morphism mappings in the preimage $M$.
    % \end{example}
    
    

    \newpage
    \section{Vertical, Horizontal Composition; Interchange Laws}
    In the previous section, we considered the idea of forming a
    composition of natural transformations, and we verified that this
    formed a valid natural transformation. 
    That is, if we have three functors $F, G, H : \cc \to \dd$ between
    two categories $\cc$ and $\dd$, and if $\sigma: F \to G$ and
    $\tau:G \to H$ are natural transformations, then we can form the
    natural transformation 
    \[
        (\tau \circ \sigma) : F \to H.
    \]
    We call such a type of composition as
    vertical compositions of natural transformations, 
    since the idea can be captured in the following diagram. 

    \begin{minipage}{0.8\textwidth}
    \vspace{0.5cm}
    
    \begin{center}
        \begin{tikzpicture}
            \draw[->] (1.1,0.65) -- (1.1, 0.15) node at (1.4, 0.4) {$\sigma$};
            \draw[->] (1.1,0) -- (1.1, -0.45) node at (1.4, -0.2) {$\tau$};
            node at (0,0) {
                \begin{tikzcd}
                \cc  \arrow[r]
                \arrow[r, shift left=6] 
                \arrow[r, shift right=6]
                & 
                \dd
            \end{tikzcd} };
        \end{tikzpicture}
    \end{center}
    \end{minipage}
    \vspace{0.3cm}

    We can also perform a different, but
    similar type of composition between natural transformations.
    Suppose $F, G: \bb \to \cc$ and $F', G': \cc \to \dd$ are functors
    between categories $\bb, \cc$, and $\dd$. Furthermore, suppose we
    have natural transformations $\eta: F \to G$ and $\eta': F' \to
    G'$. Then we have diagram such as the following. 

    \begin{minipage}{0.7\textwidth}
    \vspace{0.5cm}

    \begin{center}
    \begin{tikzpicture}
        \draw[->] (1.1,0.35) -- (1.1, -0.25) node at (1.4, 0.1) {$\eta$};
        \draw[->] (2.8,0.35) -- (2.8, -0.25) node at (3.1, 0.1) {$\eta'$};
        node at (0,0) {
            \begin{tikzcd}
            \bb 
            \arrow[r, shift left = 4, "F"]
            \arrow[r, shift right= 4, swap, "G"]
            & 
            \cc
            \arrow[r, shift left=4, "F'"]
            \arrow[r, shift right=4, swap,"G'"]
            &
            \dd
            \end{tikzcd}};
    \end{tikzpicture}
    \end{center}
    \hspace{5cm}   
    \end{minipage}
    \vspace{-0.5cm} 

    Now let $B$ be an object of $\bb$. There are two ways we can
    transfer this object to an object of $\cc$; namely, via mappings
    of $F$ and $G$. Thus $F(B)$ and $G(B)$ are two objects of $\cc$.
    Since $\eta: F \to G$ is a natural transformation between these
    objects, we see that there's a way of mapping between these two
    elements in $\cc$:
    \[
        \eta(B): F(B) \to G(B).
    \]     
    Hence, we have two objects in $\cc$ and a morphism in between
    them. Hence, we know that the natural transformation $\eta': F'
    \to G'$ implies the following diagram commutes. 
    \begin{center}
        \begin{tikzcd}[row sep = 1.2cm]
            \textcolor{Red}{F(B)} \arrow[d, "\textcolor{Green}{\eta(B)}"]\\
            \textcolor{Blue}{G(B)}
        \end{tikzcd}
        \hspace{2cm}
        \begin{tikzcd}[row sep = 1.2cm, column sep = 1.4cm]
            F'\circ \textcolor{Red}{F(B)} \arrow[r, "{\eta'\textcolor{Red}{F(B)}}"] \arrow[d,swap,
            "{F'\circ \textcolor{Green}{\eta(B)}}"] 
            &
            G'\circ \textcolor{Red}{F(B)} \arrow[d, "{G' \circ \textcolor{Green}{\eta(B)}}"]\\
            F'\circ \textcolor{Blue}{G(B)} \arrow[r, "{\eta' \textcolor{Blue}{G(B)}}"] 
            &
            G'\circ \textcolor{Blue}{G(B)}
        \end{tikzcd}
    \end{center}
    Note that in the last diagram, all of the objects and morphisms
    between them exist in $\dd$. The easiest way to see why this
    diagram commutes is to go back directly to the definition of a
    natural transformation; namely, the pair of objects along with
    their morphism on the left imply the commutativity of the diagram
    on the right. 

    This can be done in general for categories $\bb, \cc$, and $\dd$
    which have functors $F, G:\bb \to \cc$ and $F', G': \cc \to \dd$
    associated with natural transformations $\eta: F \to G$ and
    $\eta': F' \to G'$. Furthermore, it holds for all $B \in \bb$.

    Note further that this diagram is similar to a diagram which represents a natural
    transformation; but between which functors? If we look closely, we
    see that it is
    between $F\circ F'$ and $G \circ G'$. 

    This leads us to make the following formulaic definition:
    For natural transformations $\eta: F \to G$ and $\eta': F' \to
    G'$ such that $F, G: \bb \to \cc$ and $F',G' : \cc \to \dd$, then
    for $B \in \bb$ we define their "horizontal" composition as the
    diagonal of the above diagram; that is,
    \[
        (\eta \circ \eta')B = G'(\eta(B)) \circ \eta' F(B) = \eta'(G(B))\circ F'(\eta(B)).
    \]

    The above diagram doesn't quite show that $\eta \circ \eta': F'
    \circ F
    \to G \circ G' $ is a natural transformation. In order to do this, we
    need to start from two objects in $\bb$ and consider a morphism
    between them. 

    \begin{proposition}
        The function $\eta \circ \eta':F\circ F'  \to G \circ G'$ is a natural
        transformation between the functors $F'\circ F, G' \circ G:
        \bb \to \dd$. 
    \end{proposition}

    \begin{prf}
        To show this, we consider a morphism $f: B \to B'$ between two
        objects $B$ and $B'$ in $\bb$. We then claim that the
        following diagram is commutative:
        \begin{center}
            \begin{tikzcd}[row sep = 1.4cm]
                B \arrow[d, "f"]\\
                B'
            \end{tikzcd}
            \hspace{1cm}
            \begin{tikzcd}[row sep = 1.4cm, column sep = 1.4cm]
                F' \circ F(B) 
                \arrow[r, "F'\circ \eta(B)"] 
                \arrow[d, swap, "F'\circ F(f)"]
                &
                F' \circ G(B) 
                \arrow[r, "\eta' \circ G(B)"]
                \arrow[d, "F'\circ G(f)"]
                &
                G' \circ G(B) 
                \arrow[d, "G' \circ G(f)"]\\
                F'\circ F(B')
                \arrow[r,swap, "F' \circ \eta(B')"]
                &
                F' \circ G(B')
                \arrow[r, swap, "\eta' \circ \eta(B')"]
                &
                G'\circ G(B')
            \end{tikzcd}
        \end{center}
        First, observe that the left square is commutative due to the
        fact that $\eta$ is a natural transformation from $F$ to $G$.
        Therefore, it produces a commutative square diagram, and we obtain
        the above left square diagram by applying $F'$ to the
        commutative diagram produced by $\eta: F \to G$.

        The right square in the diagram is obtained by the fact that
        $\eta'$ is a natural transformation between functors $F'$ and
        $G'$. Hence the diagram is commutative, and it acts on the
        objects $G(B)$ and in $\cc$. Therefore, we see that $\eta \circ
        \eta'$ is a natural transformation. 
    \end{prf}

    \textcolor{NavyBlue}{Thus we see that we have "horizontal" and "vertical" notions of
    composing natural transformations. Let us denote "horizontal"
    transformations as $\circ$ and "vertical" transformations as
    $\cdot$ between natural transformations.} 

    It is also notationally convenient to denote functor and natural
    transformation compositions as 
    \[
        F' \circ \tau : F' \circ F \to F' \circ T  \quad \eta' \circ G: F' \circ G \to G' \circ G
    \]
    which are two additional natural transformations. \textcolor{purple}{(Remember we
    showed that the left square in the commutative diagram of the
    previous proof commuted by observing that it was obtained by the
    commutative diagram produced by the natural transformation $\eta$
    and composing it with $F'$? What we really showed is that $F' \circ
    \eta$ is a natural transformation, since this natural
    transformation described that square. Similarly, $\eta' \circ G$ is
    the natural transformation which represents the right square of
    the commutative diagram in the previous proof.)}

    With the above notation, we can then write that 
    \[
        \eta' \circ \eta = (G' \circ \eta) \cdot (\eta' \circ F) = (\eta' \circ G) \cdot (F' \circ \eta).       
    \]
    This idea of ours can be extended to a more general situation.
    Suppose we have instead three categories $\bb, \cc$, and $\dd$ and
    where $F, G, H: \bb \to \cc$ and $F, G, H: \cc \to \dd$ are
    functors associated with natural transformations $\eta: F \to G,
    \sigma : G \to H$, and $\eta': F' \to G', \sigma': G' \to H'$. The
    following diagram may be more helpful than words:
    \\

    \begin{minipage}{0.65\textwidth}
    \begin{center}
        \begin{tikzpicture}
            \draw[->] (1.1,0.65) -- (1.1, 0.15) node at (1.4, 0.4) {$\eta$};
            \draw[->] (1.1,0) -- (1.1, -0.45) node at (1.4, -0.2)
            {$\sigma$};
            \draw[->] (2.7,0.65) -- (2.7, 0.15) node at (3, 0.4) {$\eta'$};
            \draw[->] (2.7,0) -- (2.7, -0.45) node at (3, -0.2) {$\sigma'$};
            node at (0,0) {
                \begin{tikzcd}
                \bb 
                \arrow[r]
                \arrow[r, shift left = 6, "F"]
                \arrow[r, shift right= 6, swap, "H"]
                & 
                \cc
                \arrow[r]
                \arrow[r, shift left=6, "F'"]
                \arrow[r, shift right=6, swap,"H'"]
                &
                \dd
                \end{tikzcd}};
        \end{tikzpicture}
        \end{center}
        \hspace{7cm}   
    \end{minipage}
    \vspace{-0.5cm} 

    Note we've omitted the label of $G$ and $G'$ on the middle
    horizontal arrows since they don't exactly fit in there when we
    include the labels for the natural transformations. 

    Now suppose we have an object $B$ in $\bb$. Then we can create
    three objects $F(B), G(B)$ and $H(B)$ in $\cc$, and we may
    interchange between these objects via the given natural
    transformations. Specifically, $\eta(B) : F(B) \to G(B)$ and
    $\sigma(B): G(B) \to H(B)$. However, we also know that $\eta',
    \sigma'$ are natural transformations between $\cc$ and $\dd$, and
    hence imply the following commutative diagram. 

    \begin{center}
        \begin{tikzcd}[row sep = 1.4cm]
            \textcolor{Red}{F(B)}
            \arrow[d, "\eta(B)"]\\
            \textcolor{Green}{G(B)}
            \arrow[d, "\sigma(B)"]\\
            \textcolor{Blue}{H(B)}
        \end{tikzcd}
        \hspace{1cm}
        \begin{tikzcd}[row sep = 1.4cm, column sep = 1.4cm]
            F' \circ \textcolor{Red}{F(B)}
            \arrow[r, "\eta'\textcolor{Red}{F(B)}"]
            \arrow[d, swap, "F' \circ \eta(B)"]
            &
            G' \circ \textcolor{Red}{F(B)}
            \arrow[r, "\sigma'\textcolor{Red}{F(B)}"]
            \arrow[d, "G' \circ \eta(B)"]
            &
            H' \circ \textcolor{Red}{F(B)}
            \arrow[d, "H' \circ \eta(B)"]
            \\
            F' \circ \textcolor{Green}{G(B)}
            \arrow[r, "\eta'\textcolor{Green}{G(B)}"]
            \arrow[d, swap, "F' \circ \sigma(B)"]
            &
            G' \circ \textcolor{Green}{G(B)}
            \arrow[r, "\sigma'\textcolor{Green}{G(B)}"]
            \arrow[d, "G' \circ \sigma(B)"]
            &
            H' \circ \textcolor{Green}{G(B)}
            \arrow[d, "H' \circ \sigma(B)"]  
            \\
            F' \circ \textcolor{Blue}{H(B)}
            \arrow[r, swap, "\eta'\textcolor{Blue}{H(B)}"]
            &
            G' \circ \textcolor{Blue}{H(B)}
            \arrow[r, swap, "\sigma'\textcolor{Blue}{H(B)}"]
            &
            H' \circ \textcolor{Blue}{H(B)}
        \end{tikzcd}
    \end{center}
    Suppose we start at the upper left corner and want to achieve the
    value at the bottom right. There are two ways we can do this: We
    can travel within the interior of the diagram, or we can travel on
    the outside of the diagram.
    
    In traveling on the interior of the diagram, 
    note that the composition of the arrows of the upper left square
    is $\eta' \circ \eta$. In addition, composition of the arrows of
    the bottom right square is $\sigma' \circ \sigma$. 

    In traveling on the exterior of the diagram note that the
    composition of the top row is  $\eta' \cdot \sigma'$  and
    composition of the right most vertical arrows is $\eta \cdot
    \sigma$. Since both paths achieve the same value, we see that 
    \[
        (\eta' \cdot \sigma') \circ (\eta \cdot \sigma)
        =  (\eta' \circ \eta) \cdot (\sigma' \circ \sigma)
    \]
    which is known as the \textbf{Interchange Law}.

    This leads us to make the following definition.

    \begin{definition}
        We define a \textbf{double category} to be a set of arrows
        which obey two different forms of composition, generally
        denoted as $\circ$ and $\cdot$, which together satisfy the
        interchange law. 

        Furthermore, a \textbf{2-category} is a double category in
        which $\cdot$ and $\circ$ have the same exact identity arrows.
        
    \end{definition}


    

    \newpage
    \section{Slice and Comma Categories.}
    In this section we introduce comma categories, which serve as 
    a very useful categorical construction. The reason why it is so useful is 
    because the notion of a comma category has 
    the potential to simplify an otherwise complicated discussion. 
    As they can be constructed in any category, and 
    because they contain a large amount of useful data,
    they are frequently used as an intermediate step in 
    more complex categorical constructions. Thus, while the concept 
    is ``simple,'' they nevertheless appear in all kinds of complicated
    discussions in category theory.

    \begin{definition}
        Let $\cc$ be a category and suppose $\textcolor{Purple}{A}$ is an object of $\cc$.
        We define the \textbf{slice category (with $\textcolor{Purple}{A}$ over $\cc$)}, 
        denoted $(\textcolor{Purple}{A} \downarrow\cc)$, as the category 

        \begin{description}
            \item[Objects.] All pairs $(C, f: \textcolor{Purple}{A} \to C)$ 
            for all $C \in  \cc$  and morphims $f:\textcolor{Purple}{A} \to C$.
            In other words, the objects are all morphisms in $\cc$ which \emph{originate}
            at $\textcolor{Purple}{A}$. 

            \item[Morphisms.]
            For two objects $(C, f:\textcolor{Purple}{A}\to C)$ and $(C', f': \textcolor{Purple}{A} \to C')$, 
            we define 
            \[
                \textcolor{NavyBlue}{h} : (C, f) \to (C', f')
            \]
            as a morphism between the objects, where $h: C \to C'$ is a morphism in 
            our category such that $f' = h \circ f$. Alternatively we can describe 
            the homset more directly:
            \[
                \hom_{(A\downarrow \cc)}\Big( (f, C), (f', C')\Big)
                =
                \{ h: C \to C' \in \cc \mid f' = h \circ f \}.
            \]
        \end{description}
    \end{definition}
    At this point you may be a bit overloaded with notation if this is the first 
    time you've seen this before. You need to figure out how this is a category (what's the identity? composition?)
    and ultimately why you should care about this category.
    To aid your understanding, a picture might help.

    We can represent the objects and morphisms of the
    category $(\textcolor{Purple}{A} \downarrow \cc)$ in a visual manner. 
    \begin{center}
        Objects $(C, f)$:
        \begin{tikzcd}[row sep = 1.4cm]
            \textcolor{Purple}{A} \arrow[d, "f"]\\
            C
        \end{tikzcd}
        \hspace{1cm}
        \parbox{6cm}{Morphisms $\textcolor{NavyBlue}{h}: (C, f) \to (C', f')$
        are given by $\textcolor{NavyBlue}{h}: C \to C'$ such that}
        \begin{tikzcd}[row sep = 1.4cm, column sep = 0.5cm]
            & \textcolor{Purple}{A} \arrow[dl, swap, "f"] \arrow[dr, "f'"] & \\
            C \arrow[rr, swap, NavyBlue, "h"] & & C'
        \end{tikzcd}
    \end{center}
    Now, how does composition work?
    Composition of two composable morphisms $h : (f, C) \to
    (f', C')$ and $h' : (f', C') \to (f'', C'')$ is given by 
    $h' \circ h : (f, C) \to (f'', C'')$ since clearly 
    \[
        f'' = h' \circ f' \hspace{0.5cm}\text{and}\hspace{0.5cm} f' = h \circ f \implies f'' = h' \circ (h \circ f) = (h' \circ h) \circ f.
    \]  
    We can visually justify composition as well. If we have two commutative diagrams 
    as on the left, we can just squish them together to get the final commutative diagram on the right. 
    \begin{center}
        \begin{tikzcd}[row sep = 1.4cm, column sep = 0.5cm]
            & \textcolor{Purple}{A} 
            \arrow[dl, swap, "f"] 
            \arrow[dr, "f'"] & \\
            C 
            \arrow[rr, NavyBlue, swap, "h"] & & C'
        \end{tikzcd}
        and 
        \begin{tikzcd}[row sep = 1.4cm, column sep = 0.5cm]
            & \textcolor{Purple}{A} \arrow[dl, swap, "f'"] \arrow[dr, "f''"] & \\
            C' \arrow[rr, NavyBlue, swap, "h'"] & & C''
        \end{tikzcd}
        implies
        \begin{tikzcd}[row sep = 1.4cm, column sep = 0.5cm]
            & \textcolor{Purple}{A} 
            \arrow[dl, swap, "f"] 
            \arrow[dr, "f''"] & \\
            C \arrow[rr, swap, NavyBlue, "h' \circ h"] & & C''
        \end{tikzcd}
    \end{center}

    Hence, we see that $h' \circ h :(f, C) \to (f'', C'')$ is
    defined whenever $h'$ and $h$ are composable as morphisms of
    $\cc$.

    One use of comma categories is to capture and generalize 
    the notion of a pointed category.
    Such pointed categories include
    the category of pointed sets $\textbf{Set}^*$ or the category of pointed 
    topological spaces $\textbf{Top}^*$, etc. 

    We've seen, in particular on the discussion of functors, the necessity for 
    pointed categories. For example, we cannot discuss ``the'' fundamental 
    group $\pi_1(X)$ of a topological space $X$ (unless $X$ is path connected, but still only 
    up to isomorphism). To discuss
    a fundamental group in a topological space $X$, one needs to select a base point $x_0$. 
    As we saw in Example \ref{example:fundamental_group}, 
    $\pi_1$ is not a functor $\textbf{Top} \to \textbf{Grp}$, but is rather a 
    functor 
    \[
        \pi_1: \textbf{Top}^* \to \textbf{Grp}
    \]
    where $\textbf{Top}^*$, which consists of pairs $(X, x_0)$ with $x_0 \in X$,
    is the category of pointed topological spaces.

    Similarly, it makes no sense to talk about ``the'' tangent plane of 
    a smooth manifold. Such an association requires the selection of a point $p \in X$
    to calculate $T_p(M)$. 
    So, as we saw in Example \ref{example:manifold_tangent_plane}, this process is not 
    a functor from $\textbf{DMan}$ to $\textbf{Vect}$, but is rather a functor 
    \[
        T: \textbf{DMan}^* \to \textbf{Vect}
    \]
    where $\textbf{DMan}^*$, which consists of pairs $(M, p)$ with $p \in M$,
    is the category of pointed smooth manifolds. This now motivates the next two examples.

    \begin{example} 
        Consider the category $\textbf{Top}^*$ where 
        \begin{description}
            \item[Objects.] The objects are pairs $(X, x_0)$ with $X$ a topological space 
            and $x_0 \in X$.
            \item[Morphisms.] A morphism $f: (X, x_0) \to (Y, y_0)$ is any continuous function 
            $f: X \to Y$ such that $y_0 = f(x_0)$.  
        \end{description}
        Recall that the one point set $\{\bullet\}$ is trivially a topological space. 
        Then we can form the category $(\{\bullet\}\downarrow \textbf{Top})$. The claim now is that 
        \[
            (\{\bullet\}\downarrow \textbf{Top}) 
            \cong 
            \textbf{Top}^*.
        \]
        Why? Well, an object of $(\{\bullet\}\downarrow \textbf{Top})$ is simply 
        a pair $(X, f: \{\bullet\} \to X)$. Observe that 
        \[
            f(\bullet) = x_0 \in X,
        \]
        for some $x_0 \in  X$.
        So, the pair $(X, f: \{\bullet\} \to X)$ is logically equivalent to a 
        pair $(X, x_0)$ with $x_0 \in X$. \textcolor{NavyBlue}{That is, a continuous function
        from the one point set into a topological space $X$ is equivalent to simply selecting 
        a single point $x_0 \in X$}.  Hence, on objects it is clear why we have an isomorphism.

        Now, a morphism in this comma category will be of the form 
        $p: (X, f_1: \{\bullet\} \to X) \to (Y, f_1: \{\bullet\} \to Y)$. 
        Specifically, it is a continuous function $p: X \to Y$ such that the diagram below commutes. 
        \begin{center}
            \begin{tikzcd}[column sep = 0.5cm, row sep = 1.4cm]
                &\{\bullet \}
                \arrow[dl, swap, "f_1"]
                \arrow[dr, "f_2"]
                &\\
                X \arrow[rr, swap, "p"] 
                &
                &
                Y 
            \end{tikzcd}
        \end{center}
        In other words, if $f_1(\bullet) = x_0$ and $f_2(\bullet) = y_0$, 
        it is a continuous function $p: X \to Y$ such that $f(x_0) = y_0$.
        This is exactly a morphism in $\textbf{Top}^*$! We clearly have a bijection 
        as claimed.
    \end{example}

    The above example generalizes to many pointed categories, some of which are 
    \begin{itemize}
        \item $\textbf{DMan}^* \cong (\bullet \downarrow \textbf{DMan})$
        \item $\textbf{Set}^* \cong (\bullet \downarrow \textbf{Set})$
        \item $\textbf{Grp}^* \cong (\bullet \downarrow \textbf{Grp})$
    \end{itemize}

    We now briefly comment for any slice category $(A \downarrow \cc)$ 
    built from a category $\cc$, we can construct a ``projection'' functor 
    \[
        P: (A \downarrow \cc) \to \cc  
    \]
    where on objects $P(C, f: A \to C) = C$ and on morphisms
    $P(h:(C, f) \to (C', f'))= h: C  \to C'$. Clearly, this functor is 
    faithful, but it is generally not full. Such a projection functor  
    is used in technical constructions involving slice categories as 
    it has nice properties; we will make use of it later when we discuss 
    limits. 

    Next,we introduce how we can also describe the category of an objects \textit{under}
        another category. 
        \begin{definition}
            Let $\cc$ be a category, and $\textcolor{Purple}{B}$ an object of $\cc$. Then
            we define the \textbf{category $\textcolor{Purple}{B}$ under $\cc$}, denoted as
            $(\cc \downarrow \textcolor{Purple}{B})$ as follows.
            \begin{description}
                \item[Objects.] All 
                pairs $(C, f)$ where $f: C \to \textcolor{Purple}{B}$
                is a morphism in $\cc$. That is, the objects are morphisms \emph{ending} 
                at $\textcolor{Purple}{B}$. 

                \item[Morphisms.] For two objects $(C, f: C  \to \textcolor{Purple}{B})$
                and $(C', f': C'  \to \textcolor{Purple}{B})$, we define
                \[ 
                    \textcolor{NavyBlue}{h}: (C, f) \to (C', f')
                \] 
                to be a morphism between the objects to correspond  to  a morphism
                $\textcolor{NavyBlue}{h}:C \to C'$ in $\cc$ 
                such that $f = f' \circ h$.  
            \end{description}
        \end{definition}
        Composition of functions $\textcolor{NavyBlue}{h}: (f, C) \to (f', C')$ and
        $\textcolor{NavyBlue}{h'}: (f', C') \to (f'', C'')$ exists whenever 
        $\textcolor{NavyBlue}{h'} \circ \textcolor{NavyBlue}{h}$ is
        defined as morphisms in $\cc$. Again, we can represent the
        elements of the category in a visual manner 
        \begin{center}
            Objects $(C, f)$:
            \begin{tikzcd}[row sep = 1.4cm]
                C \arrow[d, "f"]\\
                \textcolor{Purple}{B}
            \end{tikzcd}
            \hspace{1cm}
            Morphisms $\textcolor{NavyBlue}{h}: (f, C) \to (f', C')$
            \begin{tikzcd}[row sep = 1.4cm, column sep = 0.5cm]
                C \arrow[rr, NavyBlue, "h"] \arrow[dr, "f",swap]& & C'\arrow[dl, "f'"]\\
                & \textcolor{Purple}{B} & 
            \end{tikzcd}
        \end{center}

        The following is a nice example that isn't traditionally seen as
        an example of a functor. 

        \begin{example}
            Let $(G, \cdot)$ and $(H, \cdot)$ be two groups, and consider a group 
            homomorphism $\phi: (G, \cdot) \to (H, \cdot)$. Abstractly, this 
            is an element of the comma category $(\textbf{Grp} \downarrow H)$.
            
            Now for for every group 
            homomorphism, we may calculate 
            the kernal of $\ker(\phi) = \{g \in G \mid \phi(g) = 0\}$. This 
            is always a subgroup of $G$. What is interesting is that, 
            from the perspective of slice categories, this 
            process is functorial:
            \[
                \ker(-): (\textbf{Grp}\downarrow H) \to \textbf{Grp}.  
            \]
            To see this, we have to understand what happens on the morphisms.
            So, suppose we have two objects $(G, \phi: G \to H)$ and $(K, \psi: K \to H)$ 
            of $(\textbf{Grp}\downarrow H)$ and a morphism $h: G \to K$ between the objects. 
            \begin{center}
                \begin{tikzcd}[row sep = 1.4cm, column sep = 0.5cm]
                    G \arrow[rr, "h"] 
                    \arrow[dr, "\phi",swap]
                    & 
                    & 
                    K
                    \arrow[dl, "\psi"]\\
                    & H & 
                \end{tikzcd}
            \end{center}
            Then we can define $\ker(h): \ker(\phi) \to \ker(\psi)$, the image 
            of $h$ under the functor, to 
            be the restriction $h|_{\text{ker}(\phi)}: \ker(\phi) \to \ker(\psi)$. 
            This is a bonafied group homomorphism: by the commutativity of the above 
            triangle, if $g \in G$ then $\phi(g) = \psi(h(g))$. Hence, if $\phi(g) = 0$, 
            i.e., $g \in \ker(\phi)$, then $\psi(h(g)) = 0$, i.e., $h(g) \in \ker(\psi)$. 
            So we see that our proposed function makes sense. 

            What this means is that the commutativity of the above triangle forces 
            a natural relationship between the kernels of $\phi$ and $\psi$; not only 
            as a function of sets, but as a group homomorphism.
            Therefore, the kernel of a group homomorphism is actually a 
            functor from a slice category. 
        \end{example}



        \begin{example}
            In geometry and topology, one often meets the need to define a $(-)$-bundle. By $(-)$
            we mean vector, group, etc. That is, we often want topological spaces to parameterize 
            a family of vector spaces or groups in a coherent way. 
            \begin{center}
                \includegraphics[width = 0.9\textwidth]{mobius_sphere.png}
            \end{center}
 
            For example, on the above left we can map the Möbius strip onto $S_1$ in such a way that the 
            inverse image of each $x \in S_1$ is homeomorphic to the interval $[0, 1]$. Hence, 
            each point of $x \in S_1$ carries the information of a topological space, specifically one of $[0, 1]$.
    
            On the right, we can recall that $S^2$ is a differentiable manifold, and so each point $p$
            has a tangent plane $T_p(S^2)$, which is a vector space. Hence every point on $S^2$, or more generally 
            for any differentiable manifold, carries the information of a vector space. 

            In general, for a topological space $X$, 
            we define a \textbf{bundle} over $X$ to be a continuous map $p: E \to X$ with $E$ being some topological 
            space of interest. If $p: E \to X$ an $p': E' \to X$ are two bundles, a 
            \textbf{morphism of bundles} $q :p  \to p'$ is given by a continuous map $q: E \to E'$ such 
            that 
            \[
                p = p' \circ q.
            \]
            Hence we see that a bundle over a topological space $X$ is an element of the 
            comma category $\textbf{Top}/X$, and a morphism of bundles is a morphism in the comma category. 
            We therefore see that $\textbf{Top}/X$ can be interpreted as the \textbf{category of 
            bundles of $X$}.

            One particular case of interest concerns \textbf{vector bundles.}
            Let $E, X$ be topological spaces. Recall that a vector bundle 
            consists of a continuous map $\pi: E \to X$ such that
            \begin{itemize}
                \item[1.] $\pi^{-1}(x)$ is a finite-dimensional 
                vector space over some field $k$
                \item[2.] For each $p \in X$, there is an open neighborhood $U_{\alpha}$ 
                and a homeomorphism 
                \[
                    \phi_{\alpha}: U_{\alpha} \times \rr^{n} \isomarrow \pi^{-1}(U_{\alpha})
                \] 
                with $n$ some natural number. We also require that $\pi \circ \phi_{\alpha} = 1_{U_\alpha}$.
            \end{itemize}
            As we might expect, a \textbf{morphism of vector bundles} between $\pi_1: E \to X$ and $\pi_2: E' \to X$
            is given by a continuous map $q: E \to E'$ such that for each $x \in X$, 
            $q\big|_{\pi_1^{-1}(x)}: \pi_1^{-1}(x) \to \pi_2^{-1}(x)$ is linear map between vector spaces. 

            To realize this in real mathematics, we can take the classic example of the 
            \textbf{tangent bundle} on a 
            smooth manifold $M$ (if you've seen this before, hopefully it is now clear why the word "bundle" 
            is here). In differential geometry this is defined as the set 
            \[
                TM = \{(p, v) \mid p \in M \text{ and } v \in T_p(M)\}
            \]
            where we recall that $T_p(M)$ is the tangent (vector) space at 
            a point $p \in M$. 
            Since $M$ is a smooth manifold there is a differentiable 
            structure $(U_\alpha, \bm{x}_\alpha: U_\alpha \to M)$
            which allow us to define a map 
            \begin{gather*}
                \bm{y}_\alpha: U_\alpha \times \rr^n \to TM \\
                ((x_1, \dots, x_n), (u_1, \dots, u_n)) \mapsto 
                \left(\bm{x_{\alpha}}(x_1, \dots, x_n), \sum_{i=1}^{n}u_i\frac{\partial}{\partial x_i} \right).
            \end{gather*}
            This actually provides a differentiable structure on $TM$, demonstrating it 
            too is a smooth manifold (see Do Carmo). Hence we see that $TM$ is in fact a topological space. 
            We then see that the mapping
            $\pi: TM \to M$ where 
            \[
                \pi(p, v) = p \text{ and } \pi^{-1}(x) = T_x(M).
            \]
            is a continuous mapping.
            Hence we've satisfied both (1.) and (2.) in the the definition of a vector bundle. 
            The other properties can be easily verified so that this provides a nice example of a 
            vector bundle.
        \end{example}

        We can also formulate categories of objects \textit{under} and
        \textit{over} functors. 

        \begin{definition}
            Let $\cc$ be a category, $C$ an object of $\cc$ and $F: 
            \bb \to \cc$ a functor. Then we define the \textbf{category $C$
            over the functor $F$}, denoted as $(C \downarrow F)$,
            as follows. 
            \begin{description}
                \item[Objects.] All
                pairs $(f, B)$ where $B \in \text{Obj}(\bb)$ such
                that 
                \[
                    f : C \to F(B)
                \]
                where $f$ is a morphism in $\cc$. 

                \item[Morphisms.] The morphisms $h: (f, B) \to (f', B')$ of
                $(C \downarrow F)$ are defined whenever there exists a
                $h:B \to B'$ in $\bb$ such that $f' = F(h) \circ f$. 
            \end{description}
        \end{definition}
        Representing this visually, we have that 
        \begin{center}
            Objects $(f, B)$:
            \begin{tikzcd}[row sep = 1.4cm]
                C \arrow[d, "f"]\\
                F(B)
            \end{tikzcd}
            \hspace{1cm}
            Morphisms $h: (f, B) \to (f', B')$
            \begin{tikzcd}[row sep = 1.4cm, column sep = 0.5cm]
                & C \arrow[dl, swap, "f"] \arrow[dr, "f'"] & \\
                F(B) \arrow[rr, swap, "F(h)"] & & F(B')
            \end{tikzcd}
        \end{center}
        Composition of the morphisms in $(C \downarrow F)$ simply
        requires composition of morphisms in $\bb$. 

        One can easily construct the \textbf{category $C$ under the
        functor $F$}, $(F \downarrow C)$, in a completely analogous
        manner as before. But we'll move onto finally defining the
        concept of the comma category. 

        \begin{definition}
            Let $\bb, \cc, \dd$ be categories and let $F: \bb \to \dd$
            and $G:\cc \to \dd$ functors. That is, 
            \begin{center}
                \begin{tikzcd}
                    \bb \arrow[r, "F"] & \dd & \cc \arrow[l, "G", swap].
                \end{tikzcd}
            \end{center}
            Then we define the \textbf{comma category} $(F \downarrow
            G)$ as follows. 
            \begin{description}
                \item[Objects.] All pairs $(B, C, f)$
                where $B, C$ are objects of $\bb, \cc$, respectively,
                such that  
                \[
                    f : F(B) \to G(C)
                \]
                where $f$ is a morphism in $\dd$. 

                \item[Morphisms.] All pairs $(h, k) : (B,
                C, f) \to (B', C', f')$ where $h: B \to B'$ and $k: C
                \to C'$ are morphisms in $\bb, \cc$, respectively, such that \
                \[
                    f' \circ F(h) = G(k) \circ f.
                \]
            \end{description}
        \end{definition}
        As usual, we can represent this visually via diagrams:
        \begin{center}
            Objects $(B, C, f)$:
            \begin{tikzcd}[row sep = 1.4cm]
                F(B) \arrow[d, "f"]\\
                G(C)
            \end{tikzcd}
            \hspace{1cm} Morphisms $(h,k)$
            \begin{tikzcd}[row sep = 1.4cm, column sep = 1.4cm]
                F(B) \arrow[d, swap, "f"] \arrow[r, "F(h)"] & F(B' )\arrow[d, "f'"] & \\
                G(C) \arrow[r, swap, "G(k)"] & G(C')
            \end{tikzcd}
        \end{center}
        where in the above picture we have that $(h, k):(B, C, f) \to
        (B', C', f')$. Since functors naturally respect composition of
        functions, one can easily define composition of morphism $(h,
        k)$ and $(h', k')$ as $(h \circ h', k \circ k')$ whenever $h
        \circ h'$ and $ k\circ k'$ are defined as morphisms in $\bb$ and
        $\cc$, respectively. 
        \vspace{0.5cm}

        {\large \textbf{Exercises}
        \vspace{0.5cm}}
        \begin{itemize}
            \item[\textbf{1.}]
            Let $\cc$ be a category with initial and terminal objects $I$ and $T$. 
            \begin{itemize}
                \item[\emph{i}.]
                Show that $(\cc \downarrow T) \cong \cc$.
                \item[\emph{i}.]
                Also show that $(I \downarrow \cc) \cong \cc$. 
            \end{itemize} 

            \item[\textbf{2.}]
            Consider again a group homomorphism $\phi: G \to H$, but this time 
            consider the image $\im(\phi) = \{\phi(g) \mid g \in G\}$. 
            Show that this defines a functor 
            \[
                \im(-): (G \downarrow \textbf{Grp}) \to \textbf{Grp}
            \] 
            where on morphisms, a morphism 
            \[
                h: (H, \phi: G \to H) \to (K, \phi: G \to K)                
            \]
            is mapped to the restriction 
            $h|_{\text{Im}(\phi)}: \im(\phi) \to \im(\psi)$. 
        
            In some sense, this is the ``opposite'' construction of the 
            kernel functor we introduced. Instead of taking the kernel of a group homomorphism, 
            we can take its image.

            \item[\textbf{3.}]
            Here we prove that the processes of imposing the \textbf{induced topology} 
            and the \textbf{coinduced topology} are functorial. Moreover, the correct language 
            to describe this is via slice categories. 
            \begin{itemize}
                \item[\emph{i}.]
                Let $X$ be a set and $(Y, \tau)$ a topological space. 
                Denote $U: \textbf{Top} \to \textbf{Set}$ 
                to be the forgetful functor. Given any function $f: X \to U(Y)$,
                we can use the topology on $Y$ to impose a topology $\tau_X$ on $X$:
                \[
                    \tau_X = \{U \subset X \mid f(U) \text{ is open in }Y\}.
                \]
                This is called the \textbf{induced topology on} $X$.  
                So, we see that (by abuse of notation) the function $f: X \to U(Y)$ 
                is now a continuous function $f: (X, \tau_X) \to (Y, \tau_Y)$. 
    
                Prove that this process forms a functor $\text{Ind}: (\textbf{Top}\downarrow U(Y)) \to (\textbf{Top}\downarrow Y)$. 

                \item[\emph{ii}.] 
                This time, let $(X, \tau)$ be a topological space, $Y$ a set, 
                and consider a function $f: U(X) \to Y$. We can similarly impose a 
                topology $\tau_Y$ on $Y$:
                \[
                    \tau_Y= \{ V \subset Y \mid f^{-1}(V) \text{ is open in }X \}.
                \]
                This is called  the \textbf{coinduced topology on} $Y$. 
                Show that this is also a functorial process. 

            \end{itemize}


            \item[\textbf{4.}]
            \begin{itemize}
                \item[\emph{i}.]
                Let $X$, $Y$ be topological spaces with $\phi: X \to Y$ a continuous function.
                Show that this induces a functor $\phi_*: (\textbf{Top}\downarrow X) \to (\textbf{Top}\downarrow Y)$ 
                where on objects $(f: E \to X) \mapsto (\phi \circ f: E \to Y)$. 

                \item[\emph{ii}.]
                Let $\cc$ be a category. Show that we generalize (\emph{i}) to define a functor 
                \[
                    (\cc \downarrow -): \cc \to \textbf{Cat}
                \]
                where $A \mapsto (\cc \downarrow A)$.


                \item[\emph{ii}.]
                Let $\textbf{Cat}_*$ be the \textbf{pointed category of categories} 
                which we describe as 
                \begin{description}
                    \item[Objects.] All pairs $(\cc, A)$ with $\cc$ a category and $A \in C$
                    \item[Morphisms.] Functors $F$ which preserve the objects.  
                \end{description}
                Can we overall describe the construction of a slice category as a functor
                \[
                    (-  \downarrow -): \textbf{Cat}_* \to \textbf{Cat}    
                \]
                where $(\cc,  A) \mapsto  (\cc \downarrow A)$? 
            \end{itemize}
            
            \item[\textbf{5.}]
            In this exercise we'll see that slice categories describe intervals for thin categories.
            \begin{itemize}
                \item[\emph{i}.] Regard $\mathbb{R}$ as 
                a thin category, specifically as one with a partial order. For a given  $a \in \rr$, 
                describe the thin category $(a \downarrow  \mathbb{R})$.

                \item[\emph{ii}.] Suppose $P$ is a partial order (so that $p \le p'$ and $p' \le p$ implies 
                $p = p'$). Describe in general the categories $(p \downarrow P)$ and $(P \downarrow p)$.
            \end{itemize}
            
        \end{itemize}

        
    \newpage
    \section{Graphs, Quivers and Free Categories}
    In studying category it is often helpful to imagine the objects
    and morphisms in action as vertices and edges corresponding to a
    graph. In fact, such a pictorial representation of a category is
    not even incorrect; one can pass categories and graphs from one to
    the other. To speak of this, we first review some terminology.
    
    \begin{definition}
        A (small) \textbf{graph} $G$ is a set of vertices $V(G)$ and a set 
        edges $E(G)$ such that there exists an assignment function
        \[
            \partial: E(G) \to V(G)\times V(G)
        \]
        which assigns every edge to the ordered pair containing its endpoints.

        On the other hand, a \textbf{directed graph} is a graph $G$ 
        where $E(G)$ is now a set of 2-tuples $(v_1, v_2)$. This allows 
        each edge of $E(G)$ to have a specified direction. In this case, 
        the assignment function has the form $\partial: E(G) \to V(G)$. 
    \end{definition}
    Now, how do we formulate a morphism between two graphs? 

    \begin{definition}
        A \textbf{graph homomorphism} between two graphs $G$ and $H$ is a 
        function $f: G \to H$ which induces maps $f_V: V(G) \to V(H)$ and
        $f_E: E(G) \to E(H)$ where if $\partial(e) = (v_1, v_2)$, then  
        \[
            \partial \circ f_E(e) = (f_V(v_1), f_V(v_2)).
        \]
    \end{definition}

    \begin{center}
        \begin{tikzpicture}
            \draw[->] (1.5, 2.2) to [bend left] (4.5, 2.2) node at (3,3) {$f_V$};
            \draw[thick, Magenta, *-*] (-1,0)--(1,2) node at (-0.2, 1.2) {$\textcolor{Black}{e}$};
            \draw[thick, ProcessBlue, *-*] (5,2)--(7,0) node at (6.2, 1.2) {$\textcolor{Black}{e'}$};
            \draw[->] (0.5, 1) to [bend right] (5, 1) node at (3, 0) {$f_E$};
            \node at (0.8, 2.3) {$\textcolor{Black}{v_2}$};
            \node at (-1.3, 0){$\textcolor{Black}{v_1}$};
            \node at (5, 2.35) {$\textcolor{Black}{v_2'}$};
            \node at (7.3, 0) {$\textcolor{Black}{v_1'}$};
        \end{tikzpicture}    
    \end{center}

    In some sense, this behaves almost like a functor. This observation 
    will become important later. 
    Now since we have a consistent way to speak of graphs and their
    morphisms, we can form the category \textbf{Grph} where the
    objects are small graphs and the morphisms are graph morphisms as
    described above. 

    Finally we introduce the concept of a quiver, which we will 
    see is basically the skeleton of a category. 

    \begin{definition}
        A \textbf{quiver} is a directed graph $G$ which allows multiple edges 
        between vertices. Instead of a function $\partial$, a quiver is equipped with 
        \textbf{source} and \textbf{target} functions 
        \[
            s: E(G) \to V(G) \qquad t: E(G) \to V(G).
        \]
        So a quiver is a 4-tuple $(E(G), V(G), s, t)$.
        Now as before, a \textbf{morphism} $f: Q \to Q'$ between quivers $(E(Q), V(Q), s, t)$ and 
        $(E(Q'), V(Q'), s', t')$ is one which preserves edge-vertex
        relations. Thus, it is a pair of functions $f_E: E(Q) \to E(Q')$ 
        and $f_V: V(Q) \to V(Q)'$ such that 
        \[
            f_V \circ s = s' \circ f_E \qquad f_V \circ t = t' \circ f_E.
        \]
    \end{definition}

    \begin{center}
        \begin{tikzpicture}
            \draw[thick, ProcessBlue, *-*] (4,0) -- (0,0) node[left]{$\textcolor{Black}{v_1}$};
            \draw[thick, ProcessBlue, *-*] (0,0) -- (6,0) node[right]{$\textcolor{Black}{v_3}$};
            \draw[thick, ProcessBlue, ->] (0, 0.2) to [bend left = 30] (3.8, 0.2);
            \draw[thick, ProcessBlue, ->] (0, 0.2) to [bend left = 80] (4, 0.2);
            \draw[thick, ProcessBlue, ->] (0, -0.2) to [bend right = 60] (5.8, -0.2);
            \draw[thick, ProcessBlue, ->] (0, -0.2) to [bend right = 40] (3.8, -0.2);
            \node at (4.2, -0.3) {$\textcolor{Black}{v_2}$};
            \node at (2, 1) {$\textcolor{Black}{e_1}$};
            \node at (2, -1.2) {$\textcolor{Black}{e_2}$};
            \node at (3, -1.9) {$\textcolor{Black}{e_4}$};
            \node at (2, 1.6) {$\textcolor{Black}{e_3}$};
        \end{tikzpicture}
    \end{center}

    \noindent Now that we have all of those definitions out of the way, what's really going 
    on here? A quiver can be abstracted as a pair of objects and morphisms.
    \begin{center}
        \begin{tikzcd}[row sep = 1.4cm, column sep = 1.4cm]
            E \arrow[r, shift left=0.6ex,"s"]
            \arrow[r,shift left =-0.6ex, swap, "t"]
            &
            V
        \end{tikzcd}
    \end{center}
    If we let $C\op$ be the category with two objects, two nontrivial morphisms 
    and two identity morphisms as below
    \begin{center}
        \begin{tikzcd}[row sep = 1.4cm, column sep = 1.4cm]
            1 \arrow[r, shift left=0.6ex,"f"]
            \arrow[r,shift left =-0.6ex, swap, "g"]
            &
            0
        \end{tikzcd}
    \end{center}
    then we see that a \textbf{quiver is a functor} $F: \cc\op \to \textbf{Set}$. 
    With that said, 
    we can define the \textbf{category of quivers Quiv}, which, based on what we just showed, 
    is a functor category with objects $F: \cc\op \to \textbf{Set}$. This 
    allows us to interpret quiver homomorphisms as natural transformations.
    
    \textcolor{MidnightBlue}{Now why on earth do we care about these things called quivers?}
    The reason is because the underlying structure of small categories 
    take the form of a quiver. For example, the category on the left below 
    can be turned into a quiver, as on the right, after "forgetting" composition and identity 
    morphisms. 
    \begin{center}
        \begin{tikzcd}[row sep = 1.4cm, column sep = 1.4cm]
            & A\arrow[dr, "f"]
            \arrow[out=120,in=60,looseness=3,loop, "1_A"] &\\
            C \arrow[ur, shift left = 0.5ex, "h"] 
            \arrow[ur, shift left=-0.5ex, swap, "g"]
            \arrow[rr, bend right = 20, swap, "f \circ h"]
            \arrow[rr, bend left = 20, "f \circ g"]
            \arrow[rr, "j"]
            \arrow[out=120,in=60,looseness=3,loop, "1_C"] 
            & & B
            \arrow[out=115,in=65,looseness=4,loop,"1_B"]
        \end{tikzcd}
        \hspace{1cm}
        \begin{tikzcd}[row sep = 1.4cm, column sep = 1.4cm, ProcessBlue]
            & \textcolor{Black}{\bullet} \arrow[dr]
            \arrow[out=120,in=60,looseness=3,loop] &\\
            \textcolor{Black}{\bullet} \arrow[ur, shift left = 0.5ex] 
            \arrow[ur, shift left=-0.5ex, swap]
            \arrow[rr, bend right = 20, swap]
            \arrow[rr, bend left = 20]
            \arrow[rr]
            \arrow[out=120,in=60,looseness=3,loop] 
            & & \textcolor{Black}{\bullet}
            \arrow[out=115,in=65,looseness=4,loop]
        \end{tikzcd}
    \end{center}

    In general, since categories allow multiple arrows between objects, 
    we can construct a forgetful functor which forgets composition 
    and identity arrows.
    \[
        U: \textbf{Cat} \to \textbf{Quiv}.
    \]
    Note that if $F : \cc \to \cc'$ is a functor then $U(F)
    : U(\cc) \to U(\cc')$ is in fact a well-behaved morphism between
    two quivers. \textcolor{NavyBlue}{Recall that the construction of a
    graph homomorphism is basically a functor as we've known to so far}. 

    Not only can we forget categories to generate quivers, we can generate 
    categories using the skeletal structure of a quiver. This leads to the concept 
    of a \textbf{free category}; the concept is no different than the concept 
    of, say, a free group generated by a set $X$. 

    \begin{definition}
        Let $Q$ be a quiver with vertex set $V$ and edge set $E$. 
        We define the \textbf{free category generated by $Q$} as 
        the category with 
        \begin{description}
            \item[Objects.] The set $V$ 
            \item[Morphisms.] The \textbf{paths} of the quiver.   
        \end{description}
        Precisely, a \textbf{path} is any sequence of edges and vertices 
        \begin{center}
            \begin{tikzcd}
                v_0 \arrow[r, "e_0"] 
                &
                v_1 \arrow[r, "e_1"]
                &
                \cdots \arrow[r, "e_{n-1}"]
                &
                v_n
            \end{tikzcd}
        \end{center}
        with composition of paths defined in the intuitive way: 
        \begin{center}
            \begin{tikzcd}
                (v_0 \arrow[r, "e_0"] 
                &
                v_1 \arrow[r, "e_1"]
                &
                \cdots \arrow[r, "e_{n-1}"]
                &
                v_n)
            \end{tikzcd}
            $\circ$
            \begin{tikzcd}
                (v_n \arrow[r, "e'_0"] 
                &
                w_0 \arrow[r, "e'_1"]
                &
                w_1 \arrow[r, "e'_2"]
                &
                \cdots \arrow[r, "e'_{m}"]
                &
                w_m)
            \end{tikzcd}
        \end{center}
        \begin{center}
            $=$
            \begin{tikzcd}
                v_0 \arrow[r, "e_0"] 
                &
                v_1 \arrow[r, "e_1"]
                &
                \cdots \arrow[r, "e_{n-1}"]
                &
                v_n \arrow[r, "e'_0"] 
                &
                w_0 \arrow[r, "e'_1"]
                &
                w_1 \arrow[r, "e'_2"]
                &
                \cdots \arrow[r, "e'_{m}"]
                &
                w_m
            \end{tikzcd}
        \end{center}
        When we generate the free category, we also remember to add identity arrows to 
        each vertex. 
    \end{definition}

    Since for each quiver $Q$, we can define a free category $F_C(Q)$ on $Q$, 
    we can realize that this mapping is functorial. That is,  we can define a
    functor 
    \[ 
        F_C: \textbf{Quiv} \to \textbf{Cat}
    \] 
    where it maps on objects and morphisms as
    \begin{statement}{Red!10}
    \begin{align_topbot}
        Q &\longmapsto F_C(Q)\\
        (f: Q \to Q') &\longmapsto (F_C(f): F_C(Q') \to F_C(Q)).
    \end{align_topbot}
    \end{statement}
    That is, quiver homomorphisms can map to functors $F_C(f)$ between the free categories
    generated by the respective quivers. 
    
    Now, what is the relationship between a quiver $Q$ and the quiver $U(F_C(Q))$? 
    There must exist an injection $i: Q \to U(F(Q))$ which sends $Q$ to the skeleton 
    of $U(F_C(Q))$. It turns out that this morphism is universal from $Q$ to $U$. 


    \begin{thm}
        Let $Q$ be a quiver. Then there is a graph homomorphism $i: Q \to U(F_C(Q))$ 
        such that, for any other graph homomorphism
        $\phi: Q \to U(\cc)$ with $\cc$ a category, there exists a unique
        functor $F: F_C(Q) \to \cc$ where $U(F) \circ i = \phi$. That is,
        \begin{center}
            \begin{tikzcd}[row sep = 1.4cm, column sep = 1.4cm]
                Q \arrow[r, "i"] \arrow[dr, swap, "\phi"] & U(F_C(Q))\arrow[d, dashed, "U(F)"] \\
                & U(\cc)
            \end{tikzcd}
            \hspace{1cm}
            \begin{tikzcd}[row sep = 1.4cm, column sep = 1.4cm]
                F_C(Q) \arrow[d, dashed, "F"]\\
                \cc
            \end{tikzcd}
        \end{center} 
    \end{thm}
    This is an example of a universal arrow; the dotted lines are the
    morphisms which are forced to exist by the conditions of the
    diagram, which is the idea of a universal element.

    \begin{prf}
        Denote each morphism or path in $F_C(Q)$ of length $n$
        \begin{center}
            \begin{tikzcd}
                v_0 \arrow[r, "e_0"] 
                &
                v_1 \arrow[r, "e_1"]
                &
                \cdots \arrow[r, "e_{n-1}"]
                &
                v_n
            \end{tikzcd}
        \end{center}
        as $(v_0, e_0e_1\cdots e_{n-1}, v_n): v_0 \to v_n$. Now define the inclusion 
        $i: Q \to U(F_C(Q))$ where each vertex and edge is sent identically. That is, vertices 
        $v$ map to $v$ in $F_C(Q)$,  and morphisms are sent identically and for each 
        edge $e: v \to v'$:
        \[
            i(e: v \to v') = (v, e, v').
        \]
        An important observation to make is the fact that every morphism $(v_0, e_0e_1\cdots e_{n-1}, v_n): v_0 \to v_n)$
        in $F_C(Q)$ is a composition of length 2-morphism: 
        \begin{center}
            \begin{tikzcd}
                v_0 \arrow[r, "e_0"] 
                &
                v_1 \arrow[r, "e_1"]
                &
                \cdots \arrow[r, "e_{n-1}"]
                &
                v_n
            \end{tikzcd}
        \end{center}
        \begin{center}
            $=$
            \begin{tikzcd}
                (v_0 \arrow[r, "e_0"] 
                &
                v_1)
            \end{tikzcd}
            $\circ$
            \begin{tikzcd}
                (v_1 \arrow[r, "e_1"] 
                &
                v_2)
            \end{tikzcd}
            $\circ$
            $\cdots$
            $\circ$
            \begin{tikzcd}
                (v_{n-1} \arrow[r, "e_{n-1}"] 
                &
                v_n)
            \end{tikzcd}
        \end{center}
        Therefore, for any graph homomorphism $\phi: Q \to U(\cc)$, 
        we can create a unique functor $F: F_C(Q) \to \cc$ where 
        \begin{align*}
            v &\longmapsto \phi(v)\\
            (v_0, e_0e_1\cdots e_{n-1}, v_n): v_0 \to v_n &\longmapsto 
            \phi(e_0:  v_0 \to v_1) \circ \phi(e_1: v_1 \to v_2)
            \circ \cdots \circ \phi(e_{n-1}: v_{n-1} \to v_n)
        \end{align*}
        which then gives us 
        \begin{align*}
            U(F) \circ i = \phi
        \end{align*}
        as desired. 
    \end{prf}
    
    \newpage
    \section{Quotient Categories}
    The quotient category is a concept that generalizes the ideas of
    forming quotient groups, rings, modules, and even topological
    spaces. The core idea of obtaining a quotient "object" revolves
    around the concept of an equivalence class. 

    For example, in constructing the quotient group, one can go about
    constructing it in two different ways. One is easy, in which you
    simply form the concept of a coset, and then observe that nice
    things happen when you make cosets with normal subgroups. The hard
    way is to construct an equivalence relation, which \textit{gives
    rise} to what we recognize as the concept of a coset, and then
    continuing further to create the quotient groups from normal
    subgroups. Both ways are equivalent, but one ignores the crucial
    and powerful idea of equivalence relations. 

    \begin{definition}
        Let $\cc$ be a locally small category. Suppose $R$ is a function which, for every  
        pair of objects $A, B$, assigns equivalence 
        relations $\sim_{A, B}$ on the hom set $\hom_{\cc}(A, B)$. Then we may define the 
        quotient category $\cc/R$ where 
        \begin{description}
            \item[Objects.] The same objects of $\cc$.
            \item[Morphisms.] For any objects $A,B$ of $\cc$, we 
            set $\hom_{\cc/R}(A, B) = \hom_{\cc}(A, B)/\sim_{A, B}$.
        \end{description}
    \end{definition}    
    Thus we see that morphisms between $f: A \to B$ in $\cc$ becomes equivalence classes 
    $[f]$ in $\cc/R$. 

    With that said, we can naturally define a canonical functor $Q: \cc \to \cc/R$ 
    where $Q$ acts identically on objects and where 
    $Q(f: A \to B) = [f] \in \hom_{\cc/R}(A,B)$. This in fact defines a functor 
    if we observe that, for a pair of composable morphisms $g, f$.
    \[
        Q(g) \circ Q(f) = [g \circ f] = Q(g \circ f). 
    \]
    A nice property of this functor is the fact that if $f \sim f'$, then $Q(f) = Q(f')$.
    What is even nicer about this functor is that it has the following property. 

    \begin{proposition}
        Let $\cc$ be a locally small category with an equivalence relation $\sim_{A, B}$ on each set 
        $\hom_{\cc}(A, B)$. Then for any functor $F: \cc \to \dd$ into some category $\dd$ 
        such that $f \sim f'$, $F(f) = F(f')$, there exists a \emph{unique} functor 
        $H: \cc/R \to \dd$ such that $H \circ Q = F$;
        or, diagrammatically, such that the following diagram commutes.
        \begin{center}
            \begin{tikzcd}[row sep = 1.4cm, column sep = 1.4cm]
                \cc \arrow[r, "Q"] 
                \arrow[dr, swap, "F"]
                &
                \cc/R
                \arrow[d, dashed, "H"]
                \\
                &
                \dd 
            \end{tikzcd}
        \end{center}
    \end{proposition}

    \begin{prf}
        Observe that one functor $H: \cc/R \to \dd$ that we can supply, 
        which will have the above diagram commute, is one where 
        $H(C) = F(C)$ on objects and where for any $[f] \in \hom_{\cc/R}(A, B)$,
        \[
            H([f]) = F(f)   
        \]
        where $f$ is an representative of the equivalence class $f$. Note that 
        this is well defined since $F(f) = F(f')$ if $f \sim_{A,B} f'$; hence this 
        will appropriately send equivalent elements to the same morphism. It 
        is not hard to show that it's unique; one can just suppose such an $H$ exists and then 
        demonstrate that it behaves like the functor we proposed initially. 
    \end{prf}

    \begin{example}
        
    \end{example}


    
    

\newpage
\section{Monoids, Groups and Groupoids in Categories}

One of the most simplest, useful and yet underrated concepts in mathematics 
is the concept of a monoid. The reason why monoids are so useful is because 
they capture three main concepts: \textbf{stacking} "things" together to create another 
"thing," in such a way that our stacking operation is \textbf{associative}, 
with the additional assumption of an \textbf{identity} element which doesn't change the 
value. Often times in cooking up a mathematical construction, we want to maintain these 
three concepts because they are so familiar to our basic human nature. 

Now recall the definition of a monoid.
\begin{definition}
    A monoid $M$ is a set equipped with a binary operation $\cdot: M \times M \to M$ 
    and an identity element $e$ such that 
    \begin{itemize}
        \item[1.] For any $x, y, z \in M$, we have that $x \cdot (y \cdot z) = (x \cdot y) \cdot z$ 
        \item[2.] For any $x \in M$, $x \cdot e = x = e \cdot x$.  
    \end{itemize}
\end{definition} 
It turns out that we can abstract the above definition very easily if we just resist the 
temptation to explicitly refer to our elements. In order to do this, we need 
to find a way to diagrammatically express the above axioms. 

Towards that goal, rename the binary operation as $\mu: M \times M \to M$ (for notational convenience). 
Then to express axiom 
(1), we mean that we have 3 elements $x, y, z \in M$ and there are two ways to compute them, but we  
want them to be the same. So lets make each different way to compute them one side of a square, which 
we'll say it commutes. 
\begin{center}
    \begin{tikzcd}[row sep = 1.4cm, column sep = 1.4cm]
        (x, y, z) 
        \arrow[r, maps to, "\mu \times 1"]
        \arrow[d, maps to, swap, "1 \times \mu"]
        &
        (x \cdot y, z)
        \arrow[d, maps to, "\mu"]
        \\
        (x, y \cdot z)
        \arrow[r, maps to, swap, "\mu"]
        &
        x\cdot (y\cdot z) = (x \cdot y) \cdot z
    \end{tikzcd}
    \hspace{1cm}
    \begin{tikzcd}[row sep = 1.4cm, column sep = 1.4cm]
        M \times M \times M 
        \arrow[r, "\mu \times 1"]
        \arrow[d, swap, "1 \times \mu"]
        &
        M \times M
        \arrow[d, "\mu"]
        \\
        M \times M
        \arrow[r, swap, "\mu"]
        &
        M
    \end{tikzcd}
\end{center}
The result is the diagram on the above left. Since we want this to hold for 
all elements in $M$, we construct the diagram more generally on the above right; 
this expresses our associativity axiom. Now to express the second axiom diagrammatically, we need a way to discuss the 
identity map. So define the map $\eta: \{\bullet\} \to M$ where $\eta(\bullet) = e$. This 
is just a stupid map that picks out the identity. Then axiom (2) can be translated 
diagramatically to state that the bottom left diagram commutes. 
\begin{center}
    \begin{tikzcd}[row sep = 1.4cm, column sep = 0.7cm]
        (\bullet, m) 
        \arrow[r, maps to, "\eta \times 1_M"]
        \arrow[dr, maps to, swap, "\pi_M", end anchor={[xshift = -8ex]}]
        &
        (e, m) 
        \arrow[d, maps to, shift right = -0.5ex, start anchor = {[xshift = 3ex]},end anchor={[xshift = 3ex]}, "\mu"]
        \quad
        (m, e)
        \arrow[d, maps to, shift right = 0.5ex, swap, start anchor = {[xshift = -3ex]}, end anchor={[xshift = -3ex]}, "\mu"]
        &
        (m, \bullet) \arrow[l, swap, maps to, "1_M \times \eta"]
        \arrow[dl, maps to, "1_M \times \eta", end anchor={[xshift=8ex]}]
        \\
        &
        m = e \cdot m = m \cdot e = m
        &
    \end{tikzcd}
    \begin{tikzcd}[row sep = 1.4cm, column sep = 1cm]
        \{\bullet\}\times M
        \arrow[r, "\eta \times 1_M"]
        \arrow[dr, swap, "\pi_M"]
        &
        M \times M
        \arrow[d, "\mu"]
        &
        M \times \{\bullet\}
        \arrow[l, swap, "1_M \times \eta"]
        \arrow[dl, "\pi_M"]
        \\
        &
        M
        &
    \end{tikzcd}
\end{center}
Since we want this to hold for all $m \in M$, we generalize this to create a
commutative diagram as on the above right. We now have what we need to define a monoid more generally. 

\begin{definition}
    Let $\cc$ be a category with cartesian products. Denote the terminal 
    object as $T$. An object $M$ is said to 
    be a \textbf{monoid} in $\cc$ if there exist maps 
    \begin{align*}
        &\mu: M \times M \to M \qquad &&\textbf{(Multiplication)}\\
        &\eta: T \to M \qquad &&\textbf{(Identity)}
    \end{align*}
    such that the diagrams below commute. 
    \begin{center}
        \begin{tikzcd}[row sep = 1.4cm, column sep = 1.4cm]
            M \times M \times M 
            \arrow[r, "\mu \times 1"]
            \arrow[d, swap, "1 \times \mu"]
            &
            M \times M
            \arrow[d, "\mu"]
            \\
            M \times M
            \arrow[r, swap, "\mu"]
            &
            M
        \end{tikzcd}
        \hspace{1cm}
        \begin{tikzcd}[row sep = 1.4cm, column sep = 1cm]
            T\times M
            \arrow[r, "\eta \times 1_M"]
            \arrow[dr, swap, "\pi_M'"]
            &
            M \times M
            \arrow[d, "\mu"]
            &
            M \times T
            \arrow[l, swap, "1_M \times \eta"]
            \arrow[dl, "\pi_M"]
            \\
            &
            M
            &
        \end{tikzcd}
    \end{center}
    Dually, a \textbf{comonoid} is an object $C$ with maps 
    \begin{align*}
        &\Delta: C \to C \times C \qquad &&\textbf{(Comultiplication)}\\
        &\epsilon: C \to T \qquad &&\textbf{(Identity)}
    \end{align*}
    such that the dual diagrams commute. 
    \begin{center}
        \begin{tikzcd}[row sep = 1.4cm, column sep = 1.4cm]
            C \arrow[r, "\Delta"]
            \arrow[d, "\Delta"]
            &
            C \times C
            \arrow[d, "\Delta \times 1_C"]
            \\
            C \times C 
            \arrow[r, swap, "1_C \otimes \Delta"]
            &
            C \times C \times C
        \end{tikzcd}
        \hspace{1cm}
        \begin{tikzcd}[row sep = 1.4cm, column sep = 1cm]
            T\times C
            &
            \arrow[l, "\epsilon \times 1_C"]
            C \times C
            \arrow[r, "1_C \times \epsilon"]
            &
            C \times T
            \\
            &
            C
            \arrow[ul, "\sigma'"]
            \arrow[ur, "\sigma"]
            \arrow[u, "\Delta"]
            &
        \end{tikzcd}
    \end{center}
\end{definition}
\textcolor{NavyBlue}{Note that we're being a little sloppy here. For example, the object 
$M \times M \times M$ doesn't actually exist; we have either $M \times (M \times M)$ or $(M \times M) \times M$. 
} However, for any category with cartesian products, we always have that these two objects 
are isomorphic. Hence we mean either of the equivalent products when we discuss $M \times M \times M$. 

\begin{example}
    Let $k$ be a field. Consider the category $\textbf{Vect}_k$. Then a monoid 
    in this category is an object $A$ equipped with maps 
\end{example}

\begin{example}
    Group object in the category of Top is a topological group.
\end{example}

\begin{example}
    Monoid in the category of $R$ modules is an associative algebra.
\end{example}