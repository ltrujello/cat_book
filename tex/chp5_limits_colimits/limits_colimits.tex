\chapter{Limits and Colimits.}
    Before we begin, we reintroduce certain terminology. 
    \begin{definition}
        Let $\cc$ be a category. We define a \textbf{diagram} of a
        \textbf{shape} $J$ to be a functor $F:J \to \cc$. 
        \\
        Here, $J$ is
        generally thought of as an indexing category.
        We use the word diagram because the image of $J$ under $F$ is
        literally a diagram of morphisms. 
    \end{definition}
    
    \begin{center}
        \begin{tikzcd}[column sep =  1.4cm, row sep = 1.4cm]
            & i \arrow[dr, "g"]& \\
            j \arrow[ur, "f"] \arrow[rr, swap, "h"]& & k
        \end{tikzcd}
        \begin{tikzcd}
            \textcolor{white}{b} \arrow[r, scale = 1.5, "F"] 
            & \textcolor{white}{b}
        \end{tikzcd}
        \begin{tikzcd}[column sep =  1.4cm, row sep = 1.4cm]
            & F(i) \arrow[dr, "F(g)"]& \\
            F(j) \arrow[ur, "F(f)"] \arrow[rr, swap, "F(h)"]& & F(k) 
        \end{tikzcd}
    \end{center}
    In this example, on the left we have the category $J$, and on the
    right we have the diagram of $J$ in $\cc$. Now recall the \textbf{diagonal functor}
    \[
        \Delta : \cc \to \cc^J   
    \]
    is a functor which sends each object $C \in \cc$ to the functor
    $\Delta(C): J \to \cc$, where, for each $j \in J$, we have 
    \[
        \Delta(C)(j) = C.  
    \]
    This motivates the following concept.
    \begin{definition}
        Let $\cc$ be a category and $F: J \to \cc$ be a functor, where
        $J$ is a small category. We define a 
        \textbf{cone over $\bm{F}$ with apex $\bm{C}$}
        to be a natural transformation 
        \[
            \Delta(C) \to F.  
        \]
        \textbf{\textcolor{purple}{Equivalently}}, it
        is an object $C$ equipped with morphisms 
        $u_i: C \to F(i)$ for each $i \in J$ 
        such that, for every $f: i \to j$ in $J$,
        the diagram 
        \begin{center}
            \begin{tikzcd}[column sep =  1.4cm, row sep = 1.4cm]
                & C \arrow[dl, swap, "u_i"] \arrow[dr, "u_j"]& \\
                F(i) \arrow[rr, swap, "f"]
                &  & F(j) 
            \end{tikzcd} 
        \end{center}
        commutes.

        In the same fashion, we may define a \textbf{cocone with base
        $C$ under $F$} as a natural transformation 
        \[
            F \to \Delta(C).
        \]
        \textbf{\textcolor{purple}{Equivalently}}, it 
        is an object $C$ equipped with morphisms $u_i: F(i) \to C$ 
        for each $i \in J$ such that, for every $f: i \to j$ in $J$,
        the diagram
        \begin{center}
            \begin{tikzcd}[column sep =  1.4cm, row sep = 1.4cm]
                F(i) \arrow[rr, "f"]
                \arrow[dr,swap, "u_i"]
                &  & F(j) \arrow[dl, "u_j"]\\
                & C &
            \end{tikzcd} 
        \end{center}
        commutes. 
    \end{definition}

    \textcolor{MidnightBlue}{Alternatively, we could have defined the
    above, second definition as a "cone," and then defined the first
    definition as the "cocone". Why? Well, it's the same arbitrary
    nature in which physicists encountered electrical charge; one was
    named negative, the other was named positive. 
    For all we know, in a parallel
    universe protons were said to have "negative" charge and 
    electrons were said to have "positive" charge. In the end, nomenclature  
    is arbitrary when it comes to duality.
    }

    Try to recall: what is a \textbf{limit} in a category $\cc$?
    When we speak of one, we're talking about the limit of a functor 
    \[
        F: J \to \cc.        
    \]
    There are multiple, but 
    equivalent ways to think about it. 
    \begin{itemize}
        \item A limit can be thought of as a \textbf{universal object} 
        $(\Lim F, u: \Delta(\Lim F) \to F)$ from $\Delta$ to $F$. 
        \begin{center}
            \begin{tikzcd}[column sep =  1.4cm, row sep = 1.4cm]
                F & \Delta(\Lim F) \arrow[l, swap, "u"]\\
                & \Delta(C) \arrow[ul, "f"] \arrow[u, dashed, swap, "\Delta(h)"]
            \end{tikzcd}
            \hspace{1cm}
            \begin{tikzcd}[column sep =  1.4cm, row sep = 1.4cm]
                \Lim F \\ 
                C \arrow[u, dashed, swap, "h"]
            \end{tikzcd}
        \end{center}
        
        \item A limit can also be thought of as a \textbf{universal
        cone}. We know that if we have a limit, then we have an 
        object $\Lim F$ and
        a natural transformation $\Delta(\Lim F) \to F$. Hence, this
        forms a cone. As we also pointed out, a cone induces a family 
        of morphisms $u_i: \Lim F \to F(i)$. 

        What makes this cone a "universal" cone is the fact that, 
        for any other cone $\Delta(C) \to F$, the above diagram establishes
        the diagram below. 
        \begin{center}
            \begin{tikzcd}[column sep = 1.4cm, row sep = 1.4cm]
                & C \arrow[d, dashed, "h"] \arrow[ddl,swap, bend right = 20,
                "f_i"] \arrow[ddr, bend left = 20, "f_j"] &\\
                & \arrow[dl, swap, "u_i"] \Lim F 
                \arrow[dr, "u_j"] & \\
                F(i)
                \arrow[rr, "F(g)"] 
                & 
                &
                F(j) 
            \end{tikzcd}
        \end{center}

        \item In a better way, one can think of it as being a
        \textcolor{NavyBlue}{\textit{universal spider!}} One could also
        think of it as a squished spider, or more optimistically, a two
        dimensional spider. 
        \begin{center}
            \begin{tikzcd}[column sep =  1.4cm, row sep = 1.4cm]
                & & C \arrow[d, dashed, "f"] 
                \arrow[ddll, bend right, swap, "v_i"]
                \arrow[ddl, bend right, crossing over, swap, "v_j"]
                \arrow[ddr, bend left, crossing over, "v_k"]
                \arrow[ddrr, bend left, "v_l"]
                & &  \\
                & & \textcolor{NavyBlue}{\Lim F}
                \arrow[dll, NavyBlue, near start, swap, "u_i"]
                \arrow[dl, NavyBlue, "u_j"] 
                \arrow[dr, NavyBlue, swap, "u_k"]
                \arrow[drr, NavyBlue, near start, "u_l"] & & \\
                F(i) & F(j) & \cdots & F(k) & F(l)
            \end{tikzcd}
        \end{center}
    \end{itemize}

    Now try to recall what \textbf{Colimits} of a diagram $F: J \to
    \cc$ are. As before, there are multiple, but 
    equivalent ways to think about it. 
    \begin{itemize}
        \item A colimit can be thought of as a \textbf{universal object} 
        $(\Colim F, u: F \to \Delta(\Colim F))$ from $F$ to $\Delta$. 
        \begin{center}
            \begin{tikzcd}[column sep =  1.4cm, row sep = 1.4cm]
                F \arrow[r, "u"] \arrow[dr, swap, "f"]
                & \Delta(\Colim F) \arrow[d, dashed, "\Delta(h)"]\\
                & \Delta(C)  
            \end{tikzcd}
            \hspace{1cm}
            \begin{tikzcd}[column sep =  1.4cm, row sep = 1.4cm]
                \Lim F \arrow[d, dashed, swap, "h"]\\ 
                C 
            \end{tikzcd}
        \end{center}
        
        \item A colimit can also be thought of as a \textbf{universal
        cocone} (or just cone). Given a colimit of $F: J \to \cc$, we have an 
        object $\Colim F$ and
        a natural transformation $u: F \to \Delta(\Colim F)$. Hence, this
        forms a cocone (again, or just cone). 
        As we also pointed out, a cocone induces a family 
        of morphisms $u_i: F(i) \to \Colim F $. 

        What makes this cocone a "universal" cocone is the fact that, 
        for any other cocone $F \to \Delta(C)$, the above diagram establishes
        the diagram below. 
        \begin{center}
            \begin{tikzcd}[column sep = 1.4cm, row sep = 1.4cm]
                F(i) \arrow[rr, "F(g)"] \arrow[ddr,swap, bend right = 20,
                "f_i"] \arrow[dr, swap, "u_i"]
                & 
                &
                F(j) \arrow[ddl, bend left = 20, "f_j"]
                \arrow[dl, "u_j"]
                \\
                & \Lim F \arrow[d, dashed, "h"] &\\
                & C &
            \end{tikzcd}
        \end{center}

        \item One can also think of this as a 
        \textcolor{NavyBlue}{\textit{universal spider!}} 
        Or it can be thought of as a 
        \textcolor{NavyBlue}{\textit{jealous object}}; if any other
        object $C$ is "the center of attention,"i.e. has morphisms 
        pointing to it, $\Colim F$ will get
        angry, so the morphisms have to go through $\Colim F$ via $f$ 
        first before they reach $C$.
        \begin{center}
            \begin{tikzcd}[column sep =  1.4cm, row sep = 1.5cm]
                & & C \arrow[<-, d, dashed, "h"] 
                \arrow[<-, ddll, bend right, swap, "f_i"]
                \arrow[<-, ddl, bend right, crossing over, swap, "f_j"]
                \arrow[<-, ddr, bend left, crossing over, "f_k"]
                \arrow[<-, ddrr, bend left, "f_l"]
                & &  \\
                & & \textcolor{NavyBlue}{\Colim F}
                \arrow[<-, dll, NavyBlue, near start, swap, "u_i"]
                \arrow[<-, dl, NavyBlue, "u_j"] 
                \arrow[<-, dr, NavyBlue, swap, "u_k"]
                \arrow[<-, drr, NavyBlue, near start, "u_l"] & & \\
                F(i) & F(j) & \cdots & F(k) & F(l)
            \end{tikzcd}
        \end{center}
    \end{itemize}

    \newpage
    \section{Every Limit in Set; Creation of Limits}
    
    While we have been discussing limits and colimits of functors, we generally 
    consider the case in which they exist. However, they sometimes don't exist; after all, 
    limits and colimits are universal objects. 
    Categories which do admit 
    these constructions are  often convenient places to work inside of.  
    This is analogous to \textbf{complete metric spaces} $X$,  
    where  every Cauchy sequence is convergent in $X$. With such an 
    analogy in  mind, the following definition should make sense.

    \begin{definition}
        Let $\cc$ be a category. We say $\cc$ is \textbf{complete} if
        all small diagrams in $\cc$ has limits in $\cc$; in other words, 
        if every functor $F: J \to \cc$, where $J$ is a small 
        category, has a limit in $\cc$.  
    \end{definition}
    Similarly, we define:       
    \begin{definition}
        Let $\cc$ be a category. We say $\cc$ is \textbf{cocomplete}
        if all small diagrams in $\cc$ has colimits in $\cc$. In other words, 
        every functor $F: J \to \cc$, where $J$ is a small category, 
        has a colimit in $\cc$.
    \end{definition}

    Now we show how to construct limits inside of \textbf{Set},
    thereby showing that this category is complete. 

    \begin{example}
        For this example, let $J = \omega\op$, where $\omega$ is the 
        preorder of natural numbers. Since we are asking for the 
        opposite category, we reverse the arrows and 
        get the diagram below.
        \begin{center}
            \begin{tikzcd}
                0 \arrow[out=120,in=60,looseness=3,loop]
                &
                1 \arrow[l] \arrow[out=120,in=60,looseness=3,loop]
                &
                2 \arrow[l] \arrow[out=120,in=60,looseness=3,loop]
                &
                3 \arrow[l] \arrow[out=120,in=60,looseness=3,loop]
                &
                \arrow[l]
                \cdots
            \end{tikzcd}
        \end{center}
        Now suppose $F: \omega\op \to \textbf{Set}$ is a functor. 
        Then if we write $F(i) = A_i$ with $A_i \in \textbf{Set}$, 
        then we see that the image of $F$ is a family of sets 
        $F_n$ with functions $f_{n}:A_{n+1} \to A_{n}$: 
        \begin{center}
            \begin{tikzcd}
                A_0
                &
                A_1 \arrow[l,swap, "f_0"]
                &
                A_2 \arrow[l,swap, "f_1"]
                &
                A_3 \arrow[l,swap, "f_2"]
                &
                \arrow[l, swap, "f_3"]
                \cdots
            \end{tikzcd}
        \end{center}
        One way we could try forming a limit of this diagram is by
        constructing a cone, using the product of these sets. 
        \begin{center}
            \begin{tikzcd}[row sep = 1.6cm]
                A_0
                &
                A_1 \arrow[l,swap, "f_0"]
                &
                A_2 \arrow[l,swap, "f_1"]
                &
                A_3 \arrow[l,swap, "f_2"]
                &
                \arrow[l, swap, "f_3"]
                \cdots
                \\
                \prod\limits_{i = 0}^{\infty}A_i 
                \arrow[u, "\pi_0"]
                \arrow[ur, "\pi_1"]
                \arrow[urr, "\pi_2"]
                \arrow[urrr, swap, "\pi_3"]
                & & & & 
            \end{tikzcd}
        \end{center}
        However, this isn't exactly what we want. A cone must
        form a commutative diagram and it's not always true
        that $f_n \circ \pi_{n+1} = \pi_n$. So let's instead restrict 
        our attention to a subset $\displaystyle L \subset \prod_{i = 0}A_i$ 
        where the points $(a_0, a_1, \dots, a_n, \dots)$ do satisfy this relation.
        \[
            L = \big\{x = (a_0, a_1, a_2, \dots,) 
            \mid f_{n}\circ \pi_{n+1}(a_) = \pi_n(x) \big\}.
        \]
        and equip $L$ with the functions $\pi'_{n}$ where 
        \[
            \textcolor{NavyBlue}{\pi'_n} = \pi_n \circ i  : L \to A_n
        \]
        where $i: L \to \prod\limits_{i = 1}^{\infty} F_i$ is the
        inclusion function. Then we have
        \begin{center}
            \begin{tikzcd}[row sep = 1.6cm]
                A_0
                &
                A_1 \arrow[l,swap, "f_0"]
                &
                A_2 \arrow[l,swap, "f_1"]
                &
                A_3 \arrow[l,swap, "f_2"]
                &
                \arrow[l, swap, "f_3"]
                \cdots
                \\
                L
                \arrow[u, NavyBlue, "\pi'_0"]
                \arrow[ur, NavyBlue, "\pi'_1"]
                \arrow[urr, NavyBlue, "\pi'_2"]
                \arrow[urrr, NavyBlue, swap, "\pi'_3"]
                & & & & 
            \end{tikzcd}
        \end{center}
        so that $L$ forms a cone. We now prove that this cone is universal. 

    \begin{lemma}
        The set $L$ is the limit of the functor $F: \omega\op \to \textbf{Set}$.
    \end{lemma}

    \begin{prf}
        Suppose $K$ is another cone over our diagram, 
        equipped with morphisms $\textcolor{Red}{\mu_n}: K \to
        F_n$. Since this is another 
        cone, we have that $f_n \circ \mu_{n+1} = \mu_n$. 
        Now let $k \in K$.  Then we can form an element 
        \[
            x = (\mu_0(k), \mu_1(k), \mu_2(k), \dots) \in 
            \prod_{i = 1}^{\infty}F_i
        \]  
        since each $\mu_n(k) \in F_n$. Now observe that 
        \[
            f_n \circ \pi_{n+1}(x) = f_n(\mu_{n+1}(k)) = 
            \mu_n(k) = \pi_n(x). 
        \]
        Thus we see that $f_n \circ \pi_{n+1}(x) = \pi_n(x)$, so that
        by definition,  $x \in L$. Hence we can create a unique
        function $g: K \to L$ where for each $k \in K$,
        \[
            g(k) = (\mu_0(k), \mu_1(k), \mu_2(k), \dots)
        \]
        so we then have that 
        \[
            \pi'_n \circ g = \mu_n.
        \]
        Hence, this shows that $(L, \pi_n: L \to F_n)$ is universal,
        so that $L = \Lim F$! 
        \begin{center}
            \begin{tikzcd}[row sep = 1.6cm]
                F_0
                &
                F_1 \arrow[l,swap, "f_0"]
                &
                F_2 \arrow[l,swap, "f_1"]
                &
                F_3 \arrow[l,swap, "f_2"]
                &
                \arrow[l, swap, "f_3"]
                \cdots
                \\
                L = \Lim F
                \arrow[u, NavyBlue]
                \arrow[ur, NavyBlue]
                \arrow[urr, NavyBlue]
                \arrow[urrr, NavyBlue]
                & & & 
                K \arrow[lll, dashed, "g"]
                \arrow[u, Red]
                \arrow[ul, crossing over, Red]
                \arrow[ull, crossing over, Red]
                \arrow[ulll, crossing over, Red]
                & 
            \end{tikzcd}
        \end{center}

    \end{prf}
        
    If we want to view this in terms of the spider diagrams, then we
    have 
    \begin{center}
        \begin{tikzcd}[column sep =  1.4cm, row sep = 1.4cm]
            & & K \arrow[d, dashed, "g"] 
            \arrow[ddll, bend right,Red,  swap, "\mu_0"]
            \arrow[ddl, bend right, Red, crossing over, swap, "\mu_1"]
            \arrow[ddr, bend left, Red, crossing over, "\mu_2"]
            \arrow[ddrr, bend left, Red, "\mu_3"]
            & & & \\
            & & \textcolor{NavyBlue}{L = \Lim F}
            \arrow[dll, NavyBlue, near start, swap, "\pi'_0"]
            \arrow[dl, NavyBlue, "\pi'_1"] 
            \arrow[dr, NavyBlue, swap, "\pi'_2"]
            \arrow[drr, NavyBlue, near start, "\pi'_3"] & & & \\
            F_0 & \arrow[l, "f_0"] F_1  
            &  & \arrow[ll, "f_1"] F_2
            & \arrow[l, "f_2"] F_3 
            & \arrow[l, "f_3"] \cdots
        \end{tikzcd}
    \end{center}
    
    \end{example}

    Here, we've taken a nice, simple diagram 
    $F: \omega\op \to \textbf{Set}$ and
    shown that there exists a limit $L$ of the diagram 
    inside of $\textbf{Set}$. However, we can do
    this more generally, so that \textbf{Set} is complete.
    To  illustrate this we need the notion of a set of cones. 

    Note that in the last example, we can
    actually think of each $x = (x_0, x_1, x_2, \dots)
    \in \Lim F$ as a cone. How so? 
    \begin{itemize}
        \item[1.] For each $x = (x_0, x_1, x_2, \dots) \in \Lim F$, 
        consider the one-point set $\{*\}$.
        
        \item[2.] Associate $\{*\}$ with 
        the family of functions $\pi^*_n: \{*\} \to F_n$, defined as
        \[
            \pi^x_n(*) = x_n.
        \] 
    \end{itemize}
    Now since $x \in \Lim F$, we know that $f_n(x_{n+1}) = x_{n}$.
    But, note that this is equivalent to stating that $f_n \circ
    \pi_{n+1}(*) = \pi_{n-1}(*)$. Therefore the diagram 
    \begin{center}
        \begin{tikzcd}[column sep =  1.4cm, row sep = 1.4cm]
            & \{*\} \arrow[dl, swap, "\pi^x_n"] 
            \arrow[dr, "\pi^x_{n+1}"]& \\
            F_n
            & & F_{n+1} \arrow[ll, "f_n"]
        \end{tikzcd} 
    \end{center}
    commutes for every $f_n: F_{n+1} \to F_n$, so that's how we can
    regard every $x \in \Lim F$ as a cone.
    Therefore, if we denote $\cone(*, F)$ as the set of all cones of
    $\{*\}$ over $F$, we see that $\cone(*, F) = \Lim F$. 
    \begin{thm}
        The category \textbf{Set} is complete. That is, if $J$ is a
        small category, every functor $F: J \to \textbf{Set}$ has a 
        limit 
        \[
            \Lim F = \cone(*, F)
        \]
        where $\cone(*, F)$ is the set of all cones of $\{*\}$ over
        $F$. The set $\cone(*, F)$ forms the limit cone with the
        morphisms $v_i: \cone(*, F) \to F_i$ described as follows.
        If $x \in \cone(*, F)$, then $x$ has a family of 
        morphisms $\sigma^x_i: \{*\} \to F_i$. Therefore,
        \[
            v_i: \cone(*, F) \to F_i \qquad v_i(x) = \sigma^x_i(*).
        \]
        \vspace{-0.8cm}
    \end{thm}

    \begin{prf}
        First, since $J$ is small, we know that $\cone(*, F)$ is a 
        set. For each $j \in J$, establish the morphism $v_j: \cone(*,
        F) \to F_j$ where $v_j(x) = \sigma^x_j(x)$, and $\sigma^x_j: \{*\}
        \to F_j$ is the morphism associated with $x$ as a cone over $F$. 

        We now show that it is a cone. 
        Suppose $f: i \to j$ is a morphism in $J$. Then observe that 
        $F(f) \circ v_i(x) = F(f) \circ \sigma^x_i(x) = \sigma^x_j(x)
        = v_j(x)$. Hence the diagram 
        \begin{center}
            \begin{tikzcd}[column sep =  1.4cm, row sep = 1.4cm]
                & \cone(*, F) \arrow[dl, swap, "v_i"] 
                \arrow[dr, "v_{j}"]& \\
                F_i \arrow[rr, swap, "F(f)"]
                & & F_j 
            \end{tikzcd} 
        \end{center}
        commutes, so $\cone(*, F)$ really does form a cone over $F$. 
        To show this is universal, and hence our limit, suppose that 
        $A$ in \textbf{Set} also forms a cone over $F$ with morphisms 
        $\tau_j: X \to F_j$. Note that for each $a \in A$, we can form
        a cone from $\{*\}$ to $F$, if we define $\sigma^a_j: \{*\} 
        \to F_j$ as $\sigma^a_j(*) = \tau_j(a)$. Then the diagram 
        \begin{center}
            \begin{tikzcd}[column sep =  1.4cm, row sep = 1.4cm]
                & \cone(*, F) \arrow[dl, swap, "\sigma^a_i"] 
                \arrow[dr, "\sigma^a_j"]& \\
                F_i \arrow[rr, swap, "F(f)"]
                & & F_j 
            \end{tikzcd} 
        \end{center}
        must also commutes since it commutes for each $\tau_j$. 
        Thus we can define a unique 
        function $g: A \to \cone(*, F)$, where each point $a$ is sent to
        the cone which it forms from $\{*\}$ over $F$. Therefore,
        $\cone(*, F)$ is universal, so that 
        \[
            \Lim F = \cone(*, F)
        \]
        as desired. 
    \end{prf}

    The above proof can be repeated to show that others categories are
    complete, like \textbf{Grp} or \textbf{Rng}. 

    In attempting to find the limit $F: J \to \cc$ in some category
    $\cc$, one strategy is to to compose this functor with another one
    $G: \cc \to \dd$, with the prior knowledge that $\dd$ is complete.
    If one knows $\dd$ is complete, one then use this information to find 
    the limit of $F: J \to \cc$. 
    
    \begin{definition}
        Let $F: J \to \cc$ be a functor. A functor $G: \cc \to \dd$ 
        \textbf{creates limits for $F$} if whenever $(\Lim G \circ F, \tau: \Delta(\Lim G \circ F) \to G \circ F)$
        exists, 
        the limit $(\Lim F, \sigma: \Delta(\Lim F) \to F)$
        such that 
        \[
            G(\Lim F) = \Lim G \circ F \qquad G(\sigma) = \tau.
        \]
        Similarly, a functor $G: \cc \to \dd$ \textbf{creates colimits for $F$}
        if whenever $(\Colim G \circ F, \tau: G \circ F \to \Delta(\Lim G \circ F)$ 
        exists, the colimit $(\Colim F, \sigma: F \to \Delta(\Colim F)$ exists 
        and 
        \[
            G(\Colim F) = \Colim G \circ F \qquad G(\sigma) = \tau.
        \]
    \end{definition}

    The diagram below visually explains this process; the existence of limit 
    in $\dd$ on the left implies the existence of the limit in $\cc$ on the right. 
    Moreover, the diagram on the left is the image of the diagram on the right 
    under $G$. 
    \begin{center}
        \begin{tikzcd}[column sep = 1cm, row sep = 1.4cm]
            & D \arrow[d, dashed, "h"] \arrow[ddl,swap, bend right = 20,
            "f_i"] \arrow[ddr, bend left = 20, "f_j"] &\\
            & \arrow[dl, swap, "\tau_i"] \Lim G \circ F 
            \arrow[dr, "\tau_j"] & \\
            G(F(i))
            \arrow[rr, swap, "G(F(u))"] 
            & 
            &
            G(F(j))
        \end{tikzcd}
        $\implies$
        \begin{tikzcd}[column sep = 1.4cm, row sep = 1.4cm]
            & C \arrow[d, dashed, "h"] \arrow[ddl,swap, bend right = 20,
            "f_i"] \arrow[ddr, bend left = 20, "f_j"] &\\
            & \arrow[dl, swap, "\sigma_i"] \Lim F 
            \arrow[dr, "\sigma_j"] & \\
            F(i)
            \arrow[rr, swap, "F(u)"] 
            & 
            &
            F(j) 
        \end{tikzcd}
    \end{center}
    
    \begin{example}
        Consider a functor $F: J \to \textbf{Grp}$. We'll show that
        the forgetful functor $U:\textbf{Grp} \to \textbf{Set}$
        creates limits for $\textbf{Grp}$.

        By the previous theorem, we know that $U \circ F; J \to
        \textbf{Set}$ must have a limit $\cone(*, U \circ F)$ 
        with the family of morphisms $v_i: \cone(*, U\circ F) \to
        U\circ F_i$. 
        Now denote the set $\cone(*,U \circ F)$ as $L$. 
        Then we can endow $L$ with a group structure.
        \begin{itemize}
            \item For any $\sigma, \tau \in L$, we define $\sigma
            \times \tau$ to be the cone where $(\sigma \times \tau)_i =
            \sigma_i \cdot \tau_i$, where $\cdot$ is the product in
            $F_i$. 

            \item For $\sigma \in L$, we define the inverse to be   
            the function $\sigma^{-1}$ where $(\sigma^{-1})_i =
            \sigma_i^{-1}$, with the inverse being taken in $F_i$. 
        \end{itemize}
        All we're really doing here is taking advantage of the fact 
        that each $\sigma, \tau$ is really just a family of functions 
        $\sigma_i, \tau_i: \{*\} \to F_i$. Thus we're taking advantage
        of the group structure in each $F_i$. 

        Thus $L = \cone(*, U \circ F)$ is a group, which then makes
        the family of morphisms $v_i: \cone(*, U \circ F)$ into a
        family of group homomorphisms. To show this, simply observe
        that 
        \[
            v_i(\sigma \times \tau) = (\sigma \times \tau)_i = \sigma_i \cdot \tau_i = v_i(\sigma)\cdot v_i(\tau).           
        \]
        Now we claim that the cone $\cone(*, U \circ F)$ with the
        morphisms $v_i: \cone(*, U \circ F) \to F_i$ is universal. To
        show this, let $G$ be a group and suppose $G$ forms a cone over $F$ 
        with morphisms $\phi_i: G \to F_i$. Then $U(G)$ forms a cone
        over $\cone(*, U \circ F)$ in \textbf{Set} with morphisms 
        $U(\phi_i): U(G) \to U(F_i)$. 

        Since we know $\cone(*, U \circ F)$ is a universal cone in \textbf{Set}, 
        there exists a $h: U(G) \to L$ such that $U(\phi_i) =
        U(v_i)\circ h_i$. However, note that $h$ can be thought of as a 
        group homomorphism. For any $g, g' \in G$, we have
        \begin{align*}
            h_i(g \cdot g') = \phi_i(g \cdot g') 
            = \phi_i(g) \times \phi_i(g')
            &= h_i(g) \times h_i(g')\\
            &= (h(g) \cdot h(g'))_i.
        \end{align*}
        Therefore, $h:U(G) \to L$ can be realized back into \textbf{Grp}
        as a group homomorphism $h: G \to L$, thereby showing $\cone(*, U \circ F)$ 
        is a universal cone in \textbf{Grp}. This is one way in showing 
        \textbf{Grp} is complete.
    \end{example}
    What we really did in the last example was nothing special. Using the fact that 
    \textbf{Set} is complete, we transferred $F: J \to \textbf{Grp}$
    over to \textbf{Set} via the forgetful functor $U: \textbf{Grp} 
    \to \textbf{Set}$. We calculated the limit, and showed that this can
    be realized as a limit in \textbf{Grp}. In this sense, $U: \textbf{Grp} \to 
    \textbf{Set}$ \textbf{creates limits} in \textbf{Grp}. A similar
    strategy can be carried out for other forgetful functors. 

    \begin{example}
        Let $\cc$ be a category and $A$ an object of $\cc$. Recall that 
        with the comma category $(A \downarrow \cc)$, we have a projection 
        functor $P: (A \downarrow \cc) \to \cc$ where on objects 
        $(C, f: A \to C)$ and morphisms $h: (C, f: A \to C) \to (C', f: A \to C')$ 
        we have that 
        \[
            P(C, f: A \to C) = C \qquad P(h) = h: C \to C'.            
        \]
        Now for any functor $F:J \to (A \downarrow \cc)$, the functor 
        $P: (A \downarrow \cc) \to \cc$ creates limits. To see this, we first interpret 
        a functor $F: J \to (A \downarrow \cc)$. For each $j$, we have that  
        \[
            F(j) = (C_j , f_j: A \to C_j) 
        \]
        for some $C_j \in \cc$ and $f_j: A \to C_j$. If $u: j \to k$ is a morphism in
        $J$, then $F(u) : C_j \to C_k$ is a morphism in $\cc$ 
        such that the diagram below commutes (as, that's what morphisms  
        do in comma categories). 
        \begin{center}
            \begin{tikzcd}[column sep = 1.4cm, row sep = 1.4cm]
                & A 
                \arrow[dr, "f_k"]
                \arrow[dl, swap, "f_j"]
                &
                \\
                C_j
                \arrow[rr, swap, "F(u)"]
                &
                &
                C_k
            \end{tikzcd}
        \end{center}
        Note that this is a cone over $F$ in $\cc$. 
        Now suppose we have a limit $\Lim P \circ F$ in $\cc$ with 
        morphisms $\mu_i: \Lim P \circ F \to C_i$ with $i \in J$. 
        Then because $\Lim P \circ F$ is a limiting cone, and we must 
        have a unique $v$ such that the diagram below commutes. 
        \begin{center}
            \begin{tikzcd}[column sep = 1.4cm, row sep = 1.4cm]
                & A \arrow[d, dashed, "v"] 
                \arrow[ddl,swap, bend right = 20,"f_i"] 
                \arrow[ddr, bend left = 20, "f_j"] &\\
                & \arrow[dl, swap, "\mu_i"] \Lim P \circ F 
                \arrow[dr, "\mu_j"] & \\
                F(i)
                \arrow[rr, swap, "F(u)"] 
                & 
                &
                F(j) 
            \end{tikzcd}
        \end{center}
        The claim now is that $(\Lim P \circ F, v: A \to  \Lim P \circ F)$ 
        is the limit $\Lim F$ of $F: J \to (A \downarrow \cc)$, which is left 
        for the reader to show.
    \end{example}

    {\large \textbf{Exercises}
    \vspace{0.2cm}}
    \begin{itemize}
        \item[\textbf{1.}] 
        \begin{itemize}
            \item[\emph{i.}]
            Let $J = \omega$, and let $F: J \to \textbf{Set}$ be a functor 
            were $F(i) = A_i$. Show that $\Colim F$ exists and give an expicit 
            description of it. 
            \\
            \emph{Hint}: It will be a set endowed with an equivalence relation.
            
            \item[\emph{ii.}] How does your answer chance when $F: J \to \textbf{Set}$ 
            is contravariant? 
        \end{itemize}
        

    \end{itemize}
    
    \newpage
    \section{Inverse and Direct Limits.}
    In the previous example, we calculated the limit of the diagram
    indexed by $\omega\op$. It turns out that in general, we can construct 
    a lot of mathematical ideas by first modeling them as the limit 
    of a functor $F: J \to \cc$, where $J$ is a partially ordered set.
    Thus we give a special name to this concept.
    
    \begin{definition}
        Let $\cc$ be a category, and suppose the $F: J\op \to
        \cc$ has 
        a limit object $\Lim F$ in $\cc$, where $J$ is a partially
        ordered set (where, if $i \le j$, then there exists $f: i \to
        j$). 
        Then $\Lim F$ is
        said to be a \textbf{inverse limit} or \textbf{projective limit}. 
        
        Dually, we define the colimit of a functor $F: J \to F$ 
        to be \textbf{direct limit}. 
    \end{definition}

    There are many famous examples of these limits, with the
    following example probably being the most familiar. 

    \begin{example}
        Consider the functor $F: \omega\op \to \textbf{Rng}$ where 
        we define $F(n) = F_n = \zz/p^n\zz$ with $p$ being a prime. 
        Then we have a diagram 
        \begin{center}
            \begin{tikzcd}[column sep =  1.4cm, row sep = 1.4cm]
                \zz 
                & 
                \zz/p\zz \arrow[l, swap, "f_0"]
                &
                \zz/p^2\zz \arrow[l, swap, "f_1"]
                & 
                \arrow[l, swap, "f_2"]
                \zz/p^3\zz
                &
                \arrow[l, swap, "f_3"]
                \cdots
            \end{tikzcd}
        \end{center}
        where the maps $f_n: \zz/p^{n+1}\zz \to \zz/p^n\zz$ are the 
        projection maps. The limit of this diagram turns out to be the  
        \textbf{$\bm{p}$-adic integers} $\zz_p$, and this is one way of
        defining them. The most popular way to define them it to work
        in ring theory, establish $p$-adic valuations, and realize
        that the 
        valuations turn $\zz$ into a metric space; one which can be 
        completed with respect to the metric to give rise to $\zz_p$.

        First, observe that they form a cone. Define the map 
        \[
            \pi_{n}: \zz_p \to \zz/p^{n}\zz 
            \qquad \pi\left( \sum_{k = 0}^{\infty}
            a_kp^k \right) = 
            \sum_{k =0}^{n-1}a_kp^k + p^{n}\zz.
        \]
        Now observe that 
        \begin{align*}
            f_n \circ \pi_{n+1} \left( \sum_{k = 0}^{\infty}a_kp^k \right)
            = f_n\left( \sum_{k =0}^{n}a_kp^k + p^{n+1}\zz \right)
            &= \sum_{k =0}^{n-1}a_kp^k + p^{n}\zz\\
            &= \pi_n\left(  \sum_{k = 0}^{\infty}a_kp^k \right)
        \end{align*}
        so we may conclude that $f_n \circ \pi{n+1} = \pi_n$.
        Therefore, $\zz_p$ does in fact form a cone with the 
        morphisms $\pi_n$, so the following diagram commutes.
        \begin{center}
            \begin{tikzcd}[column sep =  1.4cm, row sep = 1.6cm]
                \zz_p \arrow[d, swap, "\pi_0"] 
                \arrow[dr, swap,  "\pi_1"]
                \arrow[drr, swap, "\pi_2"]
                \arrow[drrr, "\pi_3"]
                & & & & \\
                \zz
                &
                \zz/p\zz \arrow[l,swap, "f_0"]
                &
                \zz/p^2\zz \arrow[l,swap, "f_1"]
                &
                \zz/p^3\zz \arrow[l, swap, "f_2"]
                &
                \arrow[l, swap, "f_3"]
                \cdots
            \end{tikzcd}
        \end{center}
        Showing this is universal is simple once we realize that each
        element of $\zz_p$ may be thought of as a cone, in the same
        fashion as we did with \textbf{Set}. That is, we can just 
        apply the previous theorem to \textbf{Rng}. 
        This then shows that it's the universal object which we desire.
    \end{example}

    What about direct limits? A less-talked about idea
    , although definitely not less interesting, is the 
    dual of the above construction.

    \begin{example}
        Consider the functor $F: \omega \to \textbf{Grp}$ where we have 
        $F(n) = F_n = \zz/p^n\zz$, with $p$ being a prime. This time however 
        we have the diagram 
        \begin{center}
            \begin{tikzcd}[column sep =  1.4cm, row sep = 1.4cm]
                \zz 
                & 
                \zz/p\zz \arrow[<-, l, swap, "f_0"]
                &
                \zz/p^2\zz \arrow[<-, l, swap, "f_1"]
                & 
                \arrow[<-, l, swap, "f_2"]
                \zz/p^3\zz
                &
                \arrow[<-, l, swap, "f_3"]
                \cdots
            \end{tikzcd}
        \end{center}
        where we define each $f_n: \zz/p^n\zz \to \zz/p^{n+1}\zz$ 
        as the homomorphism 
        \[
            f_n\left( \sum_{k =0}^{n-1}a_kp^k + p^{n}\zz \right)
            = 
            \sum_{k =0}^{n}a_kp^{k+1} + p^{n+1}\zz.
        \]
        That is, we simply multiply the sum by a power of $p$. It turn outs 
        that the direct limit is the \textbf{Prüfer $\bm{p}$-Group} $\zz(p^\infty)$.
        The Prüfer 2-Group is pictured below.
        \begin{center}
            \includegraphics{../pictures/chapter5_pic/prufer_2_group/prufer_2_group.pdf}
        \end{center}
        
        The Prüfer $p$-group is the set of all $p^n$ roots of unity,
        as $n$ ranges over all positive integers. Hence the points lie
        on the complex unit circle. Specifically, it is the group  
        \[
            \zz(p^{\infty})
            =
            \left\{\text{exp}\left({\dfrac{2\pi i m}{p^n}}\right) \mid 0 \le m < p^n, n \in \mathbb{Z}^{+}  \right\}  
        \]
        which forms a group under complex multiplication. 
        How does this form a limit for our diagram? 
 
        
        

    \end{example}       

    Inverse limits are also used in Galois Theory. In Galois Theory,
    one can define a field extension $L/F$ to be a finite, normal, separable extension. 
    However, it turns out that one can remove the requirement for  
    the extension to be finite. We then obtain infinite Galois groups,
    which are constructed as follows. 

    \begin{example}
        Let $F$ be a field, and suppose $L/F$ is normal, separable
        extension (\textcolor{Red}{not necessarily finite!}).
        Then we can define $L/F$ to be a Galois extension,
        and we may speak of a Galois group $\gal(L/F)$, as follows.

        Let $\mathcal{F}(L/F)$ be the category of all
        finite, normal extensions $K$ of $F$ such that $F \subset K
        \subset L$, and $\mathcal{G}(L/F)$ is the category of all
        their Galois groups. Note that both $\mathcal{F}(L/F)$ and 
        $\mathcal{G}(L/F)$ are partially ordered sets, ordered by subset inclusion. 
        To be precise, if $K_i \subset K_j$ are in $\mathcal{F}(L/F)$,
        then 
        \[
            \gal(K_j/F) \subset \gal(K_i/F)
        \]
        and because $\mathcal{G}(L/F)$ is a preorder on subset
        inclusion, this implies the existence of some arrow $f:
        \gal(K_j/F) \to \gal(K_i/F)$. We can describe 
        $f = \proj_{K_j/K_i}$ where 
        \[
            \proj_{K_j/K_i}: \gal(K_j/F) \to \gal(K_i/F) \qquad \proj_{K_j/K_i}(\sigma) = \sigma\big|_{K_i}.  
        \]
        That is, we take each permutation $\sigma \in \gal(K_j/F)$
        and restrict its action to $K_i$, thereby making it a
        permutation of $K_i$ which fixes $F$, 
        and therefore a member of $\gal(K_i/F)$. 

        Now consider the product with the associated morphisms
        \[  
            \prod_{K \in\mathcal{F}(L/F)}\gal(K/F)
            \qquad 
            \pi_{K_i}: \prod_{K \in\mathcal{F}(L/F)}\gal(K/F) 
            \to \gal(K_i/F)
        \]
        Then we define
        \[
            \gal(L/F)
            =
            \left\{ x = (\cdots, \sigma_k, \cdots) \in \prod_{K \in \mathcal{F}(L/F)}\gal(K/F) 
            \mid \proj_{K_i/K_j} \circ \pi_{K_i}(x) = \pi_{K_j}(x) \right\}.
        \]
        So $\gal(L/F)$ forms a cone with morphisms $\pi_{K_i}$:
        \begin{center}
            \begin{tikzcd}[column sep =  1.4cm, row sep = 1.6cm]
            & \gal(L/F) \arrow[dl, swap, "\pi_{K_i}"] \arrow[dr, "\pi_{K_j}"] & \\
            \gal(K_i/F) \arrow[rr, swap, "\proj_{K_i/K_j}"] & & 
            \gal(K_j/F)
            \end{tikzcd}
        \end{center}
        We then have to work to show that this cone is universal.
        However, the faster route is to simply recognize that we can 
        index $\mathcal{G}(L/F)$ in a monotonic way, since it is a partially order set.
        Thus there exists a partially ordered set $J$ such that 
        if $f: i \to j$ exists in $J$, then 
        \[
            F(i)= \gal(K_i/F) \quad F(j) = \gal(K_j/F) 
            \implies 
            F(f): \gal(K_i/F) \to \gal(K_j/F)  
        \]
        Thus we have a functor $F: J \to \mathcal{G}(L/F)$ which hits every 
        Galois group $\gal(K/F)$ in such
        a way that it preserves the order in $\mathcal{G}(L/F)$. Since
        the limit of every small diagram exists in \textbf{Grp},
        we can define $\gal(L/F)$ to be the \textbf{inverse limit} of this
        functor, and we already know that the limit will have the form
        \[
            \gal(L/F)
            =
            \left\{ (\cdots, \sigma_k, \cdots) \in \prod_{K \in \mathcal{F}(L/F)}\gal(K/F) \mid \proj_{K_i/K_j} \circ \pi_{K_i} = \pi_{K_j} \right\}.
        \]
        and that it will be universal. So, this is how we extend the
        definition of Galois group from a finite, normal, separable
        extension to simple a normal, separable extension.
    \end{example}

    \noindent This construction can be done more generally on a partially
    ordered system of groups, to create these things called 
    \textbf{profinite groups}. 

    \begin{definition}
        Suppose we
        are given a partially ordered set of finite groups $G_i$, indexed by some
        set $I$, equipped with morphisms $\{f^j_i: G_j \to G_i \mid i,
        j \in I \quad i \le j\}$ such that 
        \begin{itemize}
            \item[1.] $f_i^i: G_i \to G_i$ is the identity $\id_{G_i}$ 
            \item[2.] $f_i^j \circ f_j^k = f_i^k$. 
        \end{itemize}
        Then we define the \textbf{profinite group} $G$ 
        of this system to be the inverse limit:
        \[
            G = \left\{(g_i)_{i \in I} \in \prod_{i \in I} G_i \
            \mid f_i^j(g_i) = g_j \right\}.
        \]
        Note that requiring $f_i^j(g_i) = g_j$ is the same as
        requiring $f_i^j\circ \pi_i(x) = \pi_j(x)$, where $x \in G$, 
        which is how we defined $\gal(L/F)$. 
    \end{definition}
    \textcolor{MidnightBlue}{Thus in the previous example, we have
    that not only can we actually define $\gal(L/F)$, but our construction 
    leads to it to becoming a profinite group. Profinite groups are actually very 
    special, in that they can be interpreted topologically.}

    


    \newpage
    \section{Limits from Products, Equalizers, and Pullbacks.}
    In our construction of limits for \textbf{Sets}, we basically forced 
    the existence of a cone, because we could. This is usually the general strategy 
    when it comes to calculating the limit of a diagram in a given category; 
    one uses available, useful constructions which are already present 
    inside of a category. For example; in \textbf{Set}, we used 
    the fact that it is cartesian closed to formulate infinite products.

    Since the general strategy for showing \textbf{Set} is complete can 
    be extended to other categories, one may wonder "well, why? And when 
    will I no longer be able to apply this strategy?" The theorem 
    below answers this question. 
    
    \begin{thm}\label{products_equalizers_all_limits}
        Let $\cc$ be a category and $J$ a small category. 
        Suppose $\cc$ has equalizers for every
        pair of morphisms in $\cc$, and all products indexed by
        objects of $J$ and morphisms of $J$. Then every functor 
        $F: J \to \cc$ has a limit in $\cc$. 
    \end{thm}

    \textcolor{purple}{What do we mean by all products "indexed by 
    objects of $J$ and morphisms of $J$"?} What we want to do is be able 
    to \textit{create} products of the form 
    \[
        \prod_{j \in J}F_j \qquad\qquad \prod_{u:i \to k} F_\text{\text{cod}(u)} = \prod_{u:j \to k}F_k.   
    \]
    and \textit{know} that they're in $\cc$. 
    The product on the far left is indexed by objects of $J$, while the equal  
    ones on the right are indexed by morphisms $u: i \to k$ in $J$. It's a
    bit weird to think of a product "indexed by morphisms," but it's   
    exactly what it sounds like: we index over all the morphisms, and 
    take the product of the domain or codomain (in the above, we did codomain).
    
    \textcolor{purple}{Why do we need this weird concept?} To 
    answer this, let's go over the construction of limits in \textbf{Set}
    in a bit different way. 

    When we had a diagram $F: J \to \cc$ in $\cc$, our first guess 
    in constructing the limit was designing the $\displaystyle \prod_{j}F_j$
    with morphisms $\displaystyle \pi_i : \prod_{j}F_j \to F_i$. However, 
    this doesn't actually form a cone, since for each $u: j \to k$,
    we can't guarantee 
    \[
        F(u) \circ \pi_j = \pi_k
    \]
    That is, we can't guarantee the diagram 
    \begin{center}
        \begin{tikzcd}[column sep =  1.4cm, row sep = 1.4cm]
            & \displaystyle \prod\limits_{j \in J}F_j \arrow[dr, "\pi_k"] \arrow[dl, swap, "\pi_j"]
            &\\
            F_k \arrow[rr, swap, "u"] & & F_k
        \end{tikzcd}
    \end{center}
    will commute, which is what we need for a cone. Since we needed 
    $F(u) \circ \pi_j = \pi_k$, we forced it. But this forcing is 
    simply realizing that, all $\displaystyle x \in \prod_{j \in J}F_j$ 
    which satisfy $F(u) \circ \pi_j = \pi_k$, are simply 
    members of the equalizer of $F(u) \circ \pi_j$ and $\pi_k$. 

    \begin{prf}
        Consider the products $\displaystyle \prod_{j \in J}F_j$ and 
        $\displaystyle \prod_{u: i \to k}F_k$ where in the last product we 
        index over all morphisms in $J$. 
        With both products, consider the projection morphisms
        \begin{align*}
            &\pi'_j: \prod_{u:i \to k}F_k \to F_j\\
            &\pi_j: \prod_{i \in J}F_i \to F_j.
        \end{align*}
        Note that because we have products, we have universal
        properties which we can take advantage of. That is, 
        the following diagrams must commute for some $f$ and $g$. 
        \begin{center}
            \begin{tikzcd}[column sep =  1.4cm, row sep = 1.4cm]
                & \displaystyle \prod_{i \in J} F_i 
                \arrow[dl, swap, "\pi_k"] \arrow[d, dashed, Red, "f"]\\
                F_k & \displaystyle \prod_{u:j \to k} F_k \arrow[l,swap, "\pi'_k"]
            \end{tikzcd}
            \hspace{1cm}
            \begin{tikzcd}[column sep =  1.4cm, row sep = 1.4cm]
                \displaystyle \prod_{i \in J}F_i \arrow[r, Blue, dashed, "g"]
                \arrow[d, swap, "\pi_i"] 
                &
                \displaystyle \prod_{u:i \to k}F_k 
                \arrow[d, "\pi'_k"]\\
                F_i \arrow[r, swap, "F(u)"] 
                &
                F_k
            \end{tikzcd}
        \end{center}
        Note however that we can stack these diagrams on top of each other, to obtain  
        \begin{center}
            \begin{tikzcd}[column sep =  1cm, row sep = 1.4cm]
                & F_k & \\
                \displaystyle \prod_{i \in J}F_i 
                \arrow[rr,shift right =0.6ex, Red, swap, "f"]
                \arrow[rr,shift right =-0.6ex, Blue, "g"]
                \arrow[ur, "\pi_k"] 
                \arrow[d, swap, "\pi_i"] 
                &
                &
                \displaystyle \prod_{u:i \to k}F_k 
                \arrow[d, "\pi'_k"]
                \arrow[ul, swap, "\pi'_k"]
                \\
                F_i \arrow[rr, swap, "F(u)"] 
                &
                &
                F_k
            \end{tikzcd}
        \end{center}
        Since we have equalizers for every pair of arrows, we can form
        the equalizer $\displaystyle e:D \to \prod_{i \in J}F_i$ 
        of both $f$ and $g$ for some object $D$.
        \begin{center}
            \begin{tikzcd}[column sep =  1.4cm, row sep = 1.4cm]
                D \arrow[r, dashed, "e"]
                &
                \displaystyle \prod_{i \in J}F_i 
                \arrow[r,shift right =0.6ex, Red, swap, "f"]
                \arrow[r,shift right =-0.6ex, Blue, "g"]
                &
                \displaystyle \prod_{u:i \to k}F_k 
            \end{tikzcd}
        \end{center}   
        Now that we have a morphism 
        $\displaystyle e: D \to \prod_{i \in J}F_i$,  
        we can compose this with projections $\displaystyle 
        \prod_{i \in J}F_i \to F_i$ to produce a family of
        morphisms $\pi_i \circ e: D \to F_i$. If we like, we can even 
        add this to our diagram above to get the following:
        \begin{center}
            \begin{tikzcd}[column sep =  1cm, row sep = 1.4cm]
                & & F_k & \\
                D \arrow[r, dashed, "e"] \arrow[dr,swap, dashed,
                "\mu_i"]
                &
                \displaystyle \prod_{i \in J}F_i 
                \arrow[rr,shift right =0.6ex, Red, swap, "f"]
                \arrow[rr,shift right =-0.6ex, Blue, "g"]
                \arrow[ur, "\pi_k"] 
                \arrow[d, swap, "\pi_i"] 
                &
                &
                \displaystyle \prod_{u:i \to k}F_k 
                \arrow[d, "\pi'_k"]
                \arrow[ul, swap, "\pi'_k"]
                \\
                &
                F_i \arrow[rr, swap, "F(u)"] 
                &
                &
                F_k
            \end{tikzcd}
        \end{center}
        (\textcolor{Red}{It looks like a boat!}) Denote $\mu_i = \pi_i \circ e: D \to F_i$. Then what the above boat diagram 
        tells us is that 
        \[
            \pi'_k \circ g = \pi_k \qquad F(u)\circ \pi_i = \pi'_k \circ f.
        \]
        Composing both equations with $e$, we get 
        \[
            \pi'_k \circ g \circ e = \pi_k \circ e \qquad F(u)\circ \pi_i \circ e= \pi'_k \circ f\circ e.
        \]
        but since $g \circ e = f \circ e$, what this really tells us is that
        \[
            F(u) \circ \pi_i \circ e = \pi_k \circ e \implies F(u) \circ \mu_i = \mu_k.
        \]
        for every $u: i \to k$ in $J$. 
        Therefore, we see that we have that 
        \begin{center}
            \begin{tikzcd}[column sep =  1.4cm, row sep = 1.4cm]
                & D 
                \arrow[dl, swap, "\mu_i"]
                \arrow[dr, "\mu_j"]
                & \\
                F_i \arrow[rr, swap, "F(u)"] & & F_k
            \end{tikzcd}
        \end{center}
        commutes, so that $D$ equipped with the morphisms $\mu_i: D \to F_i$ forms a cone.
        We now show that this is universal, so that $D$ is our limit. We do this 
        by taking advantage of the universal property which equalizers
        posses. 
        
        Suppose $C$ is another object which forms a cone with 
        morphisms $\tau_i: C \to F_i$. Then there exists a map
        $\displaystyle e': C \to \prod_{i \in J}F_i$ such that 
        $\pi \circ e' = \tau_i$. Moreover, this implies that 
        $f \circ e = g \circ e$. But the universal property of 
        the equalizer $e$
        states that for any subject object, there exists a morphism 
        $h: D \to C$ such that the diagram below commutes. 
        \begin{center}
            \begin{tikzcd}[column sep =  1.4cm, row sep = 1.4cm]
                D \arrow[r, "e"] 
                \arrow[d, dashed, swap, "h"]
                &
                \displaystyle \prod_{i \in J}F_i 
                \arrow[r,shift right =0.6ex, Red, swap, "f"]
                \arrow[r,shift right =-0.6ex, Blue, "g"]
                \arrow[d, "\pi_i"]
                &
                \displaystyle \prod_{u:i \to k}F_k\\
                C 
                \arrow[ur, "e'"] 
                \arrow[r, "\tau_i"]
                &
                F_i 
                &
            \end{tikzcd}
        \end{center}
        Since $h: D \to C$ is unique, this shows that $D$ 
        equipped with the morphisms 
        $\displaystyle \mu_i: D \to F_i$ forms a limit of the diagram,
        so that $D = \Lim F$.
    \end{prf}
    We actually proved much more than what was stated in the theorem, 
    since we literally found the explicit form the limit. 

    As a corollary, we have the following result which is due to the
    above theorem. The only difference is we strengthen our
    hypothesis, which makes it less general. 

    \begin{corollary}
        Let $\cc$ be a category. If $\cc$ has all equalizers (coequalizers)
        and finite products (coproducts), then 
        $\cc$ has all finite limits (colimits). 
    \end{corollary}

    By Proposition \ref{prop_category_finite_products}, one can obtain finite products 
    by simply demanding the existence of binary products and a terminal object. Hence 
    we can restate the above corollary:

    \begin{corollary}
        Let $\cc$ be a category. If $\cc$ has all equalizers (coequalizers),
        binary products (coproducts) and a terminal object, then $\cc$ has 
        all finite limits.
    \end{corollary}

    Not what is even more interesting is that we can construct equalizers and 
    finite products from pullbacks. 

    Specifically, suppose our category $\cc$ has pullbacks and a terminal object $T$. 
    For any pair of objects $A, B$ in $\cc$, suppose we take the pull back on 
    the morphisms $t_A: A \to T$ and $t_B: B \to T$. This then give 
    rise to an object $P$ equipped with two morphisms $p_1: P \to A$ 
    and $p_2: P \to B$, universal in the sense demonstrated below. 
    \begin{center}
        \begin{tikzcd}[column sep =  1.4cm, row sep = 1.4cm]
            C \arrow[drr, bend left, "f"]
            \arrow[ddr, bend right, swap, "g"]
            \arrow[dr, dashed, "h"]
            &[-0.7cm]
            &
            \\[-1.2cm]
            &[-1cm]
            P
            \arrow[r, "p_1"]
            \arrow[d, swap, "p_2"]
            &
            A
            \arrow[d, "t_A"]\\
            &[-1cm]
            B 
            \arrow[r, swap, "t_B"] 
            &
            T 
        \end{tikzcd}
        $\implies$
        \begin{tikzcd}[column sep =  1.4cm, row sep = 1.4cm]
            &
            C
            \arrow[dr, "f"]
            \arrow[dl, swap, "g"]
            \arrow[d, dashed, "h"]
            &
            \\
            B
            &
            P
            \arrow[l, "p_2"]
            \arrow[r, swap, "p_1"]
            &
            A
        \end{tikzcd}
    \end{center}
    Now on the top left we have our pull back. However, on the top right, we've 
    unraveled the pullback and ignored the terminal object to observe that $P$ 
    has the universal property of what a product would demand. Hence we may denote 
    $P = A \times B$ as the product. Thus by Proposition \ref{prop_category_finite_products} 
    $\cc$ has all finite products. \textcolor{NavyBlue}{Note that we wouldn't have been able 
    to construct this if we didn't have a terminal object; For example, if $\cc$ 
    was a discrete category, we wouldn't even have any morphisms to take a pullback on!}

    Now to derive equalizers, consider a pair of parallel morphisms 
    $f, g: A \to B$. Then we may simply take their pullback to obtain the diagram below. 
    \begin{center}
        \begin{tikzcd}[column sep =  1.4cm, row sep = 1.4cm]
            C \arrow[drr, bend left, "f"]
            \arrow[ddr, bend right, "g"]
            \arrow[dr, dashed, "h"]
            &[-0.7cm]
            &
            \\[-1.2cm]
            &[-1cm]
            P
            \arrow[r, "p_1"]
            \arrow[d, swap, "p_2"]
            &
            A
            \arrow[d, "f"]\\
            &[-1cm]
            A
            \arrow[r, swap, "g"] 
            &
            B 
        \end{tikzcd}
    \end{center}
    If $p: A \times A \to A$ is the natural projection map, then 
    because we have a trivial mapping $1_A: A \to A$, there exists a canonical 
    canonical map $i: A \to A \times A$ such that $p \circ i = 1_A$. 
    Similarly, because we have mappings $p_1, p_2: P \to B$, we must have a 
    mapping $h: P \to A \times A$. 
    \begin{center}
        \begin{tikzcd}[column sep =  1.4cm, row sep = 1.4cm]
            &
            A
            \arrow[dr, "1_A"]
            \arrow[dl, swap, "1_A"]
            \arrow[d, dashed, "i"]
            &
            \\
            A
            &
            A\times A
            \arrow[l, "p"]
            \arrow[r, swap, "p"]
            &
            A
        \end{tikzcd}
        \hspace{1cm}
        \begin{tikzcd}[column sep =  1.4cm, row sep = 1.4cm]
            &
            P
            \arrow[dr, "p_1"]
            \arrow[dl, swap, "p_2"]
            \arrow[d, dashed, "h"]
            &
            \\
            A
            &
            A\times A
            \arrow[l, "p"]
            \arrow[r, swap, "p"]
            &
            A
        \end{tikzcd}
    \end{center}
    Now we can take the pullback on the morphism $h: P \to A \times A$ 
    and $i: A \to A \times A$ to obtain the equalizer.  
    \begin{center}
        \begin{tikzcd}[column sep =  1.4cm, row sep = 1.4cm]
            C \arrow[drr, bend left, "f"]
            \arrow[ddr, bend right, "g"]
            \arrow[dr, dashed, "h"]
            &[-0.7cm]
            &
            \\[-1.2cm]
            &[-1cm]
            E
            \arrow[r, "p_1"]
            \arrow[d, swap, "p_2"]
            &
            P
            \arrow[d, "h"]\\
            &[-1cm]
            A
            \arrow[r, swap, "i"] 
            &
            A \times A 
        \end{tikzcd}
    \end{center}

    Hence we see that for finite limits, we can reduce our assumptions to pullbacks and 
    a terminal object, giving rise to the final corollary. 
    \begin{thm}
        If a category has pullbacks and a terminal object, then it has all finite limits. 
    \end{thm}

    \newpage
    \section{Preservation of Limits}
   
    \begin{definition}
        Let $F: J \to C$ be a diagram and suppose 
        $G: \cc \to \dd$ is a functor. If for every limit
        $\Lim F$ exists in $\cc$ with morphisms $u_i: C \to F_i$,
        we say $G$ \textbf{preserves limits} if 
        $G(\Lim F)$ is 
        a limit with morphisms $G(u_i): G(C) \to G(F_i)$. Moreover, 
        we call such a functor a \textbf{continuous functor}.
    \end{definition}

    As an immediate consequence of the definition, it should be noted 
    that a composition of continuous functors is continuous. 

    Below we see a visual definition of a continuous functor. 
    \begin{center}
        \begin{tikzcd}[column sep = 1.4cm, row sep = 1.4cm]
            & C 
            \arrow[d, dashed, "h"] 
            \arrow[ddl,swap, bend right = 20,"f_i"] 
            \arrow[ddr, bend left = 20, "f_j"] &\\
            & 
            \arrow[dl, swap, "u_i"] \Lim F 
            \arrow[dr, "u_j"] & \\
            F(i)
            \arrow[rr, "F(g)"] 
            & 
            &
            F(j) 
        \end{tikzcd}
        \hspace{0.5cm}
        \begin{tikzcd}[column sep = 1.4cm, row sep = 1.4cm]
            & D \arrow[d, dashed, "h'"] 
            \arrow[ddl,swap, bend right = 20, "g_i"] 
            \arrow[ddr, bend left = 20, "g_j"] &\\
            & \arrow[dl, swap, "G(u_i)"] 
            G(\Lim F)
            \arrow[dr, "G(u_j)"] & \\
            G(F(i))
            \arrow[rr, "G(F(g))"] 
            & 
            &
            G(F(j)) 
        \end{tikzcd}
    \end{center}
    There's one particular and important functor which is always continuous 
    in any category. 

    \begin{thm}
        Let $\cc$ be a small category. Then for each $C \in \cc$, 
        the functor 
        \[
            \hom_{\cc}(C, -): \cc \to \textbf{Set}
        \]
        preserves limits. (Dually, the functor 
        $\hom_{\cc}(-, C) = \hom_{\cc}(C, -):
        \cc\op \to \textbf{Set}$ takes colimits to limits.)
    \end{thm}

    \begin{prf}
        Let $F: J \to \mathcal{C}$ be a diagram with a limiting object  
        $\text{Lim } F$ equipped with the morphisms $\sigma_i: \text{Lim } F \to F_i$.
        Then applying the $\text{Hom}_{\mathcal{C}}(C, -)$ functor to $\text{Lim } F$ and to 
        each $u_i$, we realize it forms a cone in $\textbf{Set}$. 
        \begin{center}
            \begin{tikzcd}[column sep = 1.4cm, row sep = 1.4cm]
                & \text{Lim } F 
                \arrow[dr, RoyalBlue, "\sigma_j"] 
                \arrow[dl, swap,RoyalBlue, "\sigma_i"]
                &\\
                F_i \arrow[rr, swap, "u"]&& F_j
            \end{tikzcd}
            \begin{tikzcd}[column sep = 0.5cm, row sep = 1.4cm]
                & \text{Hom}_{\mathcal{C}}(C, \text{Lim } F)
                \arrow[dr, RoyalBlue, "{\sigma_j}_*"] 
                \arrow[dl, swap, RoyalBlue, "{\sigma_i}_*"]
                &\\
                \text{Hom}_{\mathcal{C}}(C, F_i) \arrow[rr,swap, "u_*"]&& \text{Hom}_{\mathcal{C}}(C, F_j)
            \end{tikzcd}
        \end{center}
        Now we show that $\text{Hom}_{\mathcal{C}}(C, \text{Lim } F)$, equipped with the morphisms 
        $\sigma_{i*}$, is a universal cone; that is, it is a limit. 
        Suppose that $X$ is a set which forms a cone with the
        morphisms $\tau_i: X \to \text{Hom}_{\mathcal{C}}(C, F_i)$. 
        \begin{center}
            \begin{tikzcd}[column sep = 1.4cm, row sep = 1.4cm]
                & X 
                \arrow[dr, Red, "\tau_j"] 
                \arrow[dl, swap,Red, "\tau_i"]
                &\\
                \text{Hom}_{\mathcal{C}}(C,F_i) \arrow[rr, swap, "u_*"] && \text{Hom}_{\mathcal{C}}(C,F_j)
            \end{tikzcd}
        \end{center}
        
        Then for each $x \in X$, 
        we see that $\tau_i(x) : C \to F_i$.
        The diagram above tells us that $u \circ \tau_i(x) = \tau_j(x)$ for each $x$.
        Hence each $x \in X$ induces a 
        cone with apex $C$ with morphisms $\tau_i(x): C \to F_i$. 
        \begin{center}
            \begin{tikzcd}[column sep = 1.4cm, row sep = 1.4cm]
                & C 
                \arrow[dl, swap,Red, "\tau_i(x)"]
                \arrow[dr, Red, "\tau_j(x)"] 
                &\\
                F_i \arrow[rr, swap, "u"] && F_j
            \end{tikzcd}
        \end{center}
        However, $\text{Lim } F$ is the limit of $F: J \to \mathcal{C}$. Therefore, there 
        exists a unique arrow $h_x: C \to \text{Lim } F$ such that 
        $h_x \circ \sigma_i = \tau_i(x)$. Now we can uniquely
        define a function $: X \to \text{Hom}_{\mathcal{C}}(C, \text{Lim } F)$ where 
        $h(x) = h_x: C \to \text{Lim } F$, in such a way that the diagram below commutes.  
        \begin{center}
            \begin{tikzcd}[column sep = 1.4cm, row sep = 1.4cm]
                & X 
                \arrow[d, dashed, "h"] 
                \arrow[ddl,swap, bend right = 20,Red, "\tau_i"] 
                \arrow[ddr, bend left = 20, Red, "\tau_j"] &\\
                & 
                \text{Hom}_{\mathcal{C}}(C, \text{Lim } F)
                \arrow[dl, swap, RoyalBlue, "{\sigma_i}_*"] 
                \arrow[dr, RoyalBlue, "\sigma_{j_*}"] & \\
                \text{Hom}_{\mathcal{C}}(C, F_i)
                \arrow[rr,swap, "u_*"] 
                & 
                &
                \text{Hom}_{\mathcal{C}}(C, F_j)
            \end{tikzcd}
        \end{center}
        Therefore, $\text{Hom}_{\mathcal{C}}(C, \text{Lim } F)$ is a limit in \textbf{Set}.
    \end{prf}
    At this point, you may be wondering: What is the difference between 
    a functor which "creates limits" and one which preserves them? 
    We'll see that their definitions are different, but creating limits 
    is the same as preserving them 

    \begin{thm}
        Suppose $G: \cc \to \dd$ creates limits for $F: J \to \cc$. 
        If $G \circ F: J \to \dd$ has a limit in $\dd$, then 
        $G$ is continuous. 
    \end{thm}

    \begin{prf}
        Suppose $F: J \to \cc$ has limit $\Lim F$ in $\cc$ with morphisms 
        $v_i: \Lim F \to F_i$ for each $i \in J$. 
        Further, suppose $G \circ F: J \to \dd$ has a limit 
        $\Lim G \circ F$ with morphisms $u_i: \Lim G \circ F
        \to G\circ F_i$. 

        Since $G: \cc \to \dd$ creates limits, this implies 
        the existence of a limiting object $X$ with morphisms 
        $\sigma_i: X \to F_i$ for $F: J \to C$ 
        where $G(X) = \Lim G\circ F$ and $G(\sigma_i) = u_i$. 
        However, limiting objects are unique (by their universal properties).  
        As they must be isomorphic, there exists an isomorphism 
        $\phi: X \to \Lim F$ for which $v_i \circ \phi= \sigma_i$. 
        Thus we see that 
        \[
            G(\Lim F) \cong G(X) = \Lim G \circ F \qquad 
            G(v_i \circ \phi) = G(\sigma_i) = u_i.
        \]
        Therefore, $G$ preserves limits and so is continuous. 
    \end{prf}

    We have the following as a corollary. 

    \begin{corollary}
        Suppose $G: \cc \to \dd$ creates limits and $\cc$ is complete. 
        Then $\dd$ is complete and $G$ preserves limits. 
    \end{corollary}

    

    \newpage
    \section{Adjoints on Limits}

    Consider the free monoid functor $F$ and the 
    forgetful functor $U$, as below. Recall that they form an adjunction.
    \begin{center}
        \begin{tikzcd}
            \textbf{Set}
            \arrow[r, shift left = 0.5ex, "F"]
            &
            \textbf{Mon}
            \arrow[l, shift left = 0.5ex, "U"]
        \end{tikzcd}
    \end{center}
    The way that we philosophically 
    interpret this adjunction is as follows: For a set $X$, a monoid homomorphism
    $\phi: F(X) \to M$ gives rise to a unique set function 
    $f: X \to U(M)$. Conversely, a set function $g: X \to U(M)$ 
    gives rise to a unique monoid homomorphism $\psi: F(X) \to M$. 

    We will now observe that these functors exhibit nice behavior. 
    \begin{itemize}
        \item 
        Recall that products in $\textbf{Mon}$ are simply products of monoids, while 
        products in \textbf{Set} are cartesian products. One can show that, for two monoids 
        $M$, $N$, we have the isomorphism 
        \[
            U(M \times N) \cong U(M) \times U(N).
        \]
        Regarding this functor's behavior, 
        we say that the forgetful functor $U$ \textbf{preserves} products. 

        \item We may ask if the converse holds: Does the free functor preserve products? 
        The answer is no: Given two sets $X, Y$, it is generally not true that 
        $F(X \times Y) \cong F(X) \times F(Y)$ (as monoids). 

        An easy way to see this is to let $X = Y = \{\bullet\}$, the one point set. Then 
        $F(\{\bullet\} \times \{\bullet\}) \cong F(\{\bullet\}) \cong \zz$, while 
        $F(\{\bullet\}) \times F(\{\bullet\}) \cong \zz \times \zz$. 

        \item What is interesting, however, is that the free functor \emph{does} preserve 
        coproducts. Recall that the coproduct in \textbf{Set} is the disjoint union, while the 
        coproduct in \textbf{Mon} is the free product of monoids. Then it is true that, for two 
        sets $X, Y$, 
        \[
            F(X \amalg Y) \cong F(X) * F(Y).
        \]
    \end{itemize}
    Thus we see that we have two functors that separately preserve 
    products and coproducts. This is actually very interesting; after all, a 
    very useful question to ask about a functor is if it preserves products, coproducts, equalizers, 
    etc. For example, the fundamental group functor preserves products, and this is an interesting result 
    one usually proves a topology course. 
    
    We now explain why we have this nice behavior.
    \begin{thm}\label{theorem:RAPL}
        Suppose $G: \dd \to \cc$ is a right adjoint and $F: \cc \to \dd$ is its left adjoint.
        Then $G$ preserves limits and $F$ preserves colimits.
    \end{thm}

    Before a proof, we make some comments. 
    \begin{itemize}
        \item An easy way to remember this is \textbf{RAPL}: ``\textbf{R}ight \textbf{A}djoints \textbf{P}reserve \textbf{L}imits.'' 
        (Speaking from experience, say it in your head a bunch of times or you'll forget.) 
        If you can remember \textbf{RAPL}, then you can 
        remember that, dually, left adjoints preserve colimits. 

        \item The converse of this theorem does not hold.
        
        \item Typically, this proof is shown in one of two forms: It is ``blackboxed'' with 
        a slick application of the Yoneda Lemma, which is not illuminating or useful for a new 
        reader. Or, it is more usefully spelled out by showing that right adjoints preserve limits, and 
        the second statement is obtained by ``dualizing''. For variety, we will
        show that left adjoints preserve 
        colimits. Then, the reader can 
        try proving themselves that right adjoints preserve 
        limits.
    \end{itemize}

    \begin{prf}
        Let $(\Colim H, \sigma_i: H(i) \to \Colim H)$ be the colimit
        of the functor $H: J \to \cc$. This means that we have the universal diagram below. 
        \begin{center}
            \begin{tikzcd}[column sep = 1cm, row sep = 1.4cm]
                & A
                \arrow[<-, ddl,swap, bend right = 20,"\gamma_i"] 
                \arrow[<-, ddr, bend left = 20, "\gamma_j"] 
                &
                \\
                & 
                \Colim H
                \arrow[u, dashed]
                \arrow[<-, dl, swap, "\sigma_i"] 
                \arrow[<-, dr, "\sigma_j"] 
                & 
                \\
                H(i)
                \arrow[rr, swap, "H(f)"] 
                & 
                &
                H(j)
            \end{tikzcd}
        \end{center}        
        Mapping this to $\dd$ under $F: \cc \to \dd$, we obtain the diagram 
        \begin{center}
            \begin{tikzcd}[column sep = 1cm, row sep = 1.4cm]
                & 
                F(\Colim H)
                \arrow[<-, dl, swap, "F(\sigma_i)"] 
                \arrow[<-, dr, "F(\sigma_j)"] 
                & 
                \\
                F(H(i))
                \arrow[rr, swap, "F(H(f))"] 
                & 
                &
                F(H(j))
            \end{tikzcd}
        \end{center}    
        We see that $(F(\Colim H), F(\sigma_i): F(H(i)) \to F(\Colim H))$ 
        is a cone over the functor $F \circ H: J \to \dd$.
        We must show it is universal. Towards that goal, 
        let $(C, \tau_i: F(H(i)) \to C)$ 
        be a cone over $F \circ H: J \to \dd$. 
        We must show that 
        \begin{itemize}
            \item[1.] There exists a $\alpha: F(\Colim H) \to C$ such that 
            $\alpha \circ F(\sigma_i) = \tau_i$ for all $i \in J$
            \item[2.] $\alpha$ is the unique morphism from $F(\Colim H)$ to $C$ with 
            this property.  
        \end{itemize}
        We show existence. Observe that each 
        $\tau_i: F(H(i)) \to C$ induces a unique morphism 
        $\delta_i: H(i) \to G(C)$ such that the diagram below commutes. 
        \begin{center}
            \begin{tikzcd}[column sep = 1.4cm, row sep = 1.4cm]
                F(G(C))
                \arrow[r, "\epsilon_C"]
                &
                C
                \\
                F(H(i))
                \arrow[u, dashed, "\delta_i"]
                \arrow[ur, swap, "\tau_i"]                
            \end{tikzcd}
        \end{center}
        Hence, we have a family of $\delta_i: H(i) \to G(C)$. However, since $\Colim H$ is the colimit of $H$, 
        we obtain a unique morphism $k: \Colim H \to G(C)$ such that the diagram commutes.     
        \begin{center}
            \begin{tikzcd}[column sep = 1cm, row sep = 1.4cm]
                & G(C)
                \arrow[<-, ddl,swap, bend right = 20,"\delta_i"] 
                \arrow[<-, ddr, bend left = 20, "\delta_j"] 
                &
                \\
                & 
                \Colim H
                \arrow[u, dashed, "k"]
                \arrow[<-, dl, swap, "\sigma_i"] 
                \arrow[<-, dr, "\sigma_j"] 
                & 
                \\
                H(i)
                \arrow[rr, swap, "H(f)"] 
                & 
                &
                H(j)
            \end{tikzcd}
        \end{center}   
        We then map this diagram in $\cc$ to the diagram below in $\dd$ 
        via $F$:
        \begin{center}
            \begin{tikzcd}[column sep = 1cm, row sep = 1.4cm]
                &
                C
                &
                \\
                &
                F(G(C))
                \arrow[u, "\epsilon_C"]
                &
                \\
                & 
                F(\Colim H)
                \arrow[u, "F(k)"]
                \arrow[<-, dl, swap, "F(\sigma_i)"] 
                \arrow[<-, dr, "F(\sigma_j)"] 
                & 
                \\
                H(i)
                \arrow[rr, swap, "F(H(f))"] 
                \arrow[uur, bend left = 20,"F(\delta_i)"]
                \arrow[uuur, bend left = 40,"\tau_i"]
                & 
                &
                H(j)
                \arrow[uul, bend right = 20, swap, "F(\delta_j)"]
                \arrow[uuul, bend right = 40, swap, "\tau_j"]
            \end{tikzcd}
        \end{center}  
        Thus we see that $\epsilon_C \circ F(k): F(\Colim H) \to C$ is a morphism 
        pointing from $F(\Colim)$ to $C$ such that the above diagram commutes. 
        We have proved existence of such a morphism. It is not difficult to show 
        uniqueness, which is left as an exercise. 
        Once we have uniqueness, we can then conclude that $(F(\Colim H), F(\sigma_i): F(H(i)) \to F(\Colim H)$ 
        forms a universal cone 
        over $F \circ H: J \to \dd$, so that $F(\Colim H)$ is the colimit, as desired. 
    \end{prf}

    \begin{example}
        Using the above theorem, we now know that the free monoid functor 
        $F: \textbf{Set} \to \textbf{Mon}$ preserves coproducts. Therefore, we can say 
        that for any sets $X, Y$, we have that 
        \[
            F(X \amalg Y) \cong F(X) * F(Y).   
        \]
        Moreover, the free monoid functor is part of a larger family of free functors: 
        \begin{itemize}
            \item Free group functor, $F: \textbf{Set} \to \textbf{Grp}$
            \item Free abelian group functor, $F: \textbf{Set} \to \textbf{Ab}$
            \item Free ring functor, $F: \textbf{Set} \to \textbf{Ring}$
            \item Free $R$-module functor, $F: \textbf{Set} \to R\textbf{-Mod}$
        \end{itemize}
        who are the left adjoints to their respective forgetful functors. 
        However, the coproduct in some of these 
        categories is not always the free product. For example, the coproduct of $\textbf{Grp}$ 
        is the free product, but the coproduct in $\textbf{Ab}$ is the direct sum.
        Hence, the above theorem tells us that coproducts are preserved, but to obtain 
        the correct isomorphism, we need to remember what the coproduct in the codomain category 
        of our left adjoint is.
    \end{example}


    \begin{example}
        Let \textbf{Meas} be the category of measure spaces with measure-preserving 
        morphisms. 
        More precisely, 
        \begin{description}
            \item[Objects.] The objects are triples $(X, \mathcal{A}, \mu_X)$ 
            where $X$ is a topological space, $\mathcal{A}$ is a sigma algebra 
            on $X$, and $\mu_X$ is a measure on $X$. 
            
            \item[Morphisms.] A morphism between two objects $(X, \mathcal{A}, \mu_X)$ 
            and $(Y, \mathcal{B}, \mu_Y)$ is a function $f: X \to Y$ such that 
            $f$ is measurable and preserves measure. That is, is $f$ is measurable 
            and 
            \[
                \mu_X(f^{-1}(B)) = \mu_Y(B)
            \] 
            for every $B \in \mathcal{B}$. 
        \end{description}

        Let $U: \textbf{Meas} \to \textbf{Set}$ be the forgetful functor, forgetting 
        measure space properties and measurability of the morphisms. 
        This functor can't have a left-adjoint, since it does not preserve 
        products. In fact, \textbf{Meas} cannot even have products. 
        The main issue with this is that we cannot guarantee the projection 
        morphisms to preserve measure. For example, if we consider the 
        simple measure space $(\mathbb{R}, \mathcal{B}, \mu)$ where $\mathcal{B}$ consists 
        of the Borel algebra and $\mu$ is the Lebesgue measure, then 
        one reasonable way to try to form a product with itself is to construct the triple
        \[
            (\rr \times \rr, \mathcal{B} \times \mathcal{B}, \mu\times\mu). 
        \]
        However, observe that the projection $\pi: (\rr \times \rr, \mathcal{B} \times \mathcal{B}, \mu\times\mu) \to (\rr, \mathcal{B}, \mu)$ is 
        not measure preserving:
        \[
            \mu \times \mu(\pi^{-1}([0, 1])) = \mu \times \mu([0, 1] \times \rr) = \infty
        \]
        while 
        \[
            \mu([0, 1]) = 0.            
        \]
        Therefore, we cannot form products. Hence our forgetful functor 
        has no left adjoint. 
        
        One could guess that the left adjoint \textit{would} be the 
        measure-constructing functor $F: \textbf{Set} \to \textbf{Meas}$ where 
        \[
            X \mapsto (X, \mathcal{P}, \mu_0)
        \]
        where $\mathcal{P}$ is the sigma algebra on the power set, and $\mu_0$ assigns
        the measure of each set to zero (i.e. the trivial measure) but this is not
        the case. In fact, this functor itself also cannot have a left-adjoint 
        because it doesn't preserve products 
        (since \textbf{Meas} can't have products).


    \end{example}

    {\large \textbf{Exercises}
    \vspace{0.2cm}}
    \begin{itemize}
        \item[\textbf{1.}] Denote the free monoid functor as $F$.
        Prove directly that for two sets $X$, $Y$, we have 
        the isomorphism of monoids $F(X \amalg Y) \cong F(X) * F(Y)$. (Doing this is actually very important; 
        The proof of Theorem \ref{theorem:RAPL} will become more intuitive.)
        \item[\textbf{2.}] Finish the proof of Theorem \ref{theorem:RAPL}
        \item[\textbf{3.}] 
        Let $\cc, \dd$ be categories with finite products. 
        \begin{itemize}
            \item[\emph{i.}] Let $F: \cc \to \dd$ be a functor that preserves products, so that
            for two objects $A$, $B$ of $\cc$, there exists an isomorphism 
            \[
                F(A \times B) \cong F(A) \times F(B).
            \]
            Does this isomorphism have to be natural in $A, B$?

            \item[\emph{ii.}] Suppose $F: \cc \to \dd$ is a right adjoint. Is the isomorphism 
            $F(A \times B) \cong F(A) \times F(B)$ natural now? 
        \end{itemize}
    \end{itemize}


    

    \newpage 
    \section{Existence of Universal Morphisms and Adjoint Functors}
    When we introduced functors, we introduced several if and only 
    if propositions which gave us criterion on the existence of an adjoint 
    functor. Notably, we showed that if there exists an adjunction 
    \begin{center}
        \begin{tikzcd}[column sep = 1.4cm, row sep = 1.4cm]
            C \arrow[r, shift right = 0.5ex, swap, "G"]
            &
            D \arrow[l, shift left = -0.5ex, swap, "F"] 
        \end{tikzcd}
    \end{center}
    (that is, the classic bijection of homsets which is natural)
    then there exist universal morphisms 
    \[
        \eta_C: C \to G \circ F(C) \qquad \epsilon_D: F\circ G(D) \to D 
    \]
    for all objects $C, D$. Furthermore, we only need one of the universal morphisms 
    to derive an adjunction. Since universal morphisms are simply initial objects 
    in some comma category, we have the following proposition. 

    \begin{proposition}
        Let $G: \dd \to \cc$ be a functor. Then $G$ has a left adjoint 
        if and only if for each $C \in \cc$, the comma category $C \downarrow G$ 
        has an initial object.
    \end{proposition}

    \begin{prf}
        \begin{itemize}
            \item[$\bm{\implies}$]
            Suppose $G$ has a left adjoint $F: \cc \to \dd$. Then 
            for each $C \in \cc$, there exists a universal morphism 
            $\eta_C: C \to G(F(C))$. 
            Now in the comma category, objects will be of 
            the form 
            \[
                (D, f: C \to G(D))
            \]
            where morphisms between $(D, f: C \to G(D))$ and $(D', f': C \to G(D'))$ 
            will be induced by morphisms $h: D \to D'$ such that 
            \begin{center}
                \begin{tikzcd}[column sep = 1.4cm, row sep = 1.4cm]
                    & C \arrow[dl, swap, "f"] \arrow[dr, "f'"] & \\
                    G(D) \arrow[rr, swap, "G(h)"] & & G(D')
                \end{tikzcd}
            \end{center}
            commutes. First, observe that $(F(C), \eta_C: C \to G(F(C)))$ 
            is an object of the comma category. Second, observe that the 
            bijection of homsets 
            \[
                \hom_{\dd}(F(C), D) \cong \hom_{\cc}(C, G(D))   
            \]
            (natural in $C, D$) 
            guarantees that every object
            $(D, f: C \to G(D))$ in the comma category corresponds uniquely to
            a morphism $h: F(C) \to D$. Moreover, uniqueness guarantees that the diagram 
            \begin{center}
                \begin{tikzcd}[column sep = 1.4cm, row sep = 1.4cm]
                    & C \arrow[dl, swap, "\eta_C"] \arrow[dr, "f"] & \\
                    G(F(C)) \arrow[rr, swap, "G(h)"] & & G(D)
                \end{tikzcd}
            \end{center}
            must commute. Hence, $(F(C), \eta_C: C \to G(F(C)) )$ is an 
            initial object $C \downarrow G$. 

            \item[$\impliedby$] Now suppose that $C \downarrow G$ has an 
            initial object $(D, \eta_C: C \to G(D))$. Actually, denote 
            the object $D$ as $F(C)$. When we write $F(C)$, we're not denoting 
            a functor, because we'll show this is a functor. Anyways, our initial 
            object can be written as 
            \[
                (F(C), \eta_C: C \to G(F(C))).
            \]
            This defines a mapping on objects $C \mapsto F(C)$. To show that this 
            is a functor, suppose we have a morphism $f: C \to C'$ in $\cc$. Then 
            we have square 
            \begin{center}
                \begin{tikzcd}[column sep = 1.4cm, row sep = 1.4cm]
                    C \arrow[r, "\eta_C"] \arrow[d, swap, "f"] & G(F(C))\\
                    C' \arrow[r, "\eta_{C'}"] & G(F(C')). 
                \end{tikzcd}
            \end{center}
            Adding the final leg to this diagram would show that $F$ is a functor. But 
            since $(F(C), \eta_C: C \to G(F(C)))$ is an initial object in $(C \downarrow G)$, 
            and $(F(C'), \eta_{C'}: C' \to G(F(C)))$ is an object in this category, 
            there must be a \textit{unique} morphism $F(f):F(C) \to F(C')$. Uniqueness of 
            this morphism forces commutativity of the square 
            \begin{center}
                \begin{tikzcd}[column sep = 1.4cm, row sep = 1.4cm]
                    C \arrow[r, "\eta_C"] \arrow[d, swap, "f"] & G(F(C)) \arrow[d, "G(F(f))"]\\
                    C' \arrow[r, "\eta_{C'}"] & G(F(C')). 
                \end{tikzcd}
            \end{center}
            and therefore $F$ is a functor. Simultaneously, this shows $F$ is left adjoint 
            to $G$, as desired. 
        \end{itemize}
    \end{prf}  
    We can repeat the proof to achieve the following result as well.
    \begin{corollary}
        Let $F: \cc \to \dd$ be a functor. Then $F$ has a right adjoint if and only 
        if for each $D \in D$, the comma category $D \downarrow F$ has a terminal 
        object. 
    \end{corollary}
    Thus we see that initial and terminal objects are key to figuring out 
    when a functor has a left or right adjoint, and hence when they preserve limits.
    We can investigate a little deeper into this. 

    \begin{lemma}(Initial Object Existence.)
        If $\cc$ is a complete category with small homsets, then $\cc$ 
        has an initial object if and only if it satisfies the \textbf{Solution 
        Set Condition}: 
        \begin{center}
            \begin{minipage}{0.9\textwidth}
                There exists
                objects $(C_i)_{i \in I} \in \cc$ such that for every $C \in \cc$, there is a 
                a morphism $f_i: C_i \to C$ for at least one $i \in I$. 
            \end{minipage}
        \end{center}

    \end{lemma}

    \begin{prf}
        \begin{itemize}
            \item[$\implies$] 
             Suppose $\cc$ has an initial object $C'$. Then $I$ is the one-point 
             set since for each $C \in C$ there exists one unique morphism 
             $f: C' \to C$. 

            \item[$\impliedby$] 
            On the other hand, assume the solution set condition. 
            Since $\cc$ is complete, it must have products, so we may 
            take the product
            \[
                W = \prod_{i \in J}C_i.
            \]
            This product has associated projection morphisms 
            $\displaystyle \pi_k: \prod_{i \in J}C_i \to C_k$. Therefore, 
            for each object $C \in \cc$, there exists at least one 
            morphism between $W$ and $C$ by composition:
            \[
                f_k \circ \pi_k: W \to C.
            \]
            By hypothesis, the collection of endomorphisms $\hom_{\cc}(W, W)$ 
            is a set. Therefore, we may form an equalizer $e: V \to W$ 
            of this set. Observe that for each $C \in \cc$, there exists 
            at least one morphism between $V$ and $C$ by composition:
            \[
                f_k \circ \pi_k\circ e: V \to C.
            \]
            We'll now show that all morphisms are equal. Suppose the contrary; 
            that there are two distinct morphisms $f, g: V \to C$. Denote the equalizer 
            of this pair as $e_1: u \to v$. Then we have that 
            \begin{center}
                \begin{tikzcd}[column sep = 1.4cm, row sep = 1.4cm]
                    U \arrow[r, "e_1"] 
                    & 
                    V 
                    \arrow[r, shift right = 0.5ex, "f"] 
                    \arrow[d, "e"] 
                    & C\\
                    W\arrow[u, dashed, "s"] 
                    \arrow[r, swap, "e\circ e_1 \circ s"] 
                    & 
                    W = \prod_{i \in J}C_i 
                    \arrow[r, swap, "\pi_k"]
                    &
                    C_k
                    \arrow[u, swap, "f_i"]
                \end{tikzcd}
            \end{center}
            commutes. The morphism $s$ is induced via the universality of 
            both $U$ and $V$. Since $e \circ e_1 \circ s : W \to W$, and 
            $e$ is the equalizer of endomorphisms of $W$, we have that 
            \[
                (e \circ e_1 \circ s)\circ e = e.  
            \]
            Since equalizers are monic, we can cancel on the left side to conclude 
            that 
            \[
                e_1 \circ s \circ e = 1_V. 
            \]
            However, this implies that the right inverse of $e_1$ is $s \circ e$. 
            Since $e_1$ is already monic, it must be an isomorphism. Hence $f = g$, so that 
            $V$ is an initial object as desired. 
        \end{itemize}
    \end{prf}

    We can now combine all of our propositions and theorems into the following 
    one, which is the main adjoint functor theorem of interest. 

    \begin{thm}[ (General Adjoint Functor Theorem.)]
        Let $\dd$ be complete with small homsets. A functor 
        $G: \dd \to \cc$ has a left adjoint if and only if it preserves all small 
        limits and satisfies the \textbf{solution set condition}: 
        \begin{center}
            \begin{minipage}{0.9\textwidth}
                For each $C \in \cc$, 
                there exists a set of objects $(D_i)_{i \in I}$ $\dd$ and a family 
                of arrows
                \[
                    f_i: C \to G(D_i)
                \]
                such that for every morphism $h: C \to G(D)$, there 
                exists a $j \in I$ and a morphism $t: D_j \to D$ such that
                \[
                    h = G(t) \circ f_i.
                \]
            \end{minipage}  
        \end{center}
    \end{thm}

    \textcolor{Plum}{The above theorem helps us find out when we can get a 
    left adjoint. Prior to this theorem, we already know what happened if 
    we were given a functor who has a left adjoint. Namely, it must preserve 
    limits. This natural question one would then ask is if
    the converse holds. The above theorem tells us no, the converse doesn't hold 
    and in fact we need to make sure the functor satisfies the \textbf{solution 
    set condition}. In the next section, we'll give an example of a functor 
    which preserves limits from a complete category, but still has no left adjoint.}

    As a converse to the above theorem, we have the following. 

    \begin{thm}[ (Representability Theorem.)]
        Let $\cc$ be a small, complete category. A functor $K: \cc \to 
        \textbf{Set}$ is representable if and only if $K$ preserves limits 
        and satisfies the following \textbf{solution set condition}: 
        \begin{center}
            \begin{minipage}{0.9\textwidth}
                There exists a set $S \subset \ob(\cc)$ such that for any 
                $C \in \cc$ and any $x \in K(C)$, there exists an $s \in S$, 
                an element $y \in K(s)$ and an arrow 
                \[
                    f: s \to C \text{ such that } K(f)(y) = x.
                \]
            \end{minipage}
        \end{center}
    \end{thm}

\newpage
\section{Subobjects and Quotient Objects}
The entire point of category theory, contrary to its name, is to unify 
mathematics. Mathematicians saw the same stories over and over again in algebra 
and topology, and one day they got sick of it and decided to start naming the patterns 
they were seeing.
Mathematicians achieved a level of abstraction where we no longer really care about the objects, 
but we want to study the morphisms between them. 
However, in many categories, the objects are often things like groups, 
rings, or topological spaces; hence there are subgroups, subrings, 
and spaces with subset topologies which also exist inside categories we study. 
This presents a challenge for category theory: how do we generalize the notion of 
subgroups or subspaces if we always avoid explicit reference to the elements?

It turns out that the correct way to go about this is to consider the philosophy of 
sub-"things": whenever $S$ is a sub-"thing" of $X$, there usually exists a monomorphism
\[
    m: S \to X.
\] 
For example, in \textbf{Set}, $S \subset X$ implies that there's an injection $i: S \to X$; a monomorphism 
is injective in \textbf{Set}, so this makes sense. In \textbf{Top}, if $S \subset X$ where $S$ is given the subspace 
topology, then the inclusion function $i: S \to X$ is continuous, so there does exist 
a monomorphism $m: S \to X$ in \textbf{Top}. 

Thus we see that these monomorphisms give us sub-"things," and so we might naively say 
the set of all "subobjects" of an object $X$ in a category $\cc$ is the set
\[
    \text{Sub}_{\cc}(X) = \{S \in \text{Ob}(\cc) \mid \exists f: S \to X \text{ with } f \text{ monic }\}.
\]
However, the space of all of 
these monomorphisms is huge, and also repetitive. For example, in \textbf{Set}, if we have 
$X = \{1, 2, 3, 4, 5\}$, then there are all kinds of monomorphisms into $X$:
\begin{center}
    \begin{tikzcd}[column sep = 0.7cm, row sep = 1.4cm]
        \{\text{! },  \text{ \% }, \text{  \$ }, \text{\& }\}
        \arrow[dr, hookrightarrow]
        &
        \\
        \{Q, G, X, I\}
        \arrow[r, hookrightarrow]
        &
        \{1, 2, 3, 4, 5\}
        \\
        \{\text{All }, \text{ Politicians }, \text{ Are }, 
        \text{ Corrupt }\}
        \arrow[ur, hookrightarrow]
    \end{tikzcd}
\end{center}
Each arrow is basically saying the same thing. How do we deal with this?
Well, we can impose an equivalence relation on this space to obtain something smaller
and more manageable.

Let $A$ an object of our category $\cc$. Consider monomorphisms $f: C \to A$ and 
$g: D \to A$. Define the relation $\le$ on monomorphisms of this form where 
\begin{statement}{ProcessBlue!10}
\begin{minipage}{0.6\textwidth}
    \[
        f \le g \text{ if there exists an } h \text{ where } f = g \circ h.
    \]
\end{minipage} 
\begin{minipage}{0.4\textwidth}
    \begin{tikzcd}[column sep = 1.6cm, row sep = 0.3cm]
        C \arrow[dr, hookrightarrow, "f"]
        &
        \\
        &
        A 
        \\
        D \arrow[ur,hookrightarrow, swap, "g"]
        \arrow[uu, dashed, "h"]
    \end{tikzcd}
\end{minipage}
\end{statement}
for some monomorphism $h: D' \to D$. Note that if $f \le g$ and $f \ge g$, 
then $C$ and $D$ are isomorphic (this is not true in general; this only true here 
because $f, g$ are monomorphisms).
So we now have our equivalence relation: we say $f \sim g$ if 
there exists an isomorphism $\phi: D \to C$ which makes the above diagram commute. 

\begin{definition}
    Let $\cc$ be a category and let $A$ be an object. We say a \textbf{subobject} 
    of $A$
    is an equivalence class of monomorphisms $f: S \to A$ under the equivalence relation 
    $\sim$. We denote this space of equivalence classes as 
    \[
        \text{Sub}_{\cc}(A) = \Big\{[f] \mid f:C \to A \text{ is a monomorphism} \Big\}.
    \]
\end{definition}

\begin{example}\label{subfunctor_example}
    Let $\cc$ be a category. An interesting application of subobjects occurs in functor categories. To illustrate this 
    we consider the functor category $\textbf{Set}^{\cc}$; that is, the category with functors 
    $F: \cc \to \textbf{Set}$ whose morphisms are natural transformation $\eta: F \to G$ between 
    such functors. 
    
    If we play around with these functors long enough, we may ask the question:
    What happens when, for a functor $F: \cc \to \textbf{Set}$, there is another 
    functor $G: \cc \to \textbf{Set}$ such that 
    \[
        G(A) \subset F(A)?
    \]
    Could we logically call $G$ a "\textbf{subfunctor}" of $F$? We could with a little more work. 
    Because $G(A) \subset F(A)$,  we know that there exists a monomorphism (just an injection here)
    $i_A: G(A) \to F(A)$. Now a natural question to ask here is if this translates to a
    natural transformation. That is, does the diagram below commute?
    \begin{center}
        \begin{tikzcd}[column sep = 1.4cm, row sep = 1.4cm]
            A 
            \arrow[d, "f"]\\
            B
        \end{tikzcd}
        \hspace{1cm}
        \begin{tikzcd}[column sep = 1.4cm, row sep = 1.4cm]
            G(A)
            \arrow[d, swap, "G(f)"]
            \arrow[r, swap, hookrightarrow, swap, "i_A"]
            &
            F(A)
            \arrow[d, "F(f)"]
            \\
            G(B)
            \arrow[r, hookrightarrow, swap, "i_B"]
            & 
            F(B)
        \end{tikzcd}
    \end{center} 
    The answer is no. This is because $G(f)$ and $F(f)$ could be two entirely different functions which do 
    two entirely different things to the same elements in different domains; however, 
    one way for this diagram to commute is if $G(f)$ is $F(f)$ \emph{restricted} to 
    the set $G(A)$. That is, if 
    \[
        G(f) = F(f)\big|_{G(A)}.
    \]
    The diagram then commutes. But is this the only way to make it commute? Suppose with no assumption of $G(f)$
    that the diagram did commute. Then we can still make a morphism $F(f)\big|_{G(A)}: G(A) \to G(B)$ to 
    get the commutative diagram 
    \begin{center}
        \begin{tikzcd}[column sep = 1.4cm, row sep = 1.4cm]
            A 
            \arrow[d, "f"]\\
            B
        \end{tikzcd}
        \hspace{1cm}
        \begin{tikzcd}[column sep = 1.4cm, row sep = 1.4cm]
            G(A)
            \arrow[d, shift right = -0.5ex, "G(f)"]
            \arrow[d, shift right = 0.5ex, swap, "F(f)\big|_{G(A)}"]
            \arrow[r, swap, hookrightarrow, swap, "i_A"]
            &
            F(A)
            \arrow[d, "F(f)"]
            \\
            G(B)
            \arrow[r, hookrightarrow, swap, "i_B"]
            & 
            F(B)
        \end{tikzcd}
    \end{center}    
    Then we see that $i_B \circ G(f) = i_B \circ F(f)\big|_{G(A)}$. However, 
    $i_B$ is a monomorphism,  so $G(f) = F(f)\big|_{G(A)}$. Hence the \emph{only} way to 
    make the diagram commute is if $G(f)$ is a restriction of $F(f)$. 

    Thus we could define $G: \cc \to \textbf{Set}$ to be a subfunctor of $F: \cc \to \textbf{Set}$ 
    if $G(A) \subset F(A)$ and $G(f: A \to B) = F(f)\big|_{G(A)}$. Or, equivalently, 
    if $G(A) \subset F(A)$ and that this relation is natural. 

    However, we can recover the same concept by applying subobjects to this functor category. 
    In this case, we can (with laziness) say a $G: \cc \to \textbf{Set}$ is a subobject 
    of the functor $F: \cc \to \textbf{Set}$ 
    in $\textbf{Set}^{\cc}$ if there exists a monic natural transformation 
    $\eta: G \to F$. 

    Unwrapping this definition, we see that a monic natural transformation in this 
    case is just one where each morphism $\eta_A: G(A) \to F(A)$ is a monomorphism, which, in our case, 
    just means an inclusion function, such that the necessary square commutes. However, we 
    already showed that we get the commutativity of the necessary square if and only if 
    $G(f: A \to B) = F(f)\big|_{G(A)}$. 
    
    Hence we have recovered the same concept of a \textbf{subfunctor} in two different ones; 
    one in which we followed our intuition, and one in which we blinded applied the concept of a 
    subobject in the functor category $\textbf{Set}^{\cc}$.
\end{example}

The previous example allows us to make the definition:
\begin{definition}
    Let $\cc, \dd$ be categories. Then a functor $G: \cc \to \dd$  
    is a \textbf{subfunctor} of $F: \dd \to \cc$ if $G$ is a subobject of $F$ 
    in the functor category $\dd^{\cc}$.
\end{definition}

Now, perhaps unsurprisingly, the entire process above can be dualized. When we dualize, 
however, we obtain a generalization of the concept of quotient objects. Instead of just dualizing and 
being boring, we'll motivate why we'd even care for such a dual concept. 
\\

In interesting categories such as \textbf{Ab} or \textbf{Top}, we not only have 
subgroups and subspaces, but we also have quotient groups and quotient spaces. 
For the case of abelian groups, we can, for any such group $G$, consider any 
subgroup $H \le G$ and construct the quotient group $G/H$. This comes with a 
a nice epimorphism $\pi: G \to G/H$ where $g \mapsto g + H$. 

For topological spaces $(X, \tau)$ in \textbf{Top}, we can define an equivalence 
relation $\sim$ on $X$ and consider the topological space $(X/\sim, \tau')$ 
such that $\tau'$ is the topology where a set $U$ is open if $\{x \mid [x] \in U\}$
is open in $\tau$. We can then equip ourselves with a continuous projection map 
$\pi: X \to X/\sim$, which is also an epimorphism. 

With these few examples, we see that it is worthwhile to generalize the concept 
of quotient objects; to do this however requires no explicit mention of the elements of 
the objects of the category. However, we can maintain the philosophy seen in the previous 
two examples to generalize the concept.

For an object $A$ in a category $\cc$, we consider all \emph{epimorphisms}
\[
    e: A \to Q
\]
and call objects such objects $Q$ as quotient objects. Again, the space of these 
objects is too large, so we instead consider ordering relation 

\begin{statement}{ProcessBlue!10}
\begin{minipage}{0.6\textwidth}
    \[
        f \le g \text{ if there exists an } h \text{ where } f = h \circ g.
    \]
\end{minipage} 
\begin{minipage}{0.4\textwidth}
    \begin{tikzcd}[column sep = 1.6cm, row sep = 0.3cm]
        & C 
        \\
        A 
        \arrow[ur, ->>, "f"]
        \arrow[dr, ->>, swap, "g"]
        &
        \\
        &
        D \arrow[uu, dashed, swap, "h"]
    \end{tikzcd}
\end{minipage}
\end{statement}
Observing that $f \le g$ and $g \le f$ together imply that $C \cong D$, we see that 
we may construct an equivalence relation $\sim$ where $f \sim g $ if there exists an isomorphism 
$\phi: D \to C$ such that $f = \phi \circ g$. We can now outline a clear definition.

\begin{definition}
    Let $\cc$ be a category and let $A$ be an object. We say a \textbf{quotient object} of 
    $A$ is an equivalence class of morphisms $f: A \to Q$. We then denote 
    \[
        \text{Quot}_\cc(A) = \Big\{ [f] \mid f: A \to Q \text{ is an epimorphism }\Big\}.
    \]
\end{definition}


\begin{example}
    A quotient object in \textbf{Cat} is a quotient category (from chapter 2)
\end{example}



